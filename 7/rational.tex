\section{Rational Homology and Euler Characteristics}
\subsection{Rational Homology}
When we defined the $n$-chains of simplicial homology, we took it as the free abelian group, i.e. free $\mathbb Z$-module, generated by the $n$-simplices.
But there is no reason why we should just restrict ourself to $\mathbb Z$-modules.
\footnote{Well, you have the universal coefficient theorem.}
If we replace $\mathbb Z$ by $\mathbb Q$, we get a vector space.
And the homology theory on this is called the rational homology.
\begin{definition}
    Let $K$ be a simplicial complex.
    Define the $\mathbb Q$-vector space of rational $n$-chains to be the $\mathbb Q$-vector space with basis being the set of $n$-simplices of $K$.
    Denote this space by $C_n(K;\mathbb Q)$.\\
    The boundary map $\partial_n:C_n(K;\mathbb Q)\to C_{n-1}(K;\mathbb Q)$, the cycles $Z_n(K;\mathbb Q)$, the boundaries $B_n(K;\mathbb Q)$ and the homology groups $H_n(K;\mathbb Q)$ are defined exactly as before.
\end{definition}
Correspondingly, by viewing abelian groups as $\mathbb Z$-modules, we can think of the objects $C_n(K),Z_n(K),B_n(K),H_n(K)$ appeared in our original simplicial homology theory instead as $C_n(K;\mathbb Z),Z_n(K;\mathbb Z),B_n(K;\mathbb Z),H_n(K;\mathbb Z)$.
This base ring (e.g. $\mathbb Z$ or $\mathbb Q$ as we have seen) is called the coefficient of the simplicial homology.\\
Perhaps unsurprisingly, homology with coefficient in $\mathbb Q$ actually contains less information than homology with coefficients in $\mathbb Z$.
\begin{lemma}
    If $H_n(K;\mathbb Z)\cong\mathbb Z^b\oplus F$ where $F$ is a finite abelian group, then $H_n(K;\mathbb Q)=\mathbb Q^b$.
\end{lemma}
\begin{proof}
    There is a natural map $C_n(K;\mathbb Z)\to C_n(K;\mathbb Q)$ via the inclusion of $\mathbb Z$ in $\mathbb Q$.
    This induces a chain map $C_\bullet(K;\mathbb Z)\to C_\bullet(K;\mathbb Q)$ and hence a natural homomorphism $H_n(K;\mathbb Z)\to H_n(K;\mathbb Q)$.
    If $c\in Z_n(K;\mathbb Q)$, then there is an integer $m$ such that $mc\in Z_n(K;\mathbb Z)$ has integer coefficients.
    So $mc$ is in the image of the map $Z_n(K;\mathbb Z)\to Z_n(K;\mathbb Q)$.
    Now $H_n(K;\mathbb Q)\cong\mathbb Q^{b'}$ for some natural number $b'$ as it has to be a finite dimensional vector space.
    But the above argument then shows $b'\le b$.
    Let $[c_1],\ldots,[c_b]\in H_n(K;\mathbb Z)$ generate the $\mathbb Z^b$ factor in $H_n(K;\mathbb Z)$.
    Then it makes sense to talk about them as elements of $H_n(K;\mathbb Q)$ as well.
    Suppose there is $\lambda_1,\ldots,\lambda_b\in\mathbb Q$ not all zero such that $\sum_i\lambda_i[c_i]=0$ in $H_n(K;\mathbb Q)$, then there exists $c\in C_{n+1}(K;\mathbb Q)$ such that $\partial_{n+1}c=\sum_i\lambda_ic_i$.
    Pick integer $m>0$ such that $m\lambda_i$ are all integers, then $\partial_{n+1}(mc)=\sum_i(m\lambda_i)c_i$, therefore $\sum_i(m\lambda_i)[c_i]=0$ in $H_n(K;\mathbb Z)$.
    But $[c_i]$ are linearly independent in $H_n(K;\mathbb Z)$, so $m\lambda_i=0$ for all $i$, hence $\lambda_i=0$ and therefore $[c_i]$ are linearly independent in $H_n(K;\mathbb Q)$.
    This means $b'\ge b$.
    Combining the two gives $b'=b$.
\end{proof}
Consequently we cannot distinguish $\mathbb RP^2$ and a point with rational homology.
Then what is the point of having it?
Well, throwing away information isn't necessarily bad.
\subsection{Euler Characteristics}
\begin{definition}
    Let $K$ be a simplicial complex.
    The Euler characteristic of $K$ is
    $$\chi(K)=\sum_{n=0}^\infty(-1)^n\dim_{\mathbb Q}H_n(K;\mathbb Q)$$
    If $X$ is a topological space with $X=|K|$, then we write $\chi(X)=\chi(K)$ which is well-defined as the homology groups do not depend on specific triangulation.
\end{definition}
There is no issue about the convergence of the series as eventually $H_n=0$ and hence the series terminates.
\begin{lemma}\label{euler}
    $$\chi(K)=\sum_{n=0}^\infty(-1)^n\dim_{\mathbb Q}C_n(K;\mathbb Q)=\sum_{n=0}^\infty(-1)^n|\{\sigma\in K:\dim\sigma=n\}|$$
\end{lemma}
\begin{proof}
    Write $\dim=\dim_\mathbb Q$.
    Note that we have
    $$\dim H_n(K;\mathbb Q)=\dim Z_n(K;\mathbb Q)-\dim B_n(K;\mathbb Q)$$
    $$\dim C_n(K;\mathbb Q)=\dim\ker\partial_n+\dim\operatorname{Im}\partial_n=\dim B_{n-1}(K;\mathbb Q)+\dim Z_n(K;\mathbb Q)$$
    Thus we can just write
    \begin{align*}
        \sum_{n=0}^\infty(-1)^n\dim C_n(K;\mathbb Q)&=\sum_{n=0}^\infty (-1)^n\dim Z_n(K;\mathbb Q)+\sum_{n=1}^\infty (-1)^n\dim B_{n-1}(K;\mathbb Q)\\
        &=\sum_{n=0}^\infty (-1)^n(\dim Z_n(K;\mathbb Q)-\dim B_n(K;\mathbb Q))\\
        &=\sum_{n=0}^\infty(-1)^n\dim_{\mathbb Q}H_n(K;\mathbb Q)
    \end{align*}
    as desired.
\end{proof}
\begin{example}
    If $\dim K=2$, we get the familiar $\chi(K)=V-E+F$.
    So for example $\chi(S^2)=1-0+1=2$.
    Consequently any triangulation of the $2$-sphere, i.e. any polyhedron, has $V-E+F=2$.
\end{example}
\subsection{The Lefschetz Fixed-Point Theorem}
\begin{definition}
    Let $X$ be triangulable and $\phi:X\to X$.
    The Lefschetz number of $\phi$ is
    $$L(\phi)=\sum_{n=0}^\infty(-1)^n\operatorname{tr}(\phi_\ast:H_n(X;\mathbb Q)\to H_n(X;\mathbb Q))$$
\end{definition}
Again the sum is eventually zero.
\begin{example}
    $L(\operatorname{id}_X)=\chi(X)$.
\end{example}
\begin{lemma}
    If $f:K\to K$ is a simplicial map, then
    $$L(|f|)=\sum_{n=0}^\infty (-1)^n\operatorname{tr}(f_n:C_n(K;\mathbb Q)\to C_n(K;\mathbb Q))$$
\end{lemma}
\begin{proof}
    Given a commutative diagram of vector spaces
    \[
        \begin{tikzcd}
            0\arrow{r}&A\arrow{d}{\alpha}\arrow{r}&B\arrow{d}{\beta}\arrow{r}&C\arrow{d}{\gamma}\arrow{r}&0\\
            0\arrow{r}&A'\arrow{r}&B'\arrow{r}&C'\arrow{r}&0
        \end{tikzcd}
    \]
    with exact rows, then it is easy to check that $\operatorname{tr}\beta=\operatorname{tr}\alpha+\operatorname{tr}\gamma$.
    The proof follows almost immediately by using the same idea as we did in Lemma \ref{euler}.
\end{proof}
\begin{theorem}[Lefschetz Fixed-Point Theorem]
    Let $\phi:X\to X$ be a map where $X$ is triangulable.
    If $L(f)\neq 0$, then $\phi$ has a fixed point.
\end{theorem}
We can actually count the fixed point in the full version of the theorem, which is out of the scope of this course.
\begin{proof}
    We shall show that if $\phi$ has no fixed point, then $L(\phi)=0$.
    As $X$ is triangulable, it has to be compact, so there is some $\delta>0$ such that $\|x-\phi(x)\|>\delta$ for all $x\in X$.
    Choose $K$ such that $X=|K|$ and, possibly using barycentric subdivision, $\operatorname{mesh}K<\delta/2$.
    Then if $x\in\sigma\in K$, then $\phi(x)\notin\sigma$.
    Let $f:K^{(r)}\to K$ be a simplicial approximation to $\phi$.
    If $v\in K^{(r)}$ is a vertex with $v\in\sigma\in K$, then $\phi(v)\in\operatorname{St}_K(f(v))$, consequently $\|\phi(v)-f(v)\|<\delta/2$.
    But $\|\phi(v)-v\|>\delta$, so $\|v-f(v)\|>\delta/2$, so $f(v)\notin\sigma$.
    Let $i_\bullet:C_\bullet(K;\mathbb Q)\to C_\bullet(K^{(r)};\mathbb Q)$ be a chain map that induces the canonical isomorphism on homology, which is supposed to map an $n$-simplex in $K$ to the sum of the $n$-simplices in $K^{(r)}$ supported in it.
    Since $f$ takes vertices of $\sigma$ out of it, it follows that $f_n\circ i_n(\sigma)$ is supported on simplices disjoint from $\sigma$.\\
    Since $\phi_\ast$ is induced at the level of chains by $f_n\circ i_n$, we now have
    $$L(\phi)=\sum_{n=0}^\infty(-1)^n\operatorname{tr}f_n\circ i_n$$
    by the preceding lemma.
    But $i_n\circ i_n$ throws any $n$-simplex elsewhere, hence $\operatorname{tr}f_n\circ i_n=0$, so $L(\phi)=0$.
\end{proof}
\begin{corollary}
    If $X$ is triangulable and contractible, then any map $\phi:X\to X$ has a fixed point.
\end{corollary}
\begin{proof}
    We just need to show that it has nonzero Lefschetz number.
    $X\simeq\{\ast\}$, so essentially $H_n(X;\mathbb Q)=0$ if $n>0$ and $\dim H_0(X;\mathbb Q)=\mathbb Q$.
    So the only nonzero map $\phi_\ast$ is $\phi_\ast:H_0(X;\mathbb Q)\to H_0(X;\mathbb Q)$ which is the identity.
    Thus $L(\phi)=1$.
\end{proof}