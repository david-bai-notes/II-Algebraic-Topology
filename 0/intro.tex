\section{Introduction}
\subsection{Motivation and Examples}
We start by asking a natural question that appears frequently in the study of topology:
\begin{problem}\label{homeo}
    How can we tell if two topological space are different (i.e. not homoeomorphic)?
\end{problem}
For example, if we let $X$ to be the donut-like surface in $\mathbb R^3$, and $Y$ be a similar donut but with two holes instead.
Our intuition tells us they are not homeomorphic because, well, $X$ has one hole and $Y$ has two -- but this is not much a proof.\\
Our basic strategy of coping with problems like these in algebraic topology is to associate each topological space $X$ (often, in a specific class) with a group $H(X)$ and associate each continuous map $f:X\to Y$ a homomorphism of groups $H(f):H(X)\to H(Y)$.
We obviously want $H$ to preserve compositions of functions and identities as well.
\footnote{In short, we just want $H$ to be a functor from a subcategory of $\mathbf{Top}$ to an abelian category, preferably $\mathbf{Grp}$.}
With certain constructions, if $X\cong Y$ as topological spaces, we can ensure that $H(X)\cong H(Y)$ as groups.
In other words, $H$ exists as an algebraic invariant.
As it is much easier to show two groups are not isomorphic, we can solve some cases of Problem \ref{homeo} where $H(X)$ can be shown not to isomorphic to $H(Y)$.
The study of algebraic topology is basically the hunt of such algebraic invariants.\\
Another application of this idea is to solve the Extension Problem:
\begin{problem}[The Extension Problem]
    Let $X$ be a topological space and $A\subset X$ a subspace.
    If we have $f:A\to Y$ is a continuous map, how do we know if there is a continuous map $F:X\to Y$ with $F|_A=f$?
\end{problem}
That is, we want to know if there exists $F$ such that the diagram
\[
    \begin{tikzcd}
        A \arrow[hookrightarrow]{d} \arrow{r}{f} & Y\\
        X \arrow[dashed,swap]{ur}{F}&
    \end{tikzcd}
\]
commutes.
An example of this is the following theorem:
\begin{theorem}
    There is no continuous function $F:D^n\to S^{n-1}$ such that
    \[
        \begin{tikzcd}
            S^{n-1} \arrow[hookrightarrow]{d} \arrow{r}{\operatorname{id}} & S^{n-1}\\
            D^n \arrow[swap]{ur}{F}
        \end{tikzcd}
    \]
    commutes.
\end{theorem}
Here, $D^n$ is the $n$-disk while $S^{n-1}$ is the $(n-1)$-sphere, which is simply the hypersurface enclosing $D^n$.
How does the idea of the algebraic invariant come in handy on this problem?
Obviously we do not yet have the correct tool to do it now that the course has just started, but we can take a glimpse of the idea involved.
\begin{proof}
    Construct an invariant $H$ such that $H(S^{n-1})\cong\mathbb Z$ and $H(D^n)\cong 0$.
    Then the diagram in the statement of the theorem will implies that
    \[
        \begin{tikzcd}
            \mathbb Z\arrow{d} \arrow{r}{\operatorname{id}} & \mathbb Z\\
            0 \arrow{ur}
        \end{tikzcd}
    \]
    commute.
    But this is absurd.
\end{proof}
All these seems nice, right?
But first, we really need a clever construction of such a nontrivial algebraic invariants bearing such nice properties.
\subsection{Conventions}
Unless otherwise stated, we adopt the following conventions:\\
When we mention ``space'', we always mean a topological space.
And when we say a ``map'' between two spaces, we always mean a continuous one.
By $I$, we mean the unit interval $[0,1]$.