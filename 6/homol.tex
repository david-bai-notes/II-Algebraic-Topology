\section{Homology Calculations and Applications}
\begin{example}
    If $S^n\simeq S^m$, then $m=n$.
    This follows immediately from our calculation of the homology groups of simplices which are the triangulations of the spheres.
\end{example}
\begin{theorem}
    If $\mathbb R^n\cong\mathbb R^m$, then $m=n$.
\end{theorem}
\begin{proof}
    Suppose we are given a homeomorphism $\phi:\mathbb R^m\to\mathbb R^n$, then by a translation we can assume WLOG that $\phi(0)=0$.
    Then we can restrict $\phi$ so that $\mathbb R^n\setminus\{0\}$ and $\mathbb R^m\setminus\{0\}$, which are homotopic to $S^{n-1}$ and $S^{m-1}$ respectively, consequently $S^{n-1}\simeq S^{m-1}$, so $n=m$.
\end{proof}
\begin{proposition}
    Any map $\phi:D^n\to D^n$ has a fixed point.
\end{proposition}
\begin{proof}
    Repeat the argument as in the $D^2$ case and finish the argument by homology.
\end{proof}
\subsection{Mayer-Vietoris Theorem}
The Mayor-Vietoris theorem is an analog of Seifert-van Kampen theorem as in both deals with the algebraic invariants of spaces that are constructed by gluing.
\begin{definition}
    A sequence of homomorphisms of abelian groups
    \[
        \begin{tikzcd}
            \cdots\arrow{r}&A_{i+1}\arrow{r}{f_i}&A_i\arrow{r}{f_{i-1}}&A_{i-1}\arrow{r}&\cdots
        \end{tikzcd}
    \]
    is exact at $A_i$ if $\operatorname{Im}f_i=\ker f_{i-1}$.
    We say the sequence is exact if it is exact at every $A_i$.
\end{definition}
Consequenly, any exact sequence is a chain complex with all homology groups trivial.
\begin{definition}
    A short exact sequence is an exact sequence in the form
    \[
        \begin{tikzcd}
            0\arrow{r}&A\arrow{r}&B\arrow{r}&C\arrow{r}&0
        \end{tikzcd}
    \]
\end{definition}
So necessarily $A\to B$ is injective, $B\to C$ is surjective and $C\cong B/A$.
An exact sequence that is not short is of course called a long exact sequence.\\
Suppose $K$ is a simplicial complex with $K=L\cup M$ where $L,M$ are subcomplexes of $K$.
Then $N=L\cap M$ is also a subcomplex.
We usually write $K=L\cup_NM$.
Analogous to what we did in Seifert-van Kampen theorem, we want to relate the homology groups of $L,M,N$ to the homology group of $K$.
Of course we immediately get the natual inclusion maps
$$i:N\to L,j:N\to M,l:L\to K,m:M\to K$$
\begin{theorem}[Mayer-Vietoris]\label{mayer-vietoris}
    There exists a map (known as the connecting homomorphism) $\delta_\ast:H_n(K)\to H_{n-1}(N)$ for each $n$ such that the sequence
    \[
        \begin{tikzcd}
            \cdots\arrow{r}{\delta_\ast}&H_n(N)\arrow{r}{i_\ast\oplus j_\ast}&H_n(L)\oplus H_n(M)\arrow{r}{l_\ast-m_\ast}&H_n(K)\arrow[swap,overlay,out=0,in=180]{dll}{\delta_\ast}&\\
            &H_{n-1}(N)\arrow[swap]{r}{i_\ast\oplus j_\ast}&H_{n-1}(L)\oplus H_{n-1}(M)\arrow[swap]{r}{l_\ast-m_\ast}&H_{n-1}(K)\arrow[swap]{r}{\delta_\ast}&\cdots
        \end{tikzcd}
    \]
    is exact.
\end{theorem}
We will prove this after further development on homological algebra.
\begin{definition}
    We say a sequence of chain complexes
    \[
        \begin{tikzcd}
            \cdots\arrow{r}&A_\bullet\arrow{r}{f_\bullet}&B_\bullet\arrow{r}{g_\bullet}&C_\bullet\arrow{r}&\cdots
        \end{tikzcd}
    \]
    is exact at $B_\bullet$ if
    \[
        \begin{tikzcd}
            \cdots\arrow{r}&A_n\arrow{r}{f_n}&B_n\arrow{r}{g_n}&C_n\arrow{r}&\cdots
        \end{tikzcd}
    \]
    is exact for every $n$.
\end{definition}
So we can analogously define (short) exact sequences of chain complexes.
\begin{lemma}[Snake Lemma]
    Let
    \[
        \begin{tikzcd}
            0\arrow{r}&A_\bullet\arrow{r}{f_\bullet}&B_\bullet\arrow{r}{g_\bullet}&C_\bullet\arrow{r}&0
        \end{tikzcd}
    \]
    be a short exact sequence of chain complexes.
    Then for any $n$ there is a homomorphism $\delta_\ast:H_{n+1}(C_\bullet)\to H_n(A_\bullet)$ such that
    \[
        \begin{tikzcd}
            \cdots\arrow{r}{\delta_\ast}&H_{n+1}(A_\bullet)\arrow{r}{f_\ast}&H_{n+1}(B_\bullet)\arrow{r}{g_\ast}&H_{n+1}(C_\bullet)\arrow[swap,overlay,out=0,in=180]{dll}{\delta_\ast}&\\
            &H_n(A_\bullet)\arrow[swap]{r}{f_\ast}&H_n(B_\bullet)\arrow[swap]{r}{g_\ast}&H_n(C_\bullet)\arrow[swap]{r}{\delta_\ast}&\cdots
        \end{tikzcd}
    \]
    is exact.
\end{lemma}
\begin{proof}
    We already have this huge commutative diagram with exact rows
    \[
        \begin{tikzcd}
            &\vdots\arrow{d}&\vdots\arrow{d}&\vdots\arrow{d}&\\
            0\arrow{r}&A_{n+1}\arrow{r}{f_{n+1}}\arrow{d}{\partial_{n+1}}&B_{n+1}\arrow{r}{f_{n+1}}\arrow{d}{\partial_{n+1}}&C_{n+1}\arrow{r}\arrow{d}{\partial_{n+1}}&0\\
            0\arrow{r}&A_n\arrow{r}{f_n}\arrow{d}{\partial_n}&B_n\arrow{r}{f_n}\arrow{d}{\partial_n}&C_n\arrow{r}\arrow{d}{\partial_n}&0\\
            0\arrow{r}&A_{n-1}\arrow{r}{f_{n-1}}\arrow{d}&B_{n-1}\arrow{r}{g_{n-1}}\arrow{d}&C_{n-1}\arrow{r}\arrow{d}&0\\
            &\vdots&\vdots&\vdots&
        \end{tikzcd}
    \]
    We now construct $\delta_\ast:H_{n+1}(C_\bullet)\to H_n(A_\bullet)$.
    Take $[x]\in H_{n+1}(C_\bullet)$ for $x\in Z_{n+1}(C_\bullet)$.
    As $g_{n+1}$ is surjective, there is $x\in B_{n+1}$ such that $g_{n+1}(y)=x$.
    Now $g_n\circ\partial_{n+1}(y)=\partial_{n+1}\circ g_{n+1}(y)=\partial_{n+1}(x)=0$.
    So by exactness, there exists $z\in A_n$ such that $f_n(z)=\partial_{n+1}(y)$.
    Note that $f_{n-1}\circ\partial_n(z)=\partial_n\circ f_n(z)=\partial_n\circ\partial_{n+1}(y)=0$, so $\partial_n(z)=0$ since $f_{n-1}$ is injective.
    This means that $z\in Z_n(A_\bullet)$, so we define $\delta_\ast([x])=[z]$.
    To see it is well-defined, we have two issues to deal with:\\
    First, if we replace $x$ by $x+\partial_{n+2}(x')$ and $g_{n+2}(y')=x'$ , then replace $y$ with $y+\partial_{n+2}(y')$.
    Then $g_{n+1}(y+\partial_{n+2}(y'))=g_{n+1}(y)+\partial_{n+2}\circ g_{n+2}(y')=x+\partial_{n+2}(x')$.
    But $\partial_{n+1}(y+\partial_{n+2}(y'))=\partial_{n+1}(y)$, so $z$ does not change.\\
    Secondly, if we have made up our mind on $x$ and $y,y'$ are chosen with $g_{n+1}(y')=g_{n+1}(y)=x$, then $g_{n+1}(y'-y)=0$, therefore by exactness there is some $z'$ such that $f_{n+1}(z')=y'-y$, thus $y'=y+f_{n+1}(z')$.
    So $\partial_{n+1}(y')=\partial_{n+1}(y)+\partial_{n+1}\circ f_{n+1}(z')=\partial_{n+1}(y)+f_n\circ\partial_{n+1}(z')$, which means $f_n(z+\partial_{n+1}(z'))=\partial_{n+1}(y)+f_n\circ\partial_{n+1}(z')=\partial_{n+1}(y')$.
    This is saying that replacing $y$ by $y'$ modifies $z$ by adding $\partial_{n+1}(z')$.
    But $[z]=[z+\partial_{n+1}(z')]$, so $\delta_\ast$ is well-defined.\\
    The proof of the sequence being exact is tedious routine work.\\
    At $H_n(B_\bullet)$, if $[a]\in H_n(A_\bullet)$, then by definition $g_\ast\circ f_\ast([a])=[g_n\circ f_n(a)]=0$ by the exactness of our original short exact sequence.
    So $\operatorname{Im}f_\ast\subset \ker g_\ast$.
    Now if $g_\ast([b])=0$, then $[g_n(b)]=g_\ast([b])=0$, thus there is some $c\in C_{n+1}$ with $g_n(b)=\partial_{n+1}(c)$.
    But then choosing $b'\in B_{n+1}$ with $g_{n+1}(b')=c$ by exactness gives $g_n(b-\partial_{n+1}(b'))=0$, hence $b-\partial_{n+1}(b')\in\ker g_n=\operatorname{Im}f_n$, so there is some $a\in A_n$ such that $f_n(a)=b-\partial_{n+1}(b')$.
    In particular, $f_\ast([a])=[f_n(a)]=[b-\partial_{n+1}(b')]=[b]$, therefore $\operatorname{Im}f_\ast\supset\ker g_\ast$.
    Combining these gives $\operatorname{Im}f_\ast=\ker g_\ast$ which implies the exactness at $H_n(B_\bullet)$.\\
    At $H_n(A_\bullet)$, suppose $[z]=\delta_\ast([x])$, then $f_\ast([z])=[f_n(z)]=[\partial_{n+1}(y)]=0$ where $y$ is as in the construction of $\delta_\ast$, so $\operatorname{Im}\delta_\ast\subset\ker f_\ast$.
    Suppose now that $f_\ast([z])=0$, then $f_n(z)=\partial_{n+1}(y)$ for some $y\in B_{n+1}$.
    Take $x=g_{n+1}(y)$, then $\partial_{n+1}(x)=\partial_{n+1}\circ g_{n+1}(y)=g_n\circ\partial_{n+1}(y)=g_n\circ f_n(z)=0$, so $x$ is a cycle, so $[x]$ is a homology class.
    But $\delta_\ast([x])=[z]$ by construction, hence $\operatorname{Im}\delta_\ast\supset\ker f_\ast$, therefore $\operatorname{Im}\delta_\ast=\ker f_\ast$ which gives the exactness at $H_n(A_\bullet)$.\\
    At $H_n(C_\bullet)$, if $[x]\in\operatorname{Im}g_\ast$, then (ater replacing $x$ with another representative if necessary) WLOG there exists a cycle $y\in B_n$ such that $g_n(y)=x$.
    But then immediately $\delta_\ast([x])=0$, so $\operatorname{Im}g_\ast\subset\ker\delta_\ast$.
    Conversely, suppose $\delta_\ast([x])=0$, then let $y,z$ be the corresponding element in the construction of $\delta_\ast$.
    By hypothesis there exists $a\in A_n$ such that $\partial_n(a)=z$.
    Then $\partial_n\circ f_n(a)=f_{n-1}\circ\partial_n(a)=f_{n-1}(z)=\partial_n(y)$.
    Thus $\partial_n(y-f_n(a))=0$, so $y-f_n(a)$ is a cycle.
    Also $g_n(y-f_n(a))=x$, so $[x]=g_\ast([y-f_n(a)])$.
    Consequently $\operatorname{Im}g_\ast\supset\ker\delta_\ast$, so $\operatorname{Im}g_\ast=\ker\delta_\ast$.
    This finished the proof.
\end{proof}
\begin{proof}[Proof of Theorem \ref{mayer-vietoris}]
    Easy to check that $C_\bullet\oplus D_\bullet$ is a chain complex with all data obtained from direct sums of data of $C_\bullet,D_\bullet$.
    Also $H_n(C_\bullet\oplus D_\bullet)=H_n(C_\bullet)\oplus H_n(D_\bullet)$.
    It then suffices to check that
    \[
        \begin{tikzcd}
            0\arrow{r}&C_\bullet(N)\arrow{r}{i_\bullet\oplus j_\bullet}&C_\bullet(L)\oplus C_\bullet(M)\arrow{r}{l_\bullet-m_\bullet}&C_\bullet(K)\arrow{r}&0
        \end{tikzcd}
    \]
    is exact.
    The proof is then done by snake lemma.\\
    Note that $C_\bullet(N)$ naturally embeds into $C_\bullet (L)$ and $C_\bullet(M)$ via $i_\bullet\oplus j_\bullet$, so in particular $i_\bullet\oplus j_\bullet$ is a injective.
    Also since $K=L\cup M$, for any $c\in C_n(K)$ we can write $c=c_L+c_M$ where $c_L$ is a linear combination of simplices in $L$ and $c_M$ is that in $M$.
    Let $b_L,b_M$ be the respective copies of $c_L,c_M$ in $C_n(L),C_n(M)$ respectively, then $l_\bullet(b_L)=c_L,m_\bullet(b_M)=c_M$, and hence $c=(l_\bullet-m_\bullet)(b_L,-b_M)$, so $l_\bullet-m_\bullet$ is surjective.\\
    It remains to show the exactness in the middle.
    For $(b_L,b_M)\in C_n(L)\oplus C_n(M)$, note that $l_\bullet(b_L)-m_\bullet(b_M)=0$ iff every simplex which occurs in $b_L$ also occurs in $b_M$ with the same coefficient.
    This is just saying that $b_L,b_M$ are linear combinations of simplices in $L\cap M=N$.
    Therefore $\operatorname{Im}(i_\bullet\oplus j_\bullet)=\ker (l_\bullet-m_\bullet)$, hence the sequence is indeed exact.
\end{proof}
\begin{lemma}[Five Lemma]
    Given a commutative diagram
    \[
        \begin{tikzcd}
            A\arrow{r}\arrow{d}{\alpha}&B\arrow{r}\arrow{d}{\beta}&C\arrow{r}\arrow{d}{\gamma}&D\arrow{r}\arrow{d}{\delta}&E\arrow{d}{\epsilon}\\
            A'\arrow{r}&B'\arrow{r}&C'\arrow{r}&D'\arrow{r}&E'
        \end{tikzcd}
    \]
    If the rows are exact and $\alpha,\beta,\delta,\epsilon$ are isomorphisms, then $\gamma$ is also an isomorphism.
\end{lemma}
\begin{proof}
    Exercise.
\end{proof}
We can now prove Proposition \ref{barycentric_iso_homol}.
Recall that we want to show that $s_\ast:H_n(K')\to H_n(K)$ is an isomorphism where $s:K'\to K$ is a simplicial approximation to the identity map on $|K|=|K'|$ which sends each $\hat\sigma$ to a vertex of $\sigma$.
\begin{proof}[Proof of Proposition \ref{barycentric_iso_homol}]
    Induction on the number of simplices of $K$.
    If $K$ has only one simplex, then it is just a vertex, so the proposition is trivial.
    For the induction step, let $\sigma\in K$ be of maximal dimension, then $L=K\setminus \{\sigma\}$ is also a simplicial complex.
    Let $M$ be the simplicial complex consists of $\sigma$ and its faces.
    Then $N=M\cap L$ is just the proper faces of $\sigma$.
    By construction of $s$ we can restrict $s$ to $L',M',N'$ to get the maps
    $$s_\ast:H_n(L')\to H_n(L),s_\ast:H_n(M')\to H_n(M),s_\ast:H_n(N')\to H_n(N)$$
    By the induction hypothesis, these are all isomorphisms.
    Then by Mayer-Vietoris, we have the diagram
    \[
        \begin{tikzcd}[column sep=0.34em]
            H_n(N')\arrow{r}\arrow{d}{s_\ast}&H_n(L')\oplus H_n(M')\arrow{r}\arrow{d}{s_\ast\oplus s_\ast}&H_n(K')\arrow{r}\arrow{d}{s_\ast}&H_{n-1}(N')\arrow{r}\arrow{d}{s_\ast}&H_{n-1}(L')\oplus H_{n-1}(M')\arrow{d}{s_\ast\oplus s_\ast}\\
            H_n(N)\arrow{r}&H_n(L)\oplus H_n(M)\arrow{r}&H_n(K)\arrow{r}&H_{n-1}(N)\arrow{r}&H_{n-1}(L)\oplus H_{n-1}(M)
        \end{tikzcd}
    \]
    with exact rows.
    Easy to check that it commutes.
    Also, by the induction hypothesis, all vertical arrows except $s_\ast:H_n(K')\to H_n(K)$ are isomorphisms.
    Hence $s_\ast:H_n(K')\to H_n(K)$ is an isomorphism by five lemma.
\end{proof}
\subsection{Homology of Compact Surfaces}
Recall that we constructed the oriented surfaces of genus $g$ as $\Sigma_g=\Gamma_{2g}\cup_{\rho_g}D^2$ where $\Gamma_{2g}$ is a bouquet of $2g$ copies of $S^1$ with generators $\alpha_1,\ldots,\alpha_g,\beta_1,\ldots,\beta_g$ and $\rho_g=\alpha_1\beta_1\alpha_1^{-1}\beta_1^{-1}\cdots \alpha_g\beta_g\alpha_g^{-1}\beta_g^{-1}$.
\begin{example}
    $\Gamma_r$ can be triangulated in the obvious way by taking it as $r$ hollow triangles joined by a common vertex.
    We know that $H_n(\Gamma_1)$ is $\mathbb Z$ when $n=0,1$ and $0$ otherwise.
    We claim that
    $$H_i(\Gamma_r)=\begin{cases}
        \mathbb Z\text{, if $i=0$}\\
        \mathbb Z^r\text{, if $i=1$}\\
        0\text{, otherwise}
    \end{cases}$$
    which we shall show inductively.
    The cases except for $i=1$ are all trivial, so it suffices to show that.
    Suppose we have shown the case for $r-1$, then take $K=\Gamma_r$, $L$ be a natural copy of $\Gamma_{r-1}$ in $\Gamma_r$ and $M$ be the remaining triangle.
    Then $N=L\cap M=\{\ast\}$ and hence Mayer-Vietoris gives the exact sequence
    \[
        \begin{tikzcd}
            0\arrow{r}&H_1(N)\arrow{r}&H_1(\Gamma_{r-1})\oplus H_1(S^1)\arrow{r}& H_1(\Gamma_r)\arrow[overlay, out=0, in=180]{dll}&\\
            &H_0(N)\arrow{r}&H_0(\Gamma_{r-1})\oplus H_0(S^1)\arrow{r}& H_0(\Gamma_r)\arrow{r}&0
        \end{tikzcd}
    \]
    which, after putting in everything we already know,
    \[
        \begin{tikzcd}
            0\arrow{r}&0\arrow{r}&\mathbb Z^{r-1}\oplus \mathbb Z\arrow{r}& H_1(\Gamma_r)\arrow[overlay, out=0, in=180]{dll}&\\
            &\mathbb Z\arrow{r}&\mathbb Z\oplus \mathbb Z\arrow{r}& \mathbb Z\arrow{r}&0
        \end{tikzcd}
    \]
    By exactness of the bottom row, the map $\mathbb Z\to\mathbb Z\oplus\mathbb Z$ is injective (alternatively this can also be easily seen from the definition of that map), hence the connecting homomophism is the zero map.
    Therefore the top row reduces to the exact sequence
    \[
        \begin{tikzcd}
            0\arrow{r}&\mathbb Z^{r-1}\oplus \mathbb Z\arrow{r}& H_1(\Gamma_r)\arrow{r}&0
        \end{tikzcd}
    \]
    which precisely means that $H_1(\Gamma_r)\cong\mathbb Z^{r-1}\oplus \mathbb Z\cong\mathbb Z^r$.
\end{example}
Worth noting that $H_1(\Gamma_r)$ is the free abelian group on $r$ letters while $\pi_1(\Gamma_r)$ is the free group on $r$ letters.
Also, if $\alpha_1,\ldots,\alpha_r$ are the generating paths of the circles, then $H_1(\Gamma_r)$ is generated by $\alpha_1,\ldots,\alpha_r$.
\begin{remark}
    Like in the preceding exacmple, whenever $N$ is connected, the map $H_0(N)\to H_0(L)\oplus H_0(M)$ is injective, and hence $H_1(L)\oplus H_1(M)\to H_1(K)$ is surjective.
\end{remark}
Now we want to attach our $2$-cell to get $\Sigma_g$.
We'll do this in two steps.\\
First, we attach a cylinder $S^1\times I$ via the same $\rho_g$ but on $S^1\times\{0\}$.
Write $\Sigma_g^\star=\Gamma_{2g}\cup_{\rho_g} (S^1\times I)$
By shrinking $I$ to a point we can obtain the deformation retraction of $\Sigma_g^\star$ onto $\Gamma_{2g}$, therefore they have the same homology groups.\\
Now $\Sigma_g=\Sigma_g^\star\cup_\alpha D^2$ where $\alpha:\partial D^2\to S^1\times \{1\}$ is the natural identity.
We can choose triangulations of $\Sigma_g^\star$ and $D^2$ so that they are compatible under this gluing (so we are essentially removing a $D^2$ from $\Sigma_g$ to get $\Sigma_g^\star$).
\footnote{There are of course some technical details regarding why we can do this, but we are not going into this part of details here. It is not too hard though.}
Let $L$ be the triangulation of $\Sigma_g^\star$ and $M$ be the triangulation of $D^2$ such that this holds.
Then $N=L\cap M$ is a triangulation of $S^1\times\{1\}\cong S^1$.
Majer-Vietoris then gives the exact sequence
\[
    \begin{tikzcd}
        &H_2(L)\oplus H_2(M)\arrow{r}&H_2(\Sigma_g)\arrow[overlay, out=0,in=180]{dll}&\\
        H_1(N)\arrow[swap]{r}{i_\ast}&H_1(L)\oplus H_1(M)\arrow{r}&H_1(\Sigma_g)\arrow{r}&0
    \end{tikzcd}
\]
Note that the zero at the end is due to the preceding remark.
Again putting everything we already know into it gives
\[
    \begin{tikzcd}
        0\arrow{r}&H_2(\Sigma_g)\arrow{r}&\mathbb Z\arrow{r}{i_\ast}&\mathbb Z^{2g}\arrow{r}&H_1(\Sigma_g)\arrow{r}&0
    \end{tikzcd}
\]
The exactness means that $H_2(\Sigma_g)\cong\ker i_\ast$ and $H_1(\Sigma_g)\cong\operatorname{coker}i_\ast=\mathbb Z^{2g}/\operatorname{Im}i_\ast$.
So we just need to understand $i_\ast:\mathbb Z\to\mathbb Z^{2g}$, or in other words $i_\ast:H_1(S^1)\to H_1(\Sigma_g^\star)$.
Note that the generator of $H_1(S^1)$ is the cycle runs through it (which is basically the one corresponding to the generator of $\pi_1(S^1)$).
The deformation retract of $\Sigma_g^\star$ to $\Gamma_{2g}$ identifies $S^1$ with the image of the cycle under $\rho_g$.
Therefore $(\rho_g)_\ast$ maps the cycle to $[\alpha_1]+[\beta_1]-[\alpha_1]-[\beta_1]+\cdots+[\alpha_g]+[\beta_g]-[\alpha_g]-[\beta_g]=0$.
Therefore $i_\ast=0$ and hence $H_2(\Sigma_g)\cong\mathbb Z$ and $H_1(\Sigma_g)\cong\mathbb Z^{2g}$.
To conclude,
$$H_n(\Sigma_g)\cong\begin{cases}
    \mathbb Z\text{, if $n=0,2$}\\
    \mathbb Z^{2g}\text{, if $n=1$}\\
    0\text{, otherwise}
\end{cases}$$
which in particular allows us to know $g$ given the homology group of some $\Sigma_g$.\\
We also have the non-orientable surfaces $S_g$ which is given by $\Gamma_{g+1}\cup_\alpha D^2$ where $\alpha=\alpha_0^2\alpha_1^2\cdots\alpha_g^2$ where $\alpha_i$ is the generator of the $i^{th}$ circle in the bouquet.
Repeating the same process gives the exact sequence
\[
    \begin{tikzcd}[row sep=tiny]
        0\arrow{r}&H_2(\Sigma_g)\arrow{r}&H_1(S^1)\arrow{r}{i_\ast}&H_1(\Gamma_{g+1})\arrow{r}&H_1(\Sigma_g)\arrow{r}&0\\
        &&\mathbb Z\arrow[equal]{u}&\mathbb Z^{g+1}\arrow[equal]{u}&
    \end{tikzcd}
\]
So again $H_2(S_g)\cong\ker i_\ast$ and $H_1(S_g)\cong\operatorname{coker}i_\ast$.
Note that $i_\ast$ maps the generator of $H_1(S^1)$ to the nonzero element $2[\alpha_0]+\cdots+2[\alpha_g]$.
Hence $\ker i_\ast=0$ and $\operatorname{coker}i_\ast\cong\mathbb Z^{g+1}/(2,\ldots,2)\mathbb Z\cong\mathbb Z^g\oplus\mathbb Z/2\mathbb Z$.
In conclusion
$$H_n(S_g)\cong\begin{cases}
    \mathbb Z\text{, if $n=0$}\\
    \mathbb Z^g\oplus\mathbb Z/2\mathbb Z\text{, if $n=1$}\\
    0\text{, otherwise}
\end{cases}$$
In particular we can tell apart $S_g$ and $\Sigma_g$ as well.