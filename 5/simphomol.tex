\section{Simplicial Homology}
\subsection{Oriented Simplices and Boundary Homomorphism}
We slightly modify our definition of simplex by associating each simplex with an orientation.
For a simplex $\langle v_0,\ldots,v_n\rangle$, we can shuffle the vertices around which defines an action of $S_{n+1}$ on it.
Write $A_{n+1}\unlhd S_{n+1}$ as the alternating group, then it also acts on the simplex by restriction.
An orientation on $\sigma=\langle v_0,\ldots,v_n\rangle$ is a choice of ordering defined up to the action of $A_{n+1}$.
So we can now view $\langle v_0,\ldots,v_n\rangle$ as a simplex with an orientation.
\begin{example}
    $0$-simplex does not have a notion of orientation.\\
    There are two possible orderings of $1$-simplex, namely $\langle v_0,v_1\rangle$ and $\langle v_1,v_0\rangle$ which have different orientation.\\
    For $2$-simplices, again there are two orientations and they are $\langle v_0,v_1,v_2\rangle=\langle v_1,v_2,v_0\rangle=\langle v_2,v_0,v_1\rangle$ and $\langle v_1,v_0,v_2\rangle=\langle v_0,v_2,v_1\rangle=\langle v_2,v_1,v_0\rangle$.
\end{example}
So afterwards when we mention simplex we mean oriented simplex.
\begin{definition}
    Let $K$ be a simplicial complex, we define the group $C_n(K)$ of $n$-chains to be the free abelian group generated by the simplices of dimension $n$ in $K$, i.e.
    $$C_n(K)=\bigoplus_{\sigma\in K,\dim\sigma=n}\langle\sigma\rangle$$
    where $\langle\sigma\rangle$ is the free group generated by $\sigma$.
\end{definition}
We will convention that we have made our choice of some orientation.
For a simplex $\sigma$ in this orientation, the same simplex in the other orientation is denoted $\bar\sigma$.
In $\langle\sigma\rangle$, we identify $\bar\sigma$ by $-\sigma$.
\begin{definition}
    The $n^{th}$ boundary homomorphism is a homomorphism $\partial=\partial_n:C_n(K)\to C_{n-1}(K)$ induced by
    $$\partial_n\langle v_0,\ldots,v_n\rangle = \sum_{i=0}^n(-1)^i\langle v_1,\ldots,\hat{v}_i,\ldots,v_n\rangle$$
    where $\langle v_1,\ldots,\hat{v}_i,\ldots,v_n\rangle=\langle v_0,\ldots,v_{i-1},v_{i+1},\ldots,v_n\rangle$.
\end{definition}
\begin{example}
    $\partial\langle v_0,v_1\rangle=\langle v_1\rangle-\langle v_0\rangle$.
    $\partial\langle v_0,v_1,v_2\rangle=\langle v_1,v_2\rangle-\langle v_0,v_2\rangle+\langle v_0,v_1\rangle=\langle v_1,v_2\rangle+\langle v_2,v_0\rangle+\langle v_0,v_1\rangle$ which is indeed the (oriented) topological boundary of the simplex $\langle v_0,v_1,v_2\rangle$.
\end{example}
\begin{remark}
    We have $\partial\bar\sigma=-\partial\sigma$.
\end{remark}
\subsection{The Homology Groups of Simplicial Complexes}
\begin{definition}
    Let $K$ be a simplicial complex and $n\in\mathbb Z$, the group of $n$-cycles of $K$ is $Z_n(K)=\ker\partial_n$.
    The group of $n$-boundaries is $B_n(K)=\operatorname{Im}\partial_{n+1}$.
\end{definition}
\begin{lemma}
    $B_n(K)\subset Z_n(K)$, or in other words $\partial_{n-1}\circ\partial_n=0$.
\end{lemma}
\begin{proof}
    Pick an $n$-simplex $\langle v_0,\ldots,v_n\rangle$, then
    \begin{align*}
        \partial_{n-1}\circ\partial_n(\langle v_0,\ldots,v_n\rangle)&=\partial_{n-1}\left( \sum_{i=0}^n(-1)^i\langle v_0,\ldots,\hat{v}_i,\ldots,v_n\rangle \right)\\
        &=\sum_{i=0}^n(-1)^i\partial_{n-1}(\langle v_0,\ldots,\hat{v}_i,\ldots,v_n\rangle)\\
        &=\sum_{j<i}(-1)^j(-1)^i\langle v_0,\ldots,\hat{v}_j,\ldots,\hat{v}_i,\ldots,v_n\rangle\\
        &\quad+\sum_{j>i}(-1)^{j-1}(-1)^i\langle v_0,\ldots,\hat{v}_i,\ldots,\hat{v}_j,\ldots,v_n\rangle\\
        &=0
    \end{align*}
    as desired.
\end{proof}
\begin{definition}
    The $n^{th}$ (simplicial) homology group of the simplicial complex $K$ is defined as $H_n(K)=Z_n(K)/B_n(K)$.
\end{definition}
\begin{example}
    Take the simplicial complex $K$ generated by boundary of a $2$-simplex (which can be taken as a triangulation of $S^1$).
    Then $C_0(K)\cong C_1(K)\cong\mathbb Z^3$ and $C_n(K)=0$ for all $n>1$.
    So we only need to understand $\partial_1$.
    Say its $0$-simplices are $\langle v_0\rangle,\langle v_1\rangle,\langle v_2\rangle$ and $1$-simplices are $\langle v_0,v_1\rangle,\langle v_1,v_2\rangle,\langle v_2,v_0\rangle$, then $\partial_1:C_1(K)\to C_0(K)$ maps
    \begin{align*}
        \langle v_0,v_1\rangle &\mapsto \langle v_1\rangle-\langle v_0\rangle\\
        \langle v_1,v_2\rangle &\mapsto \langle v_2\rangle-\langle v_1\rangle\\
        \langle v_2,v_0\rangle &\mapsto \langle v_0\rangle-\langle v_2\rangle
    \end{align*}
    So if we take the free abelian groups as free $\mathbb Z$-modules generated by the simplices, then $\partial_1$ has the matrix
    $$\begin{pmatrix}
        -1&0&1\\
        1&-1&0\\
        0&1&-1
    \end{pmatrix}$$
    So $Z_1(K)=\ker\partial_1=\langle(1,1,1)\rangle=\langle \langle v_0,v_1\rangle +\langle v_1,v_2\rangle + \langle v_2,v_0\rangle\rangle$ and $B_1(K)=\ker\partial_2=0$, therefore $H_1(K)\cong\mathbb Z$.\\
    As for the $H_0(K)$, we have $Z_0(K)=\ker\partial_0=C_0(K)\cong\mathbb Z^3$ and $B_0(K)=\operatorname{Im}\partial_1\cong\langle (-1,1,0),(0,-1,1)\rangle$, hence $H_0(K)\cong\mathbb Z^3/\langle (-1,1,0),(0,-1,1)\rangle\cong\mathbb Z$.
    And $H_n(K)=0$ for $n>1$.
\end{example}
\begin{example}
    Take $L$ the simplicial complex generated by the $2$-simplex (so it is a solid triangle which is a triangulation of the closed unit disk).
    Then $C_0(L)=C_0(K),C_1(L)=C_1(K)$ but $C_2(L)=\langle\langle v_0,v_1,v_2\rangle\rangle$ where $K$ is as in the previous example.
    Now $\partial_2(\langle v_0,v_1,v_2\rangle)=\langle v_0,v_1\rangle+\langle v_1,v_2\rangle+\langle v_2,v_0\rangle$ which is the generator of $\ker\partial_1$, therefore $Z_1(L)=B_1(L)$ and hence $H_1(L)=0$.
    Note that $L,K$ coincides on $0$ and $1$-simplices they contain, so $H_0(L)\cong H_0(K)\cong\mathbb Z$.
    Now easily $Z_2(L)=\ker\partial_2=0$ and therefore $H_2(L)=0$, therefore $H_0(L)\cong\mathbb Z$ and $H_n(L)=0$ for any $n\neq 0$.
\end{example}
\begin{lemma}
    Let $K$ be a simplicial complex.
    If $d$ is the number of path components of $|K|$, then $H_0(K)\cong\mathbb Z^d$.
\end{lemma}
\begin{proof}
    Denote by $\pi_0(K)$ the set of path-connected components of $|K|$.
    Write $\mathbb Z[A]$ as the free abelian group generated by a set $A$, then $\mathbb Z[\pi_0(K)]\cong\mathbb Z^d$.
    Consider a $q:C_0(K)\to\mathbb Z[\pi_0(K)]$ sending a vertex $\langle v\rangle$ to the path component containing $v$ and extend it to a homomorphism.
    Then $q$ is surjective.
    Note that $\partial_0=0$, so $Z_0(K)=C_0(K)$, therefore $H_0(K)\cong C_0(K)/B_0(K)$, therefore it suffices to show that $\ker q=B_0(K)$.
    If $\langle v_0,v_1\rangle\in C_1(K)$, then $q\circ\partial_1(\langle v_0,v_1\rangle)=q(\langle v_1\rangle-\langle v_1\rangle)=0$ since $v_0$ and $v_1$ are joined by a path.
    This means that $B_0(K)\subset\ker q$.
    Conversely, $\ker q$ is generated by elements of $C_0(K)$ of the form $\langle w\rangle-\langle v\rangle$ where $v,w$ are in the same path components of $|K|$.
    But then there is a sequence of vertices $v=v_1,v_2,\ldots,v_n=w$ in $K$ with $\langle v_i,v_{i+1}\rangle\in K$.
    Then
    $$\langle w\rangle-\langle v\rangle=(\langle v_n\rangle-\langle v_{n-1}\rangle)+(\langle v_{n-1}\rangle-\langle v_{n-2}\rangle)\cdots +(\langle v_2\rangle-\langle v_1\rangle)\in\operatorname{Im}\partial_1=B_0(K)$$
    Hence $\ker q\subset B_0(K)$.
    This completes the proof.
\end{proof}
\begin{remark}
    There is a very rough analogy between $\pi_1(|K|)$ and $H_1(K)$ as $B_1(K)$ is kind of homotopies between the loops in $Z_1(K)$.
    They are of course not the same as $H_1(K)$ is always abelian, but in fact, there does exist a certain connection as one can show that $H_1(K)\cong\pi_1(|K|)^{\operatorname{ab}}$.
\end{remark}
\subsection{Chain Maps and Homotopies}
We want to understand the maps on homology that is induced by maps of simplicial complexes.
\begin{definition}
    A chain complex $C_\bullet$ is a sequence of abelian groups $C_n,n\in\mathbb Z$ with homomorphisms $\partial_n:C_n\to C_{n-1}$ such that $\partial_{n-1}\circ\partial_n=0$ for all $n$.\\
    A chain map $f_\bullet:C_\bullet\to D_\bullet$ between chain complexes is a collection of homomorphisms $f_n:C_n\to D_n$ indexed by $n\in\mathbb Z$ such that
    \[
        \begin{tikzcd}
            C_n\arrow{r}{\partial_n}\arrow[swap]{d}{f_n}&C_{n-1}\arrow{d}{f_{n-1}}\\
            D_n\arrow[swap]{r}{\partial_n}&D_{n-1}
        \end{tikzcd}
    \]
    commutes for any $n$.
\end{definition}
Homology usually deals with the cases where nontrivial groups only occur at nonnegative $n$.
In these situations, we can just define $C_n$ for $n\ge 0$ and $\partial_n$ for $n\ge 1$ -- because those are what we care about -- and leave the rest of the groups and boundary maps to be zero.
This will be the case for the simplicial complexes.
\begin{definition}
    Given a chain complex $C_\bullet$, we define the group of $n$-cycles to be $Z_n(C_\bullet)=\ker\partial_n$ and the group of $n$-boundaries to be $B_n(C_\bullet)=\operatorname{Im}\partial_n$.
    Then $B_n(C_\bullet)\unlhd Z_n(C_\bullet)$, so we define the $n^{th}$ homology group is then $H_n(C_\bullet)=Z_n(C_\bullet)/B_n(C_\bullet)$.
\end{definition}
\begin{lemma}
    If $f_\bullet:C_\bullet\to D_\bullet$ is a chain map, then for any $n\in\mathbb Z$, we have a well-defined homomorphism $f_\ast:H_n(C_\bullet)\to H_n(D_\bullet)$ via $[c]\mapsto [f_n(c)]$ for $c\in Z_n(C_\bullet)$.
\end{lemma}
\begin{proof}
    Suffices to show that the map is well-defined.
    For $c\in Z_n(C_\bullet)$, we have $\partial_n\circ f_n(c)=f_{n-1}\circ\partial_n(c)=0$, so indeed $f_n(c)\in Z_n(D_\bullet)$.
    Also, if $c\in B_n(C_\bullet)$, then there is some $c'\in C_{n-1}$ such that $c=\partial_{n+1}(c')$, so $f_n(c)=f_n\circ\partial_{n+1}(c')=\partial_{n+1}\circ f_{n+1}(c')$, therefore $f_n(c)\in B_n(D_\bullet)$.
    Hence $f_\ast$ is well-defined.
\end{proof}
\begin{lemma}
    A simplicial map $f:K\to L$ induces a chain map $f_\bullet:C_\bullet(K)\to C_\bullet(L)$ via
    $$f_n:\sigma\mapsto\begin{cases}
        f(\sigma)\text{, if $\dim f(\sigma)=n$}\\
        0\text{, otherwise}
    \end{cases}$$
    for $\sigma\in K,\dim\sigma=n$.
    Hence for each $n\in\mathbb N$, $f_\bullet$ induces a homomorphism $f_\ast:H_n(K)\to H_n(L)$.
\end{lemma}
\begin{proof}
    We need to show that $\partial_n\circ f_n=f_{n-1}\circ\partial_n$.
    Suffices to demonstrate this on generators.
    Let $\sigma=\langle v_0,\ldots,v_n\rangle$.
    If $\dim f(\sigma)=n$, then $f(\sigma)=\langle f(v_0),\ldots,f(v_n)\rangle$, then there is a one-to-one correspondence between faces of $\sigma$ and fases of $f(\sigma)$, hence necessarily $f_{n-1}\circ\partial_n(\sigma)=\partial_n\circ f_n(\sigma)$.
    If $\dim f(\sigma)\le n-2$, then $f_{n-1}\circ\partial_n(\sigma)=0=\partial_n\circ f_n(\sigma)$.
    We are left with the case $\dim f(\sigma)=n-1$.
    Assume $f(v_0)=f(v_1)$ and $f(v_1),\ldots,f(v_n)$ are all distinct.
    In that case $f(\langle v_0,\ldots,v_n\rangle)=f(\langle v_1,\ldots,v_n\rangle)$.
    We know that $f_n(\sigma)=0$, so $\partial_n\circ f_n(\sigma)=0$.
    Now
    \begin{align*}
        f_{n-1}\circ\partial_n(\sigma)&=f_{n-1}\left(\sum_{i=0}^n(-1)^i\langle v_0,\ldots,\hat{v}_i,\ldots,v_n\rangle\right)\\
        &=\sum_{i=0}^n(-1)^if_{n-1}(\langle v_0,\ldots,\hat{v}_i,\ldots,v_n\rangle)\\
        &=f_{n-1}(\langle v_1,\ldots,v_n\rangle)-f_{n-1}(\langle v_0,v_2,\ldots,v_n\rangle)\\
        &=0
    \end{align*}
    Therefore $\partial_n\circ f_n(\sigma)=0=f_{n-1}\circ\partial_n(\sigma)$ too, which means $f_\bullet$ is indeed a chain map.
\end{proof}
\begin{remark}
    If $f:K\to L$ and $g:L\to M$ are simplicial maps, then $(g\circ f)_\ast=g_\ast\circ f_\ast$.
    Also, if $K$ is a simplicial complex, then $(\operatorname{id}_K)_\ast=\operatorname{id}_{H_n(K)}$.
\end{remark}
A natural question is that when do chain maps induce the same maps on homology.
\begin{definition}
    Let $f_\bullet,g_\bullet:C_\bullet\to D_\bullet$ be chain maps.
    A chain homotopy $h_\bullet$ between $f_\bullet$ and $g_\bullet$ is a collection of homomorphisms $h_n:C_n\to D_{n+1}$ such that $g_n(c)-f_n(c)=\partial_{n+1}\circ h_n(c)+h_{n-1}\circ\partial_n(c)$.
    We say $f_\bullet$ and $g_\bullet$ are chain homotopic, written as $f_\bullet\simeq g_\bullet$ if such $h_\bullet$ exists.
\end{definition}
\begin{lemma}
    If $f_\bullet\simeq g_\bullet:C_\bullet\to D_\bullet$, then $f_\ast=g_\ast$.
\end{lemma}
\begin{proof}
    Let $c\in Z_n(C_\bullet)$, then $g_n(c)-f_n(c)=\partial_{n+1}\circ h_n(c)+h_{n-1}\circ\partial_n(c)=\partial_{n+1}\circ h_n(c)\in B_n(D_\bullet)$, therefore $[g_n(c)]=[f_n(c)]$.
\end{proof}
\begin{example}
    Consider the triangle $K$ and the line segment $L$, both as simplicial complexes in the obvious way.
    Say the vertices in $K$ are $e_0,e_1,e_2$ and those in $L$ are $e_0,e_1$ and let $i:L\to K$ be the natural inclusion $e_0\mapsto e_0,e_1\mapsto e_1$, $r$ be the simplicial retraction $e_0\mapsto e_0,e_1\mapsto e_1,e_2\mapsto e_0$, both as simplicial maps.
    Now $r\circ i=\operatorname{id}_L$, but $i\circ r\neq \operatorname{id}_K$.
    However, we can define a chain homotopy between $(i\circ r)_\bullet$ and $\operatorname{id}_{C_\bullet(K)}$.
    This would be given by $h_\bullet$ which is everywhere zero except $h_0(\langle e_2\rangle)=\langle e_2,e_0\rangle$ and $h_1(\langle e_1,e_2\rangle)=-\langle e_0,e_1,e_2\rangle$ which works since
    \begin{align*}
        (\partial_1\circ h_0+h_{-1}\circ\partial_0)(\langle e_2\rangle)&=\partial_1(\langle e_2,e_0\rangle)\\
        &=\langle e_0\rangle-\langle e_2\rangle\\
        &=(i_0\circ r_0-\operatorname{id}_{C_0(K)})(\langle e_2\rangle)\\
        (\partial_2\circ h_1+h_0\circ\partial_1)(\langle e_1,e_2\rangle)&=\partial_2(-\langle e_0,e_1,e_2\rangle)+h_0(\langle e_2\rangle-\langle e_1\rangle)\\
        &=-\langle e_0,e_1\rangle-\langle e_1,e_2\rangle-\langle e_2,e_0\rangle+\langle e_2,e_0\rangle\\
        &=(i_1\circ r_1-\operatorname{id}_{C_1(K)})(\langle e_1,e_2\rangle)\\
        (\partial_3\circ h_2+h_1\circ\partial_2)(\langle e_0,e_1,e_2\rangle)&=h_1(\langle e_0,e_1\rangle+\langle e_1,e_2\rangle+\langle e_2,e_0\rangle)\\
        &=-\langle e_0,e_1,e_2\rangle\\
        &=(i_2\circ r_2-\operatorname{id}_{C_2(K)})(\langle e_0,e_1,e_2\rangle)
    \end{align*}
    Therefore $(i\circ r)_\ast=i_\ast\circ r_\ast$ would just be the identity on $H_n(K)$.
    In particular, $r_\ast$ is an isomorphism on the homology groups.
\end{example}
\begin{definition}
    A simplicial complex $K$ is a cone if there is a vertex $x_0$ such that for all other simplices $\tau\in K$, there exists $\sigma\in K$ such that $x_0\in\sigma$ and $\tau\le\sigma$.
\end{definition}
Perhaps nonsurprisingly,
\begin{lemma}
    If $K$ is a cone, then $H_0(K)\cong\mathbb Z$ and $H_n(K)=0$ if $n\neq 0$.
\end{lemma}
\begin{proof}
    Let $i:\{\langle x_0\rangle\}\to K$ be the obvious inclusion and $r:K\to\{\langle x_0\rangle\}$ be constant.
    Then $r\circ i=\operatorname{id}_{\{\langle x_0\rangle\}}$, therefore $r_\ast\circ i_\ast=\operatorname{id}_{H_n(\{\langle x_0\rangle\})}$.
    We shall show that $i_\ast\circ r_\ast=\operatorname{id}_{H_n(K)}$ which implies $H_n(K)\cong H_n(\{\langle x_0\rangle\})$ from where the result follows.\\
    We will build a chain homotopy between $\operatorname{id}_{C_\bullet(K)}$ and $i_\bullet\circ r_\bullet$ where $i_\bullet$ and $r_\bullet$ are the induced chain maps.
    Let $\sigma=\langle v_0,\ldots,v_n\rangle\in K$, then we define
    $$h_n(\sigma)=\begin{cases}
        0\text{, if $x_0\in\sigma$}\\
        \langle x_0,v_0,\ldots,v_n\rangle\text{, otherwise}
    \end{cases}$$
    which is well-defined as $K$ is a cone with vertex $x_0$.
    We want to show that $\partial_{n+1}\circ h_n+h_{n-1}\circ\partial_n=\operatorname{id}_{C_n(K)}-i_n\circ r_n$.
    Suppose $n>0$ and $x_0\notin\sigma$, then
    \begin{align*}
        &\quad(\partial_{n+1}\circ h_n+h_{n-1}\circ\partial_n)(\sigma)\\
        &=\partial_{n+1}(\langle x_0,v_0,\ldots,v_n\rangle)+h_{n-1}\left( \sum_{i=0}^n(-1)^i\langle v_0,\ldots,\hat{v}_i,\ldots,v_n \rangle\right)\\
        &=\langle v_0,\ldots,v_n\rangle+\sum_{i=0}^n(-1)^{i+1}\langle x_0,v_0,\ldots,\hat{v}_i,\ldots,v_n \rangle\\
        &\quad+\sum_{i=0}^n(-1)^i\langle x_0,v_0,\ldots,\hat{v}_i,\ldots,v_n \rangle\\
        &=\langle v_0,\ldots,v_n\rangle=\sigma\\
        &=(\operatorname{id}_{C_n(K)}-i_n\circ r_n)(\sigma)
    \end{align*}
    which works.
    If $n>0$ but $x_0\in\sigma$, then $x_0=v_j$ for some $j$.
    Consequently,
    \begin{align*}
        &\quad(\partial_{n+1}\circ h_n+h_{n-1}\circ\partial_n)(\sigma)\\
        &=0+h_{n-1}\left( \sum_{i=0}^n(-1)^i\langle v_0,\ldots,\hat{v}_i,\ldots,v_n \rangle\right)\\
        &=(-1)^j\langle x_0=v_j,v_0,\ldots,\hat{v}_j,\ldots,v_n\rangle\\
        &=\langle v_0,\ldots,v_n\rangle=\sigma\\
        &=(\operatorname{id}_{C_n(K)}-i_n\circ r_n)(\sigma)
    \end{align*}
    The case $n=0$ is trivial.
    Therefore $h_\bullet$ is indeed a chain homotopy as desired.
\end{proof}
\begin{example}
    Any $n$-simplex $K$ is a cone, so by the lemma
    $$H_n(K)\cong\begin{cases}
        \mathbb Z\text{, if $n=0$}\\
        0\text{, otherwise}
    \end{cases}$$
    Take $L=\partial\sigma_n$ be the boundary of the $n$-simplex, then $|L|\cong S^{n-1}$ for $n\ge 2$.
    The obvious inclusion $L\hookrightarrow K$ of simplicial complexes induces a chain map
    \[
        \begin{tikzcd}
            0\arrow{r}&0\arrow{r}\arrow{d}&C_{n-1}(L)\arrow{r}{\partial_{n-1}}\arrow[equal]{d}&\cdots\arrow{r}{\partial_1}&C_0(L)\arrow{r}\arrow[equal]{d}&0\\
            0\arrow{r}&C_n(K)\arrow[swap]{r}{\partial_n}&C_{n-1}(K)\arrow[swap]{r}{\partial_{n-1}}&\cdots\arrow[swap]{r}{\partial_1}&C_0(K)\arrow{r}&0
        \end{tikzcd}
    \]
    So evidently $H_d(L)\cong H_d(K)$ for $d\le n-2$.
    Since $H_{n-1}(K)=0$, we have $Z_{n-1}(L)=Z_{n-1}(K)=B_{n-1}(K)$.
    Also $B_{n-1}(L)=0$, therefore $H_{n-1}(L)\cong Z_{n-1}(L)=B_{n-1}(K)$.
    But this is easy enough to calculate:
    $K$ only has one $n$-simplex $\sigma$, so $C_n(K)=\mathbb Z\sigma$ and hence $B_{n-1}\cong\mathbb Z$ as $\partial_n$ has to be injective.
    Thus
    $$H_d(L)=\begin{cases}
        \mathbb Z\text{, if $d=0$ or $d=n-1$}\\
        0\text{, otherwise}
    \end{cases}$$
    This means homology can actually detect ``higher dimensional holes''.
\end{example}
\subsection{Continuous Maps and Homotopies}
For a map $\phi:|K|\to|L|$, we want to associate to it a homomorphism $\phi_\ast:H_n(K)\to H_n(L)$.
Note that a chief difficulty in this is that $\phi$ may not contain much information about the structures of $K,L$ as simplicial complexes, since it is just a continuous map between topological spaces.
The idea is to use a simplicial approximation.
That is, instead of looking for $\phi_\ast:H_n(K)\to H_n(L)$ directly, we seek $\phi_\ast:H_n(K^{(r)})\to H_n(L^{(r)})$ for sufficiently large $r$ and show that $H_n(K^{(r)})\cong H_n(K)$ for any simplicial complex $K$ and $r\in\mathbb N$.
Eventually, we will show that $H_n(K)$ only depends on $|K|$.\\
First step on that journey is the notion of homotopy on simplicial maps.
\begin{definition}
    Two simplicial maps $f,g:K\to L$ are contiguous if, for every $\sigma\in K$, there exists some $\tau\in L$ such that $f(\sigma)$ and $g(\sigma)$ are both faces of $\tau$.
\end{definition}
\begin{remark}
    Suppose given $\phi:|K|\to |L|$ with $f,g:K\to L$ different simplicial approximations to $\phi$.
    Then choose any $x\in\sigma^\circ,\phi(x)\in\tau^\circ$, we know that $f(\sigma)\le\tau$ and $g(\sigma)\le\tau$ and hence $f,g$ are contiguous.
\end{remark}
\begin{lemma}
    If $f,g:K\to L$ are contiguous, then $f_\ast=g_\ast:H_n(K)\to H_n(L)$ for all $n$.
\end{lemma}
\begin{proof}
    We shall construct a chain homotopy.
    Fix a total order $<$ on vertices of $K$ and use the convention that $\sigma=\langle v_0,\ldots,v_n\rangle$ is oriented in such a way that $v_0<\ldots<v_n$.
    We write $\langle v_0,\ldots,v_n\rangle=0$ if $v_0,\ldots,v_n$ are not in general position.
    Easy to see this is compatible with everything.
    Define $h_n:C_n(K)\to C_{n+1}(L)$ by
    $$h_n(\langle v_0,\ldots,v_n\rangle)=\sum_{i=0}^n(-1)^i\langle f(v_0),\ldots,f(v_i),g(v_i),\ldots,g(v_n)\rangle$$
    Note that whenever the summand is nonzero, it would be an $(n+1)$-simplex of $L$ since $f,g$ are contiguous.
    We have some calculations to do.
    \begin{align*}
        (\partial\circ h+h\circ\partial)(\sigma)&=\partial\left( \sum_{i=0}^n(-1)^i\langle f(v_0),\ldots,f(v_i),g(v_i),\ldots,g(v_n)\rangle \right)\\
        &\quad+h\left( \sum_{i=0}^n(-1)^i\langle v_0,\ldots,\hat{v}_i,\ldots,v_n\rangle \right)\\
        &=\sum_{i\le j}(-1)^{i+j}\langle f(v_0),\ldots,\widehat{f(v_i)},\ldots,f(v_j),g(v_j),\ldots,g(v_n)\rangle\\
        &\quad-\sum_{i\ge j}(-1)^{i+j}\langle f(v_0),\ldots, f(v_j),g(v_j),\ldots,\widehat{g(v_i)},\ldots,g(v_n)\rangle\\
        &\quad+\sum_{j<i}(-1)^{i+j}\langle f(v_0),\ldots, f(v_j),g(v_j),\ldots,\widehat{g(v_i)},\ldots,g(v_n)\rangle\\
        &\quad-\sum_{j>i}(-1)^{i+j}\langle f(v_0),\ldots,\widehat{f(v_i)},\ldots,f(v_j),g(v_j),\ldots,g(v_n)\rangle\\
        &=\sum_{i=0}^n\langle f(v_0),\ldots,f(v_{i-1}),g(v_i),\ldots,g(v_n)\rangle\\
        &\quad-\sum_{i=0}^n\langle f(v_0),\ldots,f(v_i),g(v_{i+1}),\ldots,g(v_n)\rangle\\
        &=\langle g(v_0),\ldots,g(v_n)\rangle-\langle f(v_0),\ldots,f(v_n)\rangle\\
        &=g(\sigma)-f(\sigma)
    \end{align*}
    as desired.
\end{proof}
\begin{lemma}
    Let $K$ be a simplicial complex and $K'$ be its barycentric subdivision.
    A simplicial map $s:K'\to K$ is a simplicial approximation to $\operatorname{id}_{|K|}$ iff for every $\sigma\in K$, $s(\hat\sigma)$ is a vertex of $\sigma$.
    Also, such $s$ always exists.
\end{lemma}
\begin{proof}
    Let $s:K'\to K$ be a simplicial approximation to $\operatorname{id}_{|K|}$, which just means $\operatorname{id}_{|K|}(\operatorname{St}_{K'}(\hat\sigma))\subset\operatorname{St}_K(s(\hat\sigma))$.
    In particular, $\sigma\circ\subset\operatorname{St}_K(s(\hat\sigma))$, therefore $s(\hat\sigma)$ is a vertex of $\sigma$.\\
    Conversely, suppose $s(\hat\sigma)$ be a vertex of $\sigma$ for any $\sigma\in K$.
    Choose any $\tau'\in K'$ with $\tau'^\circ\subset\operatorname{St}_{K'}(\hat\sigma)$ (i.e. $\hat\sigma$ is a vertex of $\tau'$).
    Then $\tau'^\circ$ is contained in the interior of a simplex $\tau\in K$ such that $\sigma\le\tau$.
    Thus $s(\hat\sigma)$ is also a vertex of $\tau$.
    But $\tau'^\circ\subset\tau^\circ\subset\operatorname{St}_K(s(\hat\sigma))$.
    But such $\tau'^\circ$ necessarily cover $\operatorname{St}_{K'}(\hat\sigma)$, therefore $\operatorname{id}_{|K|}(\operatorname{St}_{K'}(\hat\sigma))\subset\operatorname{St}_K(s(\hat\sigma))$ as desired.\\
    To see such an $s$ exists, we simply just need to send $\hat\sigma$ to an arbitrarily chosen vertex of $\sigma$ which, as one can verify, works.
\end{proof}
\begin{proposition}\label{barycentric_iso_homol}
    Let $s:K'\to K$ be the simplicial approximaion to the identity obtained like in the proof above, then $s_\ast:H_n(K')\to H_n(K)$ is an isomorphism for all $n$.
\end{proposition}
We will postpone this proof until more machinery is developed.
But let us see some implications first.
\begin{corollary}
    Let $K$ be a simplicial complex, then for all $r$, there is a canonical isomorphism $H_n(K)\cong H_n(K^{(r)})$.
\end{corollary}
\begin{proof}
    Suffices to make the choice for $r=1$.
    Choose a simplicial approximation $s:K'\to K$ to the identity on $|K|$ which induces an ismorphism $s_\ast:H_n(K')\to H_n(K)$ by the preceding proposition.
    To see this is canonical, we shall show that this isomorphism is independent of the choice of $s$.
    But this is obvious since any other choice $s'$ is contiguous with $s$.
\end{proof}
We write $\nu_{K,r,s}:H_n(K^{(r)})\to H_n(K^{(s)})$ as the canonical ismorphism for $r\ge s$ and write $\nu_{K,r}=\nu_{K,r,0}$.
Then $\nu_{K,r_2,r_3}\circ \nu_{K,r_1,r_2}=\nu_{K,r_1,r_3}$.
\begin{proposition}
    To each continuous map $f:|K|\to|L|$, there is an associated homomorphism $f_\ast:H_n(K)\to H_n(L)$ given by $f_\ast=s_\ast\circ\nu_{K,r}^{-1}$ where $s:K^{(r)}\to L$ is a simplicial approximation to $f$.
    This homomorphism does not depend on the choice of $r$ or $s$.
    Furthermore, if $g:|M|\to |K|$ is continuous for some other simplicial complex $M$, then $(f\circ g)_\ast=f_\ast\circ g_\ast$.
\end{proposition}
\begin{proof}
    We already know that the homomorphism does not depend on the choice of $s$.
    If $s:K^{(r)}\to L,t:K^{(q)}\to L$ are both simplicial approximations to $f$ where WLOG $r\ge q$, then let $a:K^{(r)}\to K^{(q)}$ be a simplicial approximation to the identity on $|K|=|K^{(q)}|$.
    Now $s,t\circ a:K^{(r)}\to L$ are both simplicial approximations to $f$, hence are contiguous and induces the same homomorphism $s_\ast=(t\circ a)_\ast=t_\ast\circ a_\ast=t_\ast\circ\nu_{K,r,q}$, so $s_\ast\circ\nu_{K,r}^{-1}=t_\ast\circ\nu_{K,r,q}\circ\nu_{K,r}^{-1}=t_\ast\circ\nu_{K,q}$ as desired.\\
    Now let $s:K^{r}\to L$ and $t:M^{q}\to K^{r}$ be simplicial approximations to $f,g$ respectively (here we used $|K|=|K^{(r)}|$).
    Then $s\circ t$ is a simplicial approximation of $f\circ g$, so
    $$(f\circ g)_\ast=(s\circ t)_\ast\circ\nu_{M,q}^{-1}=s_\ast\circ t_\ast\circ \nu_{M,q}^{-1}=(s_\ast\circ\nu_{K,r}^{-1})\circ(\nu_{K,r}\circ t_\ast\circ\nu_{M,q}^{-1})=f_\ast\circ g_\ast$$
    as desired.
\end{proof}
And now we finally arrive at:
\begin{corollary}
    If $|K|\cong|L|$, then $H_n(K)\cong H_n(L)$.
\end{corollary}
\begin{proof}
    Immediate.
\end{proof}
We can do even better.
\begin{lemma}
    If $L$ is a simplicial complex residing in $\mathbb R^m$, then there exists $\epsilon=\epsilon(L)>0$ such that if $f,g:|K|\to|L|$ satisfies $|f(x)-g(x)|<\epsilon$ for any $x\in |K|$, then $f_\ast=g_\ast$.
\end{lemma}
\begin{proof}
    The set $\{\operatorname{St}_L(w):w\in L\}$ forms an open cover of $|L|$.
    So by Lebesgue number lemma, there exists $\epsilon>0$ such that each ball of radius $2\epsilon$ in $L$ lies in some $\operatorname{St}_L(w)$.
    We take $\epsilon(L)=\epsilon$.
    Let $f,g:|K|\to|L|$ as in the statement and consider the open cover of $|K|$ given by $\{f^{-1}(B_\epsilon(y)):y\in L\}$ which admits $\delta>0$ such that each $B_\delta(x)$ is contained in some member of this cover again by Lebesgue number lemma.
    This would mean that $f(B_\delta(x))\subset B_\epsilon(x)$, so $g(B_\delta(x))\subset B_{2\epsilon}(y)$.
    Choose some large $r$ such that $\operatorname{mesh}(K^{(r)})<\delta/2$, then for each vertex $v\in K^{(r)}$, the diameter of $\operatorname{St}_{K^{(r)}}(v)$ is strictly less than $\delta$, so both $f(\operatorname{St}_{K^{(r)}}(v))$ and $g(\operatorname{St}_{K^{(r)}}(v))$ are contained in some $\operatorname{St}_L(w)$.
    Set $s(v)=w$, then $s$ is a simplicial approximation to both $f$ and $g$, hence $f_\ast=s_\ast\circ \nu_{K,r}^{-1}=g_\ast$.
\end{proof}
\begin{theorem}
    If two maps $f,g:|K|\to|L|$ are homotopic, then $f_\ast=g_\ast$.
\end{theorem}
\begin{proof}
    Let $H:|K|\times I\to |L|$ be the homotopu between $f,g$, so $H(\cdot,0)=f,H(\cdot,1)=g$.
    As $|K|\times I$ is compact, $H$ is uniformly continuous.
    Thus for $\epsilon=\epsilon(L)$ as in the preceding lemma, there is some $\delta>0$ such that $|H(x,s)-H(x,t)|<\epsilon$ whenever $|s-t|<\delta$.
    Now choose $0=t_0<t_1<\ldots<t_k=1$ such that $t_i-t_{i-1}<\delta$ for any $i$ and let $f_i(x)=H(x,t_i)$.
    By construction $|f_i(x)-f_{i-1}(x)|<\epsilon$ for any $x\in |K|$, therefore $(f_i)_\ast=(f_{i-1})_\ast$ for all $i$.
    In particular, $f_\ast=(f_0)_\ast=\cdots=(f_k)_\ast=g_\ast$.
\end{proof}
\begin{corollary}
    If $|K|,|L|$ are homotopy equivalent, then $H_n(K)\cong H_n(L)$ for all $n$.
\end{corollary}
\begin{proof}
    Follows directly.
\end{proof}
\begin{definition}
    We write $H_n(X)=H_n(K)$ if $X=|K|$.
\end{definition}