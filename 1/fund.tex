\section{The Fundamental Group}
\subsection{Basic Ideas}
Fix a space $X$ and a point $x_0\in X$.
We consider loops based at $x_0$, i.e. maps $\gamma:I\to X$ with $\gamma(0)=\gamma(1)=x_0$.
\begin{example}
    We can take $X=\mathbb R^2\setminus{(0,0)}$ and $x_0\in X$.
    We can take $\gamma$ to be a loop that doesn't come near $(0,0)$ at all, or one that encloses it.
\end{example}
There can be many loops satisfying these conditions, of course, but is it necessary to consider all of them differently?
For example, we certainly wish to consider two loops to be the same if one can become the other by a ``small perturbation''.\\
To do this, we will define a notion of equivalence relationship characterising this ``equal after perturbation'' condition.
The fundamental group of $X$ at $x_0$, as a set, will be the set of equivalence classes of loops defined in this way.\\
But how would we make this a group?
Certainly, we want a group operation defined there.
Pick two loops starting and finishing at $x_0$, we want to define their product as the loop that first goes through one loop, then the other.
As one expect, this may or may not be commutative, but we do expect it to be associative and has the obvious identity and inverse.
More importantly, we do not yet know if it is well defined on the equivalence class of loops.
So we need some technicalities.
\subsection{Homotopy}
\begin{definition}
    Let $f_0,f_1:X\to Y$ be maps.
    A homotopy between $f_0$ and $f_1$ is a map $F:X\times I\to Y$ such that $F(x,0)=f_0(x)$ and $F(x,1)=f_1(x)$.
    We often write $f_t(x)=F(x,t)$ to represent the interpretation of $F$ as some kind of deformation.\\
    If such a map exists for $f_0,f_1$, we say $f_0$ is homotopic to $f_1$, written as $f_0\simeq_F f_1$ or simply $f_0\simeq f_1$.
\end{definition}
\begin{example}
    If $Y\subset\mathbb R^2$ is convex, then any maps $f_0,f_1:X\to Y$ are homotopic by taking $F(x,t)=tf_0(x)+(1-t)f_1(x)$.
\end{example}
\begin{definition}
    For $f_0,f_1:X\to Y$ and $f_0\simeq_F f_1$, if $Z\subset X$ has the property that $F(z,t)=f_0(z)=f_1(z)$ for any $z\in Z,t\in I$, then we say $f_0\simeq f_1$ relative to $Z$.
\end{definition}
\begin{lemma}
    Let $Z\subset X,Y$ be spaces.
    Then $\simeq$ relative to $Z$ is an equivalence relation on the set of continuous maps $X\to Y$
\end{lemma}
\begin{proof}
    Trivial but let's write it.
    $f_0\simeq f_0$ via $F(x,t)=f_0(x)$ so it is reflexive.
    If $f_0\simeq_F f_1$, then $f_1\simeq_{F'}f_0$ via $F^\prime(x,t)=F(x,1-t)$.\\
    If $f_0\simeq_{F_0}f_1$ and $f_1\simeq_{F_1}f_2$, then $f_0\simeq_F f_2$ via
    $$F(x,t)=\begin{cases}
        F_0(x,2t)\text{, for $t\in[0,1/2]$}\\
        F_1(x,2t-1)\text{, for $t\in[1/2,1]$}
    \end{cases}$$
    whose continuity is guaranteed by the gluing lemma.
\end{proof}
Recall that a map $f:X\to Y$ is a homeomorphism if it is a bijection and has continuous inverse (in addition to it being continuous itself).
We can extend this idea to characterise two spaces being ``homotopically the same''.
\begin{definition}
    A homotopy equivalence between spaces $X,Y$ is a map $f:X\to Y$ such that there exists a map $g:Y\to X$ such that $f\circ g\simeq \operatorname{id}_Y,g\circ f\simeq\operatorname{id}_X$.
    If such a map exists, we say $X$ and $Y$ are homotopy equivalent.
\end{definition}
Obviously homeomorphic spaces are homotopy equivalent, but the converse is not true as we shall see.
\begin{example}
    The letters `$\delta$' and `$o$', as topological spaces, are homotopy equivalent.
    But obviously they are not homeomorphic.
\end{example}
\begin{remark}
    All of the invariants of this course are homotopy invariants, as we will see.
\end{remark}
\begin{example}
    Let $\ast$ be the one-point space and $f:\mathbb R^n\to\ast$ the unique map and $g:\ast\to\mathbb R^n$ constantly $0$.
    Then $f\circ g=\operatorname{id}_\ast\simeq\operatorname{id}_\ast$.
    Now $g\circ f$ is the zero map, which is not the identity but is homotopically equivalent to the identity via $F(x,t)=tx$.
\end{example}
\begin{definition}
    If $X$ is homotopy equivalent to $\ast$, we say $X$ is contractible.
\end{definition}
\begin{example}
    Let $f:S^{n-1}\hookrightarrow\mathbb R^n\setminus\{0\}$be the inclusion and $g:\mathbb R^n\setminus\{0\}\to S^{n-1}$ be $g(x)=x/\|x\|$.
    Then $g\circ f=\operatorname{id}_{S^{n-1}}$.
    Although $f\circ g\neq\operatorname{id}_{\mathbb R^n\setminus\{0\}}$, we can consider
    $$F(x,t)=(1-t)x+t\frac{x}{\|x\|}$$
    which is a homotopy between $f\circ g$ and $\operatorname{id}_{\mathbb R^n\setminus\{0\}}$.
    Therefore $S^{n-1}\simeq \mathbb R^n\setminus\{0\}$.
\end{example}
\begin{definition}
    Let $f:X\to Y,g:Y\to X$ be maps.
    If $g\circ f=\operatorname{id}_X$, we say $X$ is a retract of $Y$ and $g$ is a retraction.\\
    If in addition that $f\circ g\simeq \operatorname{id}_Y$ relative to $f(X)$, then we say $X$ is a deformation retract of $Y$.
\end{definition}
\begin{lemma}
    Homotopy equivalences of spaces is an equivalence relation on spaces.
\end{lemma}
We got a bit imprecise here as we usually learnt equivalence relations on sets, but the collection of all topological spaces is a proper class.
Nevertheless, we can still simply check for reflexivity, symmetry and transitivity.
\begin{proof}
    Reflexivity and symmetry is obvious.
    For transitivity, suppose the maps shown below are homotopy equivalences:
    \[
        \begin{tikzcd}
            X \arrow[bend left]{r}{f} & Y \arrow[bend left]{l}{g} \arrow[bend left]{r}{f'} & Z \arrow[bend left]{l}{g'}
        \end{tikzcd}
    \]
    Then obviously we want to show $f'\circ f$ and $g\circ g'$ are homotopy inverses of each other.
    Suppose $g'\circ f'\simeq_{F'}\operatorname{id}_Y$, then the function
    $$(x,t)\mapsto g\circ F'(f(x),t)$$
    is a homotopy between $g\circ f$ and $g\circ (g'\circ f')\circ f=(g\circ g')\circ (f'\circ f)$.
    Therefore $(g\circ g')\circ (f'\circ f)\simeq g\circ f\simeq \operatorname{id}_X$.
    Using the exact same idea, $(f'\circ f)\circ (g\circ g')\simeq\operatorname{id}_Z$.
\end{proof}
\subsection{Loops and the Fundamental Group}
\begin{definition}
    Let $X$ be a space, a path in $X$ is a map $\gamma:I\to X$.
    It is a path from $x_0$ to $x_1$ ($x_0,x_1\in X$) if $\gamma(0)=x_0$ and $\gamma(1)=x_1$.\\
    A loop based at $x_0\in X$ is a path from $x_0$ to $x_0$.
\end{definition}
\begin{definition}
    Let $\gamma_0,\gamma_1$ be paths from $x_0$ to $x_1$.
    We say $\gamma_0$ and $\gamma_1$ are (path-)homotopic if they are homotopic relative to $\{0,1\}$.\\
    This has been shown to be an equivalence relation.
    We write $[\gamma]$ to denote the equivalence class containing $\gamma$ in the set of all paths from $x_0$ to $x_1$.
\end{definition}
\begin{definition}
    Let $X$ be a space and $x,y,z\in X$.
    Let $\gamma_1$ be a path from $x$ to $y$ and $\gamma_2$ a path from $y$ to $z$.\\
    1. The concatenation $\gamma_1\cdot \gamma_2$ of $\gamma_1$ and $\gamma_2$ is the path from $x$ to $z$ defined by
    $$(\gamma_1\cdot \gamma_2)(t)=\begin{cases}
        \gamma_1(2t)\text{, for $t\in [0,1/2]$}\\
        \gamma_2(2t-1)\text{, for $t\in [1/2,1]$}
    \end{cases}$$
    which is a proper path as it is continuous due to the gluing lemma.\\
    2. The constant path at $x$ is the constant function $c_x:t\mapsto x$.\\
    3. The inverse of a path $\gamma_1$ is a path $\bar\gamma_1$ from $y$ to $x$ defined by $\bar\gamma_1(t)=\gamma_1(1-t)$.
\end{definition}
\begin{theorem}\label{fund_group}
    Let $X$ be a space and $x_0\in X$.
    Write $\pi_1(X,x_0)$ to denote the set of homotopy classes of loops based at $x_0$.
    Then $\pi_1(X,x_0)$ is a group under the operation $[\gamma_1][\gamma_2]=[\gamma_1\cdot\gamma_2]$, with identity $[c_{x_0}]$ and $[\gamma]^{-1}=[\bar\gamma]$.
\end{theorem}
This group is called the fundamental group of $X$.
\begin{lemma}\label{fund_group_well_def}
    If $\gamma_0\simeq\gamma_1$ are paths to $y$ and $\delta_0\simeq \delta_1$ are paths from $y$, then $\gamma_0\cdot\delta_0\simeq\gamma_1\cdot\delta_1$ and $\bar\gamma_0\simeq\bar\gamma_1$.
\end{lemma}
\begin{proof}
    Suppose $\gamma_0\simeq_F\gamma_1,\delta_0\simeq_G\delta_1$.
    Define
    $$H(s,t)=\begin{cases}
        F(2s,t)\text{, for $s\in[0,1/2]$}\\
        G(2s-1,t)\text{, for $s\in[1/2,1]$}
    \end{cases}$$
    Then we immediately have $\gamma_0\cdot\delta_0\simeq_H\gamma_1\cdot\delta_1$.\\
    Now for the inverses, $F'(s,t)=F(1-s,t)$ gives $\bar\gamma_0\simeq_{F'}\bar\gamma_1$.
\end{proof}
\begin{lemma}\label{fund_group_ax}
    Let $x,y,z\in X$.
    If there are paths $\alpha$ from $w$ to $x$, $\beta$ from $x$ to $y$, $\gamma$ from $y$ to $z$.
    Then\\
    1. $(\alpha\cdot\beta)\cdot\gamma\simeq\alpha\cdot(\beta\cdot\gamma)$.\\
    2. $\alpha\cdot c_x\simeq c_x\cdot\alpha\simeq\alpha$.\\
    3. $\alpha\cdot\bar\alpha\simeq c_x$.
\end{lemma}
\begin{proof}
    Note that the composition of any path $\gamma$ and an order-preserving surjection $\phi I\to I$ is homotopic to the original path via $F(s,t)=\gamma(t\phi(s)+(1-t)s)$.
    This is called a reparameterisation.
    Now take $\phi$ to be the function
    $$\phi(t)=\begin{cases}
        t/2\text{, for $t\in[1,1/2]$}\\
        t-1/4\text{, for $t\in[1/2,3/4]$}\\
        2t-1\text{, for $t\in[3/4,1]$}
    \end{cases}$$
    Then, as one can check
    $$(\alpha\cdot\beta)\cdot\gamma\simeq((\alpha\cdot\beta)\cdot\gamma)\circ\phi=\alpha\cdot(\beta\cdot\gamma)$$
    To see $\alpha\cdot c_x\simeq\alpha$, just reparameterise $\alpha$ by
    $$t\mapsto\begin{cases}
        2t\text{, for $t\in [0,1/2]$}\\
        1\text{, for $t\in[1/2,1]$}
    \end{cases}$$
    The other side is analogous.\\
    Indeed $c_x\simeq_F\alpha\cdot\bar\alpha$ via
    $$F(s,t)=\begin{cases}
        \alpha(2s)\text{, for $s\in[0,t/2]$}\\
        \alpha(t)\text{, for $s\in[t/2,1-t/2]$}\\
        \alpha(2-2s)=\bar\alpha(2s-1)\text{, for $s\in[1-t/2,1]$}
    \end{cases}$$
    which can be verified to work.
\end{proof}
\begin{proof}[Proof of Theorem \ref{fund_group}]
    Combining Lemma \ref{fund_group_well_def} and Lemma \ref{fund_group_ax} shows the result immediately.
\end{proof}
\begin{example}
    Consider $X=\mathbb R^n$ and $x_0=0$, then for any loop $\gamma$ based at $x_0$ we have $\gamma\simeq c_{x_0}$ via the straightline homotopy $F(x,t)=(1-t)\gamma(x)+tx_0$.
    Therefore $\pi_1(X,x_0)$ is the trivial group.
\end{example}
Now, as we mentioned in the introduction, we still want a property of this algebraic invariant regarding maps between the relevant objects.
\begin{lemma}
    Let $f:X\to Y$ be a map, $x_0\in X$ and $y_0=f(x_0)$.
    Then there is an induced group homomorphism $f_\ast:\pi_1(X,x_0)\to\pi_1(Y,y_0)$ defined by $f_\ast([\gamma])=[f\circ \gamma]$.
    Further:\\
    1. If $f\simeq f'$ relative to $x_0$, then $f_\ast=f_\ast'$.\\
    2. if $g:Y\to Z$ with $g(y_0)=z_0$ is another map, then $g_\ast\circ f_\ast=(g\circ f)_\ast$.\\
    3. $(\operatorname{id}_X)_\ast=\operatorname{id}_{\pi_1(X,x_0)}$.
\end{lemma}
\begin{proof}
    $f_\ast$ is always well-defined as a function.
    Suppose $\gamma_1\simeq_F\gamma_2$, then we obviously have $f\circ \gamma_1\simeq_{f\circ F}f\circ \gamma_2$.\\
    To see it is a group homomorphism, we observe
    $$f\circ(\gamma_1\cdot\gamma_2)=(f\circ \gamma_1)\cdot(f\circ\gamma_2)\implies f_\ast([\gamma_1\cdot\gamma_2])=f_\ast([\gamma_1])\cdot f_\ast([\gamma_2])$$
    For 1, if $f\simeq_F f'$ relative to $x_0$, then $f\circ\gamma\simeq_{F'} f'\circ\gamma$ via $F'(s,t)=F(\gamma(s),t)$, so $f_\ast([\gamma])=[f\circ\gamma]=[f'\circ\gamma]=f_\ast'([\gamma])$.
    2 and 3 are completely obvious.
\end{proof}
One thing we are not satisfied:
We define the fundamental group with reference to a basepoint.
Is there a way to remove it, at least for path-connected space?
\begin{lemma}\label{indep_basepoint}
    Let $X$ be a space.
    A path $\alpha$ from $x_0$ to $x_1$ induces a group isomorphism $\alpha_\#:\pi_1(X,x_0)\to\pi_1(X,x_1)$ via $\alpha_\#([\gamma])=[\bar\alpha\cdot\gamma\cdot\alpha]$.
    Further:\\
    1. If $\alpha\simeq\alpha'$, then $\alpha_\#=\alpha_\#'$.\\
    2. $(c_{x_0})_\#=\operatorname{id}_{\pi_1(X,x_0)}$.\\
    3. If $\beta$ is a path from $x_1$ to $x_2$, then $(\alpha\cdot\beta)_\#=\beta_\#\circ\alpha_\#$.\\
    4. If $f:X\to Y$ is a map with $y_i=f(x_i)$, then
    $$(f\circ\alpha)_\#\circ f_\ast=f_\ast\circ\alpha_\#$$
\end{lemma}
In short, the path $\alpha_\#(\gamma)$ goes from $x_1$, walk through $\bar\alpha$ to $x_0$, do the loop, then go back to $x_1$ via $\alpha$.
\begin{proof}
    It is very easy to see that $\alpha$ is well-defined.
    To see it is a group homomorphism, observe that for loops $\gamma,\delta$ based at $x_0$, we have
    \begin{align*}
        \alpha_\#(\gamma)\cdot\alpha_\#(\delta)&\simeq(\bar\alpha\cdot\gamma\cdot\alpha)\cdot(\bar\alpha\cdot\delta\cdot\alpha)\\
        &\simeq\bar\alpha\cdot\gamma\cdot(\alpha\cdot\bar\alpha)\cdot\delta\cdot\alpha\\
        &\simeq\bar\alpha\cdot(\gamma\cdot\delta)\cdot \alpha\\
        &\simeq\alpha_\#(\gamma\cdot\delta)
    \end{align*}
    Note also that $\bar\alpha_\#$ has
    $$\bar\alpha_\#\circ\alpha_\#(\gamma)\simeq \alpha\cdot(\bar\alpha\cdot\gamma\cdot\alpha)\cdot\bar\alpha\simeq(\alpha\cdot\bar\alpha)\cdot\gamma\cdot(\alpha\cdot\bar\alpha)\simeq c_{x_0}\cdot\gamma\cdot c_{x_0}\simeq\gamma$$
    for any $\gamma$, so $\bar\alpha_\#$ is inverse to $\alpha_\#$, hence $\alpha_\#$ is indeed a group isomorphism.\\
    1,2,3 are completely trivial.
    For 4, we basically just want
    \[
        \begin{tikzcd}
            \pi_1(X,x_0)\arrow{r}{\alpha_\#}\arrow[swap]{d}{f_\ast}&\pi_1(X,x_1)\arrow{d}{f_\ast}\\
            \pi_1(Y,y_0)\arrow[swap]{r}{(f\circ\alpha)_\#}&\pi_1(Y,y_1)
        \end{tikzcd}
    \]
    to commute.
    To see this,
    \begin{align*}
        ((f\circ\alpha)_\#\circ f_\ast)([\gamma])&=(f\circ\alpha)_\#([f\circ\gamma])\\
        &=[(\overline{f\circ\alpha})\cdot(f\circ\gamma)\cdot (f\circ\alpha)]\\
        &=[f\circ(\bar\alpha\cdot\gamma\cdot\alpha)]\\
        &=f_\ast(\alpha_\#(\gamma))
    \end{align*}
    As we want.
\end{proof}
In particular, the fundamental group does not depend on the basepoint if the space is path-connected.
\begin{definition}
    If $X$ is a path-connected soace and $\pi_1(X,x_0)$ is trivial for some (hence every) $x_0\in X$, then we say $X$ is simply connected.
\end{definition}
\begin{lemma}\label{hom_commute_path}
    Let $x_0\in X_1$ and $f,g:X\to Y$ with $f\simeq_Fg$.
    Set $\alpha(t)=F(x_0,t)$ a path from $f(x_0)$ to $g(x_0)$.
    Then
    \[
        \begin{tikzcd}
            \pi_1(Y,f(x_0))\arrow{r}{\alpha_\#}&\pi_1(Y,g(x_0))\\
            \pi_1(X,x_0)\arrow{u}{f_\ast}\arrow[swap]{ur}{g_\ast}&
        \end{tikzcd}
    \]
    commutes.
\end{lemma}
\begin{proof}
    We need to check that for a loop $\gamma$ based at $x_0$, we have $\bar\alpha\cdot(f\circ\gamma)\cdot\alpha\simeq g\circ\gamma$.
    Consider $G:I\times I\to Y$ defined by $G(s,t)=F(\gamma(s),t)$.
    Let $a,b_1,b_2,b_3,b:I\to I^2$ be paths defined by
    $$a(t)=(t,1),b_1(t)=(0,1-t),b_2(t)=(t,0),b_3(t)=(1,t),b=b_1\cdot b_2\cdot b_3$$
    Easily $a\simeq b$.
    Then $(G\circ a)(s)=G(s,1)=F(\gamma(s),1)=(g\circ\gamma)(s)$, so $G\circ a=g\circ\gamma$.
    Calculation shows that $G\circ b_1=\bar\alpha, G\circ b_2=f\circ\gamma, G\circ b_3=\alpha$.
    Hence
    $$g\circ\gamma\simeq G\circ a\simeq G\circ b\simeq G\circ (b_1\cdot b_2\cdot b_3)\simeq (G\circ b_1)\cdot(G\circ b_2)\cdot(G\circ b_3)\simeq \bar\alpha\cdot(f\circ\gamma)\cdot\alpha$$
    As desired.
\end{proof}
\begin{theorem}\label{hom_eqv_iso}
    If $f:X\to Y$ is a homotopy equivalence, then for any $x_0\in X$,
    $$f_\ast:\pi_1(X,x_0)\to\pi_1(Y,f(x_0))$$
    is an isomorphism of groups.
\end{theorem}
\begin{proof}
    Let $g:Y\to X$ be an homotopy inverse of $f$.
    Then $\operatorname{id}_X\simeq_Fg\circ f$ and $\operatorname{id}_Y\simeq_Gf\circ g$ for some homotopy $F,G$.
    Let $\alpha(t)=F(x_0,t)$ be a path joining $\operatorname{id}_X(x_0)=x_0$ and $g(f(x_0))$, then by Lemma \ref{hom_commute_path} we have
    $$g_\ast\circ f_\ast=(g\circ f)_\ast=\alpha_\#\circ (\operatorname{id}_X)_\ast=\alpha_\#$$
    But $\alpha_\#$ is an isomorphism hence injective by Lemma \ref{indep_basepoint}, so in particular $f_\ast$ is an injection.
    By the same argument $\beta(t)=G(f(x_0),t)$ has $f_\ast\circ g_\ast=\beta_\#$ which is an isomorphism hence surjective, so $f_\ast$ is surjective.
    Therefore $f_\ast$ is bijective, hence an isomorphism.
\end{proof}
\begin{corollary}
    Contractible spaces are simply connected.
\end{corollary}
\begin{proof}
    By definition and Theorem \ref{hom_eqv_iso}.
\end{proof}