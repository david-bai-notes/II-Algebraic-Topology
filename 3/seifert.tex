\section{The Seifert-van Kampen Theorem}
\subsection{Some Group Theory}
Recall that we can sometimes represent groups using generators and relations, e.g. the dihedral group $D_{2n}$ can be written as $\langle r,s|s^2=r^n=srsr=1\rangle$ meaning it is the group generated by $r,s$ subject to the given relations $s^2=r^n=1,srs=r^{-1}$.
\begin{definition}
    Let $A$ be a set.
    The free group on $A$ is a group $F(A)$ such that there is a function $\phi:A\to F(A)$ satisfying the following universal property:\\
    Whenever there is a group $G$ and a function $h:A\to G$, there exists a unique homomorphism $f:F(A)\to G$ such that
    \[
        \begin{tikzcd}
            F(A)\arrow[dashed]{dr}{f}&\\
            A\arrow{u}{\phi} \arrow[swap]{r}{h}&G
        \end{tikzcd}
    \]
    commutes.
\end{definition}
\begin{example}
    Let $A=\{\alpha\}$ and $\phi:A\to\mathbb Z$ be the map $\alpha\mapsto 1$.
    Then for any $h:A\to G$ mapping $\alpha$ to $g\in G$, we necessarily have $f:\mathbb Z\to G$ via $n\mapsto g^n$ such that $f\circ\phi=h$.
    Hence $\mathbb Z$ is the free group on $\{a\}$.
\end{example}
\begin{remark}
    This definition, as a universal property, does not give an explicit construction of $F(A)$ but captured all of its group theorectical properties.
    As such, if such an $F(A)$ exists it must be unique up to group isomorphism -- that's why we used ``the'' free group instead of ``a'' free group.
\end{remark}
\begin{proposition}
    Let $A$ be a set and $F(A),F'(A)$ be both free groups on $A$, then $F(A)\cong F'(A)$ as groups.
\end{proposition}
\begin{proof}
    Suppose $\phi:A\to F(A),\phi':A\to F'(A)$ be the maps described in the definition.
    Take $G=F'(A)$ gives a unique homomorphism $f:F(A)\to F'(A)$ such that $f\circ\phi=\phi'$ since $F(A)$ s a free group on $A$; $G=F(A)$ also gives a unique homomorphism $g:F'(A)\to F(A)$ such that $g\circ\phi'=\phi$.
    We claim that $f,g$ are inverse to each other.
    Now $f\circ g$ makes the diagram
    \[
        \begin{tikzcd}
            F'(A)\arrow{dr}{f\circ g}&\\
            A\arrow{u}{\phi'}\arrow[swap]{r}{\phi'}&F'(A)
        \end{tikzcd}
    \]
    commute.
    But then since $F'(A)$ is free, $f\circ g$ must equal to $\operatorname{id}_{F'(A)}$ (which also makes the diagram commute when putting it in the place of $f\circ g$) by uniqueness.
    Similarly $g\circ f=\operatorname{id}_{F(A)}$.
    Hence $f,g$ are indeed inverses of each other.
    This shows $F(A)\cong F'(A)$.
\end{proof}
The proof also shows that the isomorphism is uniquely determined if we ask it to be compatible with $\phi,\phi'$.
In other word, it is canonical with respect to this universal property.
\begin{definition}
    If $A$ is a finite set of cardinality $r$, then we write $F_r=F(A)$ to be the free group of rank $r$.
\end{definition}
\begin{definition}
    The words in $A$ are the strings composed of elements in $A$ and their (symbolic) inverses.
\end{definition}
For example, $w=aba^{-1}bba^{-1}a^{-1}b$ is a word in $A$.
We can canonically identify a word in $A$ by an element of $F(A)$ by mapping it to the product (with the same order of elements) of the respective images, e.g. $abba^{-1}b$ can be identified with $\phi(a)\phi(b)^2\phi(a)^{-1}\phi(b)\in F(A)$.\\
It is easy to check that $\phi(A)$ also satisfies the same universal property of a free group on $A$, therefore by uniqueness we can assume $F(A)$ is generated by $\phi(A)$.
So $F(A)$ can be seen as the set of words in $A$, possibly with some identifications (e.g. $aa^{-1}b=b$).
As a recreational exercise, one can precisely construct a free group on any given set in this way.
After the construction, we can happily identify $A$ as a generating subset of $F(A)$.
\begin{definition}
    A presentation is a set $A$ and a subset of relations $R\subset F(A)$ which identifies the group
    $$\langle A|R\rangle=F(A)/\langle\langle R\rangle\rangle$$
    where $\langle\langle R\rangle\rangle$ is the smallest normal subgroup of $F(A)$ containing $R$ (i.e. intersection of all normal subgroups containing $R$).
    The presentation is finite if both $A,R$ are.
\end{definition}
Easily $\langle\langle R\rangle\rangle=\langle\{srs^{-1}:s\in F(A),r\in R\}\rangle$.
\begin{lemma}[Universal Property of Presentation]
    Let $q:F(A)\to\langle A|R\rangle$ be the quotient map.
    Whenever $f:F(A)\to G$ is a group homomorphism such that $R\subset\ker f$, there is a unique homomorphism $g:\langle A|R\rangle\to G$ such that
    \[
        \begin{tikzcd}
            \langle A|R\rangle\arrow[dashed]{dr}{g}&\\
            F(A)\arrow{u}{q}\arrow[swap]{r}{f}&G
        \end{tikzcd}
    \]
    commutes.
\end{lemma}
\begin{proof}
    Necessarily $g(w\langle\langle R\rangle\rangle)=f(w)$ which works since $\langle\langle R\rangle\rangle\le\ker f$ by definition.
\end{proof}
\begin{example}
    1. We know that $F(\{a\})\cong\mathbb Z$ and every subgroup of $\mathbb Z$ is normal, so $\langle\langle a^n\rangle\rangle=\langle a^n\rangle$ corresponds to $n\mathbb Z$, so $\langle a|a^n\rangle\cong\mathbb Z/n\mathbb Z$.\\
    2. We claim that $G=\langle r,s|r^n,s^2,rsrs\rangle$ is isomorphic to $D_{2n}$.
    Indeed, the homomorphism $F(\{r,s\})\to D_{2n}$ sending $r$ to a rotation of $D_{2n}$ and $s$ to a reflection takes $R=\{r^n,s^2,rsrs\}$ to the identity.
    Thus it factors through $G$ via $\phi:G\to D_{2n}$ by the universal property.
    Now $\phi$ is obviously surjective and injective since we can write $G=\{1,r,\ldots,r^{n-1},s,sr,\ldots,sr^{n-1}\}$ which has the correct size.\\
    3. Every group has a presentation.
    The identity map from $G$ as a set to $G$ as a group induces a group homomorphism $F(G)\to G$ which is surjective.
    Let $R$ be the kernel of this map, then $G=\langle G|R\rangle$.
    (This is, however, not very useful since this is a very inefficient choice of generators and relations.)
\end{example}
\begin{definition}[Pushouts]
    Consider a commutative square
    \[
        \begin{tikzcd}
            \Gamma&A\arrow[swap]{l}{k}\\
            B\arrow{u}{l}&C\arrow{l}{j}\arrow[swap]{u}{i}
        \end{tikzcd}
    \]
    It is called a pushout if it satisfies the following universal property:
    If $G$ is a group with homomorphism $f:A\to G,g:B\to G$ such that
    \[
        \begin{tikzcd}
            G&A\arrow[swap]{l}{f}\\
            B\arrow{u}{g}&C\arrow{l}{j}\arrow[swap]{u}{i}
        \end{tikzcd}
    \]
    commutes, then there is a unique $\phi:\Gamma\to G$ such that
    \[
        \begin{tikzcd}
            G&&\\
            &\Gamma\arrow[swap,dashed]{ul}{\exists!\phi}&A\arrow[swap]{l}{k}\arrow[swap, bend right]{ull}{f}\\
            &B\arrow{u}{l}\arrow[bend left]{uul}{g}&C\arrow{l}{j}\arrow[swap]{u}{i}
        \end{tikzcd}
    \]
    commutes.\\
    If this is indeed the case, then we write $\Gamma=A\sqcup_CB$ (the arrows $i,j$ are equipped with $A,B$ and $k,l$ are equipped with $\Gamma$).
\end{definition}
Given $A,B,C,i,j$, one can check that such a $\Gamma$ is unique up to isomorphism.
\begin{definition}
    If $C=\{e\}$ and $i,j$ be the unique homomorphisms from $C$ to $A,B$, then $A\sqcup_CB$ is called the free product, denotes by $A\star B$.\\
    If $i,j$ are injective, then $A\coprod_CB$ is called the amalgamated product and is written as $A\star_CB$.
\end{definition}
\begin{lemma}\label{pushout_onetrivial}
    For $i:C\to A,j:C\to B=\{e\}$, we have $A\sqcup_C\{e\}\cong A/\langle\langle i(C)\rangle\rangle$.
\end{lemma}
\begin{proof}
    Take $q:A\to A/\langle\langle i(C)\rangle\rangle$ to be the quotient map and $\iota:\{e\}\to A/\langle\langle i(C)\rangle\rangle$ the natural inclusion, then the diagram
    \[
        \begin{tikzcd}
            A/\langle\langle i(C)\rangle\rangle&A\arrow[swap]{l}{q}\\
            \{e\}\arrow{u}{\iota}&C\arrow{l}{j}\arrow[swap]{u}{i}
        \end{tikzcd}
    \]
    does commute.
    Now suppose that there are $f,g$ such that $j\circ g=i\circ f$, i.e. the bigger diagram
    \[
        \begin{tikzcd}
            G&&\\
            &\Gamma\arrow[swap,dashed]{ul}{?}&A\arrow[swap]{l}{q}\arrow[swap, bend right]{ull}{f}\\
            &\{e\}\arrow{u}{\iota}\arrow[bend left]{uul}{g}&C\arrow{l}{j}\arrow[swap]{u}{i}
        \end{tikzcd}
    \]
    commutes.
    Then by commutativity, $f\circ i$ is the constant homomorphism since $g$ (hence $j\circ g$) has to be, so $f(i(C))=\{e\}\in G$, therefore $i(C)\subset\ker f$.
    As $f$ is normal, necessarily $\langle\langle i(C)\rangle\rangle\subset\ker f$, so we necessarily have to choose
    $$\phi:A/\langle\langle i(C)\rangle\rangle\to G,w\langle\langle i(C)\rangle\rangle\mapsto f(w)$$
    which works.
\end{proof}
Do pushouts always exist?
\begin{lemma}
    Let $A=\langle S_1|R_1\rangle$ and $B=\langle S_2|R_2\rangle$ and let $T\subset C$ be a generating set for $C$.
    Suppose $\tilde\imath:T\to F(S_1)$ is a lift of a function $i|_T$ and $\tilde\jmath:T\to F(S_2)$ is a lift of $j|_T$ (so $q_1\circ\tilde\imath=i,q_2\circ\tilde\jmath=j$ on $T$ where $q_1:F(S_1)\to A,q_2:F(S_2)\to B$ are the projections), then
    $$\Gamma=\langle S_1\sqcup S_2|R_1\cup R_2\cup \{\tilde\imath(t)^{-1}\tilde\jmath(t):t\in T\}\rangle$$
    is a presentation of $A\sqcup_CB$.
\end{lemma}
\begin{proof}
    Again we check the universal property.
    Suppose there is a group $G$ with $f:A\to G,g:B\to G$ homomorphisms such that $j\circ g=i\circ f$, then we have the commutative diagram
    \[
        \begin{tikzcd}
            G&&&\\
            &\Gamma\arrow[swap,dashed]{ul}{?}&A\arrow[swap]{l}{k}\arrow[swap, bend right]{ull}{f}&F(S_1)\arrow[swap]{l}{q_1}\\
            &B\arrow{u}{l}\arrow[bend left]{uul}{g}&C\arrow{l}{j}\arrow[swap]{u}{i}\arrow[hookleftarrow]{dr}&\\
            &F(S_2)\arrow{u}{q_2}&&T\arrow{ll}{\tilde\jmath}\arrow[swap]{uu}{\tilde\imath}
        \end{tikzcd}
    \]
    where $k,l$ are induced by the natural inclusions $S_1\hookrightarrow S_1\sqcup S_2,S_2\hookrightarrow S_1\sqcup S_2$ via the universal property of presentations.
    Let $\tilde{f}=f\circ q_1$ and $\tilde{g}=g\circ q_2$.
    Then $\tilde{f}(R_1)=\tilde{g}(R_2)=\{e\}\subset G$ and $\tilde{f}\circ\tilde\imath=\tilde{g}\circ\tilde\jmath$.
    Now let $\phi:F(S_1\sqcup S_2)\to G$ induced by $\tilde{f}$ on $S_1$ and $\tilde{g}$ on $S_2$.
    But then we have $\phi(R_1\cup R_2)=0$ by definiton and $\phi(\tilde\imath(t)^{-1}\tilde\jmath(t))=e$ for any $t\in T$ since $\tilde{f}\circ\tilde\imath=\tilde{g}\circ\tilde\jmath$.
    So by the universal property of presentations, $\phi$ induces the desired map.
\end{proof}
\subsection{The Seifert-van Kampen Theorem}
\begin{definition}
    let $(X,x_0),(Y,y_0)$ be based spaces, then the wedge of $X,Y$ is $X\vee Y=(X\sqcup Y)/\sim$ where $\sim$ is the smallest equivalence relations such that $x_0\sim y_0$.
    The equivalence class $[x_0]=[y_0]$ is the wedge point.
\end{definition}
So we are basically just glueing $X,Y$ together by attaching $x_0$ to $y_0$.
\begin{theorem}[Seifert-van Kampen Theorem for Wedges]\label{s-vk_wedge}
    Suppose $X=Y_1\vee Y_2$ where $Y_1,Y_2$ are path connected and let $x_0\in X$ be the wedge point.
    Then $\pi_1(X,x_0)\cong\pi_1(Y_1,x_0)\star\pi_1(Y_2,x_0)$.
\end{theorem}
\begin{proof}[Sketch of proof]
    Let $i_1:Y_1\to X,i_2:Y_2\to X$ be the natural inclusions which are continuous.
    We shall attempt to show
    \[
        \begin{tikzcd}
            \pi_1(X,x_0)&\pi_1(Y_1,x_0)\arrow[swap]{l}{(i_1)_\ast}\\
            \pi_1(Y_2,x_0)\arrow{u}{(i_2)_\ast}&\{e\}\arrow{u}\arrow{l}
        \end{tikzcd}
    \]
    is a pushout.
    Suppose we are given $f_1:\pi_1(Y_1,x_0)\to G$ and $f_2:\pi_1(Y_2,x_0)\to G$ for some group $G$.
    We need to prove that there is a unique map $g:\pi_1(X,x_0)\to G$ such that $g\circ (i_1)_\ast=f_1,g\circ (i_2)_\ast=f_2$.
    Given a loop in $X$, write it as a concatenation $\gamma=\alpha_1\beta_1\alpha_2\beta_2\cdots\alpha_n\beta_n$ with $\alpha_i$ loops in $Y_1$ and $\beta_i$ loops in $Y_2$.
    We leave out the details of the proof that it is always possible.
    Hence we necessarily have $g([\gamma])=f_1([\alpha_1])f_2([\beta_1])\cdots f_1([\alpha_n])f_2([\beta_n])$, which one can check is well-defined and works.
\end{proof}
\begin{example}
    1. The figure 8 has fundamental group $\pi_1(S^1\vee S^1)\cong\pi_1(S^1)\star\pi_1(S^1)\cong\mathbb Z\star\mathbb Z\cong F_2$.
    Worth noting that this group is nonabelian.\\
    2. Let $A$ be any finite set, then define $\bigvee_AS^1=(A\times S^1)/\sim$ where $\sim$ is the smallest equivalence relation such that $(a,1)\sim (a',1)$ for any $a,a'\in A$.
    Then inductively $\pi_1(\bigvee_AS^1,1)\cong F_{|A|}$.
    In particular, there exists spaces whose fundamental group is $F_n$ for any positive integer $n$.
\end{example}
\begin{theorem}[Seifert-van Kampen Theorem]\label{s-vk_open}
    Suppose $Y_1,Y_2\subset X$ are open, $X=Y_1\cup Y_2$ and $Z=Y_1\cap Y_2$ is nonempty.
    Suppose also that they are all path-connected.
    Let $i_k:Z\hookrightarrow Y_k$ and $j_k:Y_k\hookrightarrow X$ be the inclusions and fix $x_0\in Z$.
    Then the diagram
    \[
        \begin{tikzcd}
            \pi_1(X,x_0)&\pi_1(Y_2,x_0)\arrow[swap]{l}{(j_2)_\ast}\\
            \pi_1(Y_1,x_0)\arrow{u}{(j_1)_\ast}&\pi_1(Z,x_0)\arrow[swap]{u}{(i_2)_\ast}\arrow{l}{(i_1)_\ast}
        \end{tikzcd}
    \]
    is a pushout.
\end{theorem}
\begin{proof}
    Omitted.
\end{proof}
\begin{example}
    We want to calculate the fundamental group of the $n$-sphere $S^n$ for $n\ge 2$.
    Let $x_\pm=(\pm 1,0,\ldots,0)$ and $U_\pm=S^n\setminus\{x_\pm\}$ and $V=U_+\cap U_-$.
    Now that $V=S^n\setminus\{x_+,x_-\}\cong S^{n-1}\times (-1,1)$ via
    $$(x_1,\ldots,x_{n+1})\mapsto\left( \frac{(x_2,\ldots,x_{n+1})}{|(x_2,\ldots,x_{n+1})|},x_1 \right)$$
    which is path-connected for $n\ge 2$.
    Also, $U_\pm$ are both homeomorphic to $\mathbb R^n$ via stereographic projection.
    We know that $X=U_+\cup U_-$, so by Seifert-van Kampen the diagram
    \[
        \begin{tikzcd}
            \pi_1(S^n)&\{e\}\arrow{l}\\
            \{e\}\arrow{u}&\pi_1(V)\arrow{u}\arrow{l}
        \end{tikzcd}
    \]
    is a pushout.
    But then $\pi_1(S^n)$ is necessarily the trivial group.
    Therefore $S^n$ is simply connected for $n\ge 2$.
    Note that this argument breaks down for $n=1$ since $V$ is not path-connected in that case.
\end{example}
Note that this version of Seifeit-van Kampen does not directly generalise Theorem \ref{s-vk_wedge}.
But of course we want to have a version that generalises it, so here goes.
\begin{definition}
    A subset $Y\subset X$ is called a neighbourhood retract if there is some $V\subset X$ open and contains $Y$ such that $Y$ is a deformation retract of $V$.
\end{definition}
\begin{theorem}[Seifert-van Kampen Theorem for Closed Sets]\label{s-vk_closed}
    Suppose $Y_1,Y_2\subset X$ are closed and $Y_1\cup Y_2=X$, $Y_1\cap Y_2=Z\neq\varnothing$.
    Assume everything is path-connected and $Z$ is a neighbourhood retract in both $Y_1$ and $Y_2$, then the diagram
    \[
        \begin{tikzcd}
            \pi_1(X,x_0)&\pi_1(Y_2,x_0)\arrow[swap]{l}{(j_2)_\ast}\\
            \pi_1(Y_1,x_0)\arrow{u}{(j_1)_\ast}&\pi_1(Z,x_0)\arrow[swap]{u}{(i_2)_\ast}\arrow{l}{(i_1)_\ast}
        \end{tikzcd}
    \]
    is a pushout where as usual $i_k:Z\hookrightarrow Y_k$ and $j_k:Y_k\hookrightarrow X$ are the inclusions and $x_0\in Z$.
\end{theorem}
\begin{proof}
    Also omitted.
\end{proof}
\subsection{Attaching Cells}
\begin{definition}
    Let $X$ be a space and let $\alpha:S^{n-1}\to X$ be a map.
    The space obtained by attaching an $n$-cell to $X$ is $X\cup_\alpha D^n=(X\sqcup D^n)/\sim$ where $\sim$ is the smallest equivalence relation that identifies $x\sim \alpha(x)$ for $x\in S^{n-1}$.
\end{definition}
\begin{example}
    For $n=1$, we are just attaching a string to two (possibly one) points on $X$.
    For $n=2$, things can get pretty complicated.
    Although we are just identifying $S^1$ with a loop in $X$ and attach $D^n$ there accordingly, this can give many varieties as the loop can intersect and/or wind itself.
\end{example}
We want to study what happen to the fundamental group when we attach an $n$-cell.
\begin{lemma}
    If $n\ge 3$ and $i:X\to X\cup_\alpha D^n$ be the natural inclusion.
    The $i_\ast$ is an isomorphism of fundamental groups.
\end{lemma}
\begin{proof}
    We are going to construct something called the mapping cylinder of $\alpha$, which is the space
    $$M_\alpha=(X\sqcup(S^{n-1}\times I))/\sim$$
    where $\sim$ is the smallest equivalence relation containing $(\theta,0)\sim\alpha(\theta)$ where $\theta\in S^{n-1}$.
    Now identify $S^{n-1}$ with $S^{n-1}\times \{1\}\subset M_\alpha$ which is now a neighbourhood retract of $M_\alpha$.
    Note that $X\cup_\alpha D^n\cong M_\alpha\cup_{\operatorname{id}_{S^{n-1}}}D^n$.
    Also, $S^{n-1}$ is a neighbourhood retract in $D^n$.
    Now by Theorem \ref{s-vk_closed} with basepoint $x\in S^{n-1}\subset M_\alpha\cup_{\operatorname{id}_{S^{n-1}}}D^n$ and take $Y_1=M_\alpha$, $Y_2=D^n$ and $Z=Y_1\cap Y_2=S^{n-1}$.
    This gives us the pushout
    \[
        \begin{tikzcd}
            \pi_1(X\cup_\alpha D^n,x_0)&\pi_1(M_\alpha,x_0)\arrow{l}\\
            \pi_1(D^n,x_0)\arrow{u}&\pi_1(S^{n-1},x_0)\arrow{u}\arrow{l}
        \end{tikzcd}
    \]
    Now for $n\ge 3$, $\pi_1(S^{n-1},x_0)=\pi_1(D^n,x_0)=\{e\}$.
    which means that $\pi_1(X\cup_\alpha D^n,x_0)\cong\pi_1(M_\alpha,x_0)$.
    But $X$ is obviously a deformation retract of $M_\alpha$, therefore $\pi_1(X,x_0')\cong\pi_1(M_\alpha,x_0)\cong\pi_1(X\cup_\alpha D^n,x_0)$ where $x_0'$ is the image of $x_0$ under the deformation retract.
\end{proof}
What if we attach a $2$-cell?
\begin{lemma}
    Let $\alpha:S^1\to X$  be a map and $x_0=\alpha(\theta_0)$ for some $\theta_0\in S^1$.
    Then
    $$\pi_1(X\cup_\alpha D^2,x_0)\cong\pi_1(X,x_0)/\langle\langle [\alpha]\rangle\rangle$$
    viewing $\alpha$ as a loop based at $x_0$.
    The quotient map, in particular, is induced by the inclusion map $i:X\to X\cup_\alpha D^2$.
\end{lemma}
\begin{proof}
    By the same procedure as above we obtain the pushout
    \[
        \begin{tikzcd}
            \pi_1(X\cup_\alpha D^n,x_0)&\pi_1(X,x_0)\arrow{l}\\
            \pi_1(D^2,\theta_0)=\{e\}\arrow{u}&\pi_1(S^1,\theta_0)=\mathbb Z\arrow{l}\arrow[swap]{u}{1\mapsto [\alpha]}
        \end{tikzcd}
    \]
    The result then follows from Lemma \ref{pushout_onetrivial}.
\end{proof}
\begin{theorem}
    For any finitely presented group $G$, i.e. $G\cong\langle A|R\rangle$ where $A,R$ are both finite, there exists a compact space $X$ with $\pi_1(X,x_0)\cong G$ for some $x_0\in X$.
\end{theorem}
\begin{proof}
    Let $Y=\bigvee_AS^1$ and let $y_0$ be the common wedge point.
    We already know that $\pi_1(Y,y_0)\cong F(A)$.
    Now for each relation $r\in R$, we get a loop $\alpha_r:S^1\to Y$ representing it in the obvious way.
    Then attach $D^2$ to $Y$ via $\alpha_r$ for each $r\in R$ gives a (necessarily compact) space with fundamental group isomorphic to $G$.
\end{proof}
\subsection{Classification of Surfaces}
\begin{definition}
    An $n$ dimensional (topological) manifold (or $n$-manifold) is a Hausdorff space $M$ such that every point $x\in M$ has a neighbourhood $U$ homeomorphic to an open set of $\mathbb R^n$.
\end{definition}
\begin{example}
    1. The $S^n$ is an $n$-manifold.\\
    2. (non-example) The figure 8 is not a manifold since the wedge point does not have a neighbourhood that is locally homeomorphic to $\mathbb R^n$.\\
    3. Take $\alpha:S^1\to X=\{\ast\}$, then $Y=X\cup_\alpha D^2\cong S^2$ is a $2$-manifold.
\end{example}
There is a more interesting and influential example:
Let $g$ be a positive integer and let
$$\Gamma_{2g}=\bigvee_{i=1}^{2g}S^1_i$$
be the wedge product of $2g$ copies $S_i^1$ of the circle with a common wedge point $x_0$.
Let $\alpha_i:I\to S_i^1,\beta_i:I\to S_{i+g}^1$ be simple loops with basepoint $x_0$ for $i=1,\ldots,g$ and consider the loop
$$\rho_g=\alpha_1\beta_1\bar\alpha_1\bar\beta_1\cdots\alpha_g\beta_g\bar\alpha_g\bar\beta_g$$
If we think of $\rho_g$ as a map $S^1\to \Gamma_{2g}$ and define $\Sigma_g=\Gamma_{2g}\cup_{\rho_g}D^2$.
We claim that $\Sigma=\Sigma_g$ is a compact $2$-manifold.
Obviously any interior points of $D^2$ has an open neighbourhood homeomorphic to an open set in $\mathbb R^2$.
At non-wedge point in $S_i^1$, the path $\alpha_i$ (if $i\le g$) or $\beta_{i-g}$ (if $i>g$) appears with its inverse in $\rho_g$, so we can obtain a neighbourhood we want near that point as well.
At wedge point, note that $g=1$ gives the standard identification of the torus on a square, in which case the wedge point is simply the corners (which are identified as the same point) which obviously has a neighbourhood homeomorphic to an open set in $\mathbb R^2$.
An analogy works for higher $g$.
$\Sigma_g$ is called the orientable surface of genus $g$.\\
Now easily $g=1$ just gives the torus.
For $g=2$, we can think of the disk as an octagon and do a little bit of imagination by gluing the respective edges, which will give a $2$-torus, i.e. a torus with two holes.
A little bit more of imagination shows that $\Sigma_n$ is the $n$-torus.
Their fundamental groups are clear by our previous discussion:
$$\pi_i(\Sigma_g)=\langle a_1,\ldots,a_g,b_1,\ldots b_g|a_1b_1a_1^{-1}b_1^{-1}\cdots a_gb_ga_g^{-1}b_g^{-1}\rangle$$
Now take $\Gamma_{g+1}=\sum_{i=0}^gS_i^1$ as the wedge of $g+1$ circles and $\alpha_i:I\to S_i^1$ be the loop around the $i^{th}$ circle and $\sigma_g=\alpha_0\alpha_0\alpha_1\alpha_1\cdots\alpha_g\alpha_g$ viewed as a map $\sigma_g:\partial D^2\to\Gamma_{g+1}$.
Then take $S_g=\Gamma_{g+1}\cup_{\sigma_g}D^2$ which is a $2$=manifold called the non-orientable surface of genus $g$.
As one can see, $S_0$ is simply the real projective plane, and $S_1$ the Klein bottle.
$S_g$ for $g>1$ would shaped like attaching some orientable surface to the Klein bottle.
The fundamental groups are
$$\pi_1(S_g)\cong\langle a_0,\ldots,a_g|a_0^1a_1^2\cdots a_g^2\rangle$$
So for example $\pi_1(S_0)\cong\mathbb Z/2\mathbb Z$.
\begin{theorem}
    Any compact surface $S$ is homeomorphic to $S_g$ or $\Sigma_g$ for some $g$.
\end{theorem}
\begin{proof}
    Omitted.
\end{proof}
But how do we know that they are topologically distinct?
\begin{lemma}
    Let $g\in\mathbb N$, then $\pi_1(\Sigma_g)$ surjects onto $\mathbb Z^{2g}$ but not $\mathbb Z^{2g}\oplus(\mathbb Z/2\mathbb Z)$ and $\pi_1(S_g)$ surjects onto $\mathbb Z^g\oplus(\mathbb Z/2\mathbb Z)$ but not $\mathbb Z^{g+1}$.
\end{lemma}
\begin{proof}
    Let $\{\bar{a}_i,\bar{b}_i\}$ be the standard basis of $\mathbb Z^{2g}$.
    Then the map $a_i\mapsto\bar{a}_i,b_i\mapsto \bar{b}_i$ respects the relation, therefore by the universal property there is a surjective homomorphism from $\pi_1(\Sigma_g)$ to $\mathbb Z^{2g}$.
    Now suppose we have a surjective homomorphism $f:\pi_1(\Sigma_g)\to\mathbb Z^{2g}\oplus(\mathbb Z/2\mathbb Z)$.
    Compose it with the reduction $\mathbb Z^{2g}\oplus(\mathbb Z/2\mathbb Z)\to(\mathbb Z/2\mathbb Z)^{2g+1}$ gives a surjective homomorphism $f':\pi_1(\Sigma_g)\to(\mathbb Z/2\mathbb Z)^{2g+1}$.
    Therefore $f'(a_1),\ldots,f'(a_g),f'(b_1),\ldots,f'(b_g)$ should generate $(\mathbb Z/2\mathbb Z)^{2g+1}$, which is impossible by viewing $(\mathbb Z/2\mathbb Z)^{2g+1}$ as a vector space over $\mathbb Z/2\mathbb Z$ with dimension $2g+1$.\\
    For $\pi_1(S_g)$, let $\{\bar{a}_i\}$ be a basis for the $\mathbb Z^g$ part of $\mathbb Z^g\oplus(\mathbb Z/2\mathbb Z)$ and let $\bar{c}_0$ generate the $\mathbb Z/2\mathbb Z$ part.
    Then the map $a_0\mapsto \bar{c}_0-\sum_{i=1}^g\bar{a}_i, a_i\mapsto \bar{a}_i$ respects the relation, hence induces the desired surjective homomorphism $\pi_1(S_g)\to\mathbb Z^g\oplus(\mathbb Z/2\mathbb Z)$ via the universal property.
    Now if there is a surjective homomorphism $f:\pi_1(S_g)\to\mathbb Z^{g+1}$, then $\mathbb Z^{g+1}$ is generated by $f(a_0),\ldots,f(a_g)$.
    But then $0=f(\sigma_g)=2f(a_0)+\cdots+2f(a_g)$ which is a nontrivial relation between the generators $f(a_0),\ldots,f(a_g)$.
    This is a contradiction.
\end{proof}
\begin{corollary}
    $\Sigma_g$ and $S_g$ have mutually distinct fundamental groups.
\end{corollary}
In particular, they are mutually distinct in terms of homotopy equivalence and hence in terms of homeomorphism.
\begin{proof}
    If $\Sigma_g$ and $\Sigma_{g'}$ have the same fundamental group but $g<g'$, then by the preceding lemma, there is a surjection $\pi_1(\Sigma_g)=\pi_1(\Sigma_{g'})\to\mathbb Z^{g'}\to\mathbb Z^g\oplus(\mathbb Z/2\mathbb Z)$, contradiction.
    The other cases can be argued similarly.
\end{proof}
\begin{remark}
    We can easily generalise the fundamental group $\pi_1$ to higher dimension, which are known as the $n^{th}$ homotopy groups $\pi_n$ which is the set of homotopy classes of maps from the $n$-sphere $S^n$ to a based space with a slightly more complicated but analogous concatenation law that induces a group operation.
    Turns out, $\pi_n$ is always abelian for $n>1$.
    The problem, however, with this sort of groups is that they are hard to calculate.
    Even $\pi_n(S^m)$ for general $n,m$ are still unknown.
    So they are not an effective algebraic invariant -- we need a different approach called homology (and cohomology).
\end{remark}