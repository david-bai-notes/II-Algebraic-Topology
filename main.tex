\documentclass[a4paper]{article}

\usepackage{hyperref}

\newcommand{\triposcourse}{Algebraic Topology}
\newcommand{\triposterm}{Michaelmas 2020}
\newcommand{\triposlecturer}{Prof. M. Gross}
\newcommand{\tripospart}{II}

\usepackage{amsmath}
\usepackage{amssymb}
\usepackage{amsthm}
\usepackage{mathrsfs}

\usepackage{tikz-cd}

\theoremstyle{plain}
\newtheorem{theorem}{Theorem}[section]
\newtheorem{lemma}[theorem]{Lemma}
\newtheorem{proposition}[theorem]{Proposition}
\newtheorem{corollary}[theorem]{Corollary}
\newtheorem{problem}[theorem]{Problem}
\newtheorem*{claim}{Claim}

\theoremstyle{definition}
\newtheorem{definition}{Definition}[section]
\newtheorem{conjecture}{Conjecture}[section]
\newtheorem{example}{Example}[section]

\theoremstyle{remark}
\newtheorem*{remark}{Remark}
\newtheorem*{note}{Note}

\title{\triposcourse{}
\thanks{Based on the lectures under the same name taught by \triposlecturer{} in \triposterm{}.}}
\author{Zhiyuan Bai}
\date{Compiled on \today}

\setcounter{section}{-1}

\begin{document}
    \maketitle
    This document serves as a set of revision materials for the Cambridge Mathematical Tripos Part \tripospart{} course \textit{\triposcourse{}} in \triposterm{}.
    However, despite its primary focus, readers should note that it is NOT a verbatim recall of the lectures, since the author might have made further amendments in the content.
    Therefore, there should always be provisions for errors and typos while this material is being used.
    \tableofcontents
    \section{Introduction}
\subsection{Motivation and Examples}
We start by asking a natural question that appears frequently in the study of topology:
\begin{problem}\label{homeo}
    How can we tell if two topological space are different (i.e. not homoeomorphic)?
\end{problem}
For example, if we let $X$ to be the donut-like surface in $\mathbb R^3$, and $Y$ be a similar donut but with two holes instead.
Our intuition tells us they are not homeomorphic because, well, $X$ has one hole and $Y$ has two -- but this is not much a proof.\\
Our basic strategy of coping with problems like these in algebraic topology is to associate each topological space $X$ (often, in a specific class) with a group $H(X)$ and associate each continuous map $f:X\to Y$ a homomorphism of groups $H(f):H(X)\to H(Y)$.
We obviously want $H$ to preserve compositions of functions and identities as well.
\footnote{In short, we just want $H$ to be a functor from a subcategory of $\mathbf{Top}$ to an abelian category, preferably $\mathbf{Grp}$.}
With certain constructions, if $X\cong Y$ as topological spaces, we can ensure that $H(X)\cong H(Y)$ as groups.
In other words, $H$ exists as an algebraic invariant.
As it is much easier to show two groups are not isomorphic, we can solve some cases of Problem \ref{homeo} where $H(X)$ can be shown not to isomorphic to $H(Y)$.
The study of algebraic topology is basically the hunt of such algebraic invariants.\\
Another application of this idea is to solve the Extension Problem:
\begin{problem}[The Extension Problem]
    Let $X$ be a topological space and $A\subset X$ a subspace.
    If we have $f:A\to Y$ is a continuous map, how do we know if there is a continuous map $F:X\to Y$ with $F|_A=f$?
\end{problem}
That is, we want to know if there exists $F$ such that the diagram
\[
    \begin{tikzcd}
        A \arrow[hookrightarrow]{d} \arrow{r}{f} & Y\\
        X \arrow[dashed,swap]{ur}{F}&
    \end{tikzcd}
\]
commutes.
An example of this is the following theorem:
\begin{theorem}
    There is no continuous function $F:D^n\to S^{n-1}$ such that
    \[
        \begin{tikzcd}
            S^{n-1} \arrow[hookrightarrow]{d} \arrow{r}{\operatorname{id}} & S^{n-1}\\
            D^n \arrow[swap]{ur}{F}
        \end{tikzcd}
    \]
    commutes.
\end{theorem}
Here, $D^n$ is the $n$-disk while $S^{n-1}$ is the $(n-1)$-sphere, which is simply the hypersurface enclosing $D^n$.
How does the idea of the algebraic invariant come in handy on this problem?
Obviously we do not yet have the correct tool to do it now that the course has just started, but we can take a glimpse of the idea involved.
\begin{proof}
    Construct an invariant $H$ such that $H(S^{n-1})\cong\mathbb Z$ and $H(D^n)\cong 0$.
    Then the diagram in the statement of the theorem will implies that
    \[
        \begin{tikzcd}
            \mathbb Z\arrow{d} \arrow{r}{\operatorname{id}} & \mathbb Z\\
            0 \arrow{ur}
        \end{tikzcd}
    \]
    commute.
    But this is absurd.
\end{proof}
All these seems nice, right?
But first, we really need a clever construction of such a nontrivial algebraic invariants bearing such nice properties.
\subsection{Conventions}
Unless otherwise stated, we adopt the following conventions:\\
When we mention ``space'', we always mean a topological space.
And when we say a ``map'' between two spaces, we always mean a continuous one.
By $I$, we mean the unit interval $[0,1]$.
    \section{The Fundamental Group}
\subsection{Basic Ideas}
Fix a space $X$ and a point $x_0\in X$.
We consider loops based at $x_0$, i.e. maps $\gamma:I\to X$ with $\gamma(0)=\gamma(1)=x_0$.
\begin{example}
    We can take $X=\mathbb R^2\setminus{(0,0)}$ and $x_0\in X$.
    We can take $\gamma$ to be a loop that doesn't come near $(0,0)$ at all, or one that encloses it.
\end{example}
There can be many loops satisfying these conditions, of course, but is it necessary to consider all of them differently?
For example, we certainly wish to consider two loops to be the same if one can become the other by a ``small perturbation''.\\
To do this, we will define a notion of equivalence relationship characterising this ``equal after perturbation'' condition.
The fundamental group of $X$ at $x_0$, as a set, will be the set of equivalence classes of loops defined in this way.\\
But how would we make this a group?
Certainly, we want a group operation defined there.
Pick two loops starting and finishing at $x_0$, we want to define their product as the loop that first goes through one loop, then the other.
As one expect, this may or may not be commutative, but we do expect it to be associative and has the obvious identity and inverse.
More importantly, we do not yet know if it is well defined on the equivalence class of loops.
So we need some technicalities.
\subsection{Homotopy}
\begin{definition}
    Let $f_0,f_1:X\to Y$ be maps.
    A homotopy between $f_0$ and $f_1$ is a map $F:X\times I\to Y$ such that $F(x,0)=f_0(x)$ and $F(x,1)=f_1(x)$.
    We often write $f_t(x)=F(x,t)$ to represent the interpretation of $F$ as some kind of deformation.\\
    If such a map exists for $f_0,f_1$, we say $f_0$ is homotopic to $f_1$, written as $f_0\simeq_F f_1$ or simply $f_0\simeq f_1$.
\end{definition}
\begin{example}
    If $Y\subset\mathbb R^2$ is convex, then any maps $f_0,f_1:X\to Y$ are homotopic by taking $F(x,t)=tf_0(x)+(1-t)f_1(x)$.
\end{example}
\begin{definition}
    For $f_0,f_1:X\to Y$ and $f_0\simeq_F f_1$, if $Z\subset X$ has the property that $F(z,t)=f_0(z)=f_1(z)$ for any $z\in Z,t\in I$, then we say $f_0\simeq f_1$ relative to $Z$.
\end{definition}
\begin{lemma}
    Let $Z\subset X,Y$ be spaces.
    Then $\simeq$ relative to $Z$ is an equivalence relation on the set of continuous maps $X\to Y$
\end{lemma}
\begin{proof}
    Trivial but let's write it.
    $f_0\simeq f_0$ via $F(x,t)=f_0(x)$ so it is reflexive.
    If $f_0\simeq_F f_1$, then $f_1\simeq_{F'}f_0$ via $F^\prime(x,t)=F(x,1-t)$.\\
    If $f_0\simeq_{F_0}f_1$ and $f_1\simeq_{F_1}f_2$, then $f_0\simeq_F f_2$ via
    $$F(x,t)=\begin{cases}
        F_0(x,2t)\text{, for $t\in[0,1/2]$}\\
        F_1(x,2t-1)\text{, for $t\in[1/2,1]$}
    \end{cases}$$
    whose continuity is guaranteed by the gluing lemma.
\end{proof}
Recall that a map $f:X\to Y$ is a homeomorphism if it is a bijection and has continuous inverse (in addition to it being continuous itself).
We can extend this idea to characterise two spaces being ``homotopically the same''.
\begin{definition}
    A homotopy equivalence between spaces $X,Y$ is a map $f:X\to Y$ such that there exists a map $g:Y\to X$ such that $f\circ g\simeq \operatorname{id}_Y,g\circ f\simeq\operatorname{id}_X$.
    If such a map exists, we say $X$ and $Y$ are homotopy equivalent.
\end{definition}
Obviously homeomorphic spaces are homotopy equivalent, but the converse is not true as we shall see.
\begin{example}
    The letters `$\delta$' and `$o$', as topological spaces, are homotopy equivalent.
    But obviously they are not homeomorphic.
\end{example}
\begin{remark}
    All of the invariants of this course are homotopy invariants, as we will see.
\end{remark}
\begin{example}
    Let $\ast$ be the one-point space and $f:\mathbb R^n\to\ast$ the unique map and $g:\ast\to\mathbb R^n$ constantly $0$.
    Then $f\circ g=\operatorname{id}_\ast\simeq\operatorname{id}_\ast$.
    Now $g\circ f$ is the zero map, which is not the identity but is homotopically equivalent to the identity via $F(x,t)=tx$.
\end{example}
\begin{definition}
    If $X$ is homotopy equivalent to $\ast$, we say $X$ is contractible.
\end{definition}
\begin{example}
    Let $f:S^{n-1}\hookrightarrow\mathbb R^n\setminus\{0\}$be the inclusion and $g:\mathbb R^n\setminus\{0\}\to S^{n-1}$ be $g(x)=x/\|x\|$.
    Then $g\circ f=\operatorname{id}_{S^{n-1}}$.
    Although $f\circ g\neq\operatorname{id}_{\mathbb R^n\setminus\{0\}}$, we can consider
    $$F(x,t)=(1-t)x+t\frac{x}{\|x\|}$$
    which is a homotopy between $f\circ g$ and $\operatorname{id}_{\mathbb R^n\setminus\{0\}}$.
    Therefore $S^{n-1}\simeq \mathbb R^n\setminus\{0\}$.
\end{example}
\begin{definition}
    Let $f:X\to Y,g:Y\to X$ be maps.
    If $g\circ f=\operatorname{id}_X$, we say $X$ is a retract of $Y$ and $g$ is a retraction.\\
    If in addition that $f\circ g\simeq \operatorname{id}_Y$ relative to $f(X)$, then we say $X$ is a deformation retract of $Y$.
\end{definition}
\begin{lemma}
    Homotopy equivalences of spaces is an equivalence relation on spaces.
\end{lemma}
We got a bit imprecise here as we usually learnt equivalence relations on sets, but the collection of all topological spaces is a proper class.
Nevertheless, we can still simply check for reflexivity, symmetry and transitivity.
\begin{proof}
    Reflexivity and symmetry is obvious.
    For transitivity, suppose the maps shown below are homotopy equivalences:
    \[
        \begin{tikzcd}
            X \arrow[bend left]{r}{f} & Y \arrow[bend left]{l}{g} \arrow[bend left]{r}{f'} & Z \arrow[bend left]{l}{g'}
        \end{tikzcd}
    \]
    Then obviously we want to show $f'\circ f$ and $g\circ g'$ are homotopy inverses of each other.
    Suppose $g'\circ f'\simeq_{F'}\operatorname{id}_Y$, then the function
    $$(x,t)\mapsto g\circ F'(f(x),t)$$
    is a homotopy between $g\circ f$ and $g\circ (g'\circ f')\circ f=(g\circ g')\circ (f'\circ f)$.
    Therefore $(g\circ g')\circ (f'\circ f)\simeq g\circ f\simeq \operatorname{id}_X$.
    Using the exact same idea, $(f'\circ f)\circ (g\circ g')\simeq\operatorname{id}_Z$.
\end{proof}
\subsection{Loops and the Fundamental Group}
\begin{definition}
    Let $X$ be a space, a path in $X$ is a map $\gamma:I\to X$.
    It is a path from $x_0$ to $x_1$ ($x_0,x_1\in X$) if $\gamma(0)=x_0$ and $\gamma(1)=x_1$.\\
    A loop based at $x_0\in X$ is a path from $x_0$ to $x_0$.
\end{definition}
\begin{definition}
    Let $\gamma_0,\gamma_1$ be paths from $x_0$ to $x_1$.
    We say $\gamma_0$ and $\gamma_1$ are (path-)homotopic if they are homotopic relative to $\{0,1\}$.\\
    This has been shown to be an equivalence relation.
    We write $[\gamma]$ to denote the equivalence class containing $\gamma$ in the set of all paths from $x_0$ to $x_1$.
\end{definition}
\begin{definition}
    Let $X$ be a space and $x,y,z\in X$.
    Let $\gamma_1$ be a path from $x$ to $y$ and $\gamma_2$ a path from $y$ to $z$.\\
    1. The concatenation $\gamma_1\cdot \gamma_2$ of $\gamma_1$ and $\gamma_2$ is the path from $x$ to $z$ defined by
    $$(\gamma_1\cdot \gamma_2)(t)=\begin{cases}
        \gamma_1(2t)\text{, for $t\in [0,1/2]$}\\
        \gamma_2(2t-1)\text{, for $t\in [1/2,1]$}
    \end{cases}$$
    which is a proper path as it is continuous due to the gluing lemma.\\
    2. The constant path at $x$ is the constant function $c_x:t\mapsto x$.\\
    3. The inverse of a path $\gamma_1$ is a path $\bar\gamma_1$ from $y$ to $x$ defined by $\bar\gamma_1(t)=\gamma_1(1-t)$.
\end{definition}
\begin{theorem}\label{fund_group}
    Let $X$ be a space and $x_0\in X$.
    Write $\pi_1(X,x_0)$ to denote the set of homotopy classes of loops based at $x_0$.
    Then $\pi_1(X,x_0)$ is a group under the operation $[\gamma_1][\gamma_2]=[\gamma_1\cdot\gamma_2]$, with identity $[c_{x_0}]$ and $[\gamma]^{-1}=[\bar\gamma]$.
\end{theorem}
This group is called the fundamental group of $X$.
\begin{lemma}\label{fund_group_well_def}
    If $\gamma_0\simeq\gamma_1$ are paths to $y$ and $\delta_0\simeq \delta_1$ are paths from $y$, then $\gamma_0\cdot\delta_0\simeq\gamma_1\cdot\delta_1$ and $\bar\gamma_0\simeq\bar\gamma_1$.
\end{lemma}
\begin{proof}
    Suppose $\gamma_0\simeq_F\gamma_1,\delta_0\simeq_G\delta_1$.
    Define
    $$H(s,t)=\begin{cases}
        F(2s,t)\text{, for $s\in[0,1/2]$}\\
        G(2s-1,t)\text{, for $s\in[1/2,1]$}
    \end{cases}$$
    Then we immediately have $\gamma_0\cdot\delta_0\simeq_H\gamma_1\cdot\delta_1$.\\
    Now for the inverses, $F'(s,t)=F(1-s,t)$ gives $\bar\gamma_0\simeq_{F'}\bar\gamma_1$.
\end{proof}
\begin{lemma}\label{fund_group_ax}
    Let $x,y,z\in X$.
    If there are paths $\alpha$ from $w$ to $x$, $\beta$ from $x$ to $y$, $\gamma$ from $y$ to $z$.
    Then\\
    1. $(\alpha\cdot\beta)\cdot\gamma\simeq\alpha\cdot(\beta\cdot\gamma)$.\\
    2. $\alpha\cdot c_x\simeq c_x\cdot\alpha\simeq\alpha$.\\
    3. $\alpha\cdot\bar\alpha\simeq c_x$.
\end{lemma}
\begin{proof}
    Note that the composition of any path $\gamma$ and an order-preserving surjection $\phi I\to I$ is homotopic to the original path via $F(s,t)=\gamma(t\phi(s)+(1-t)s)$.
    This is called a reparameterisation.
    Now take $\phi$ to be the function
    $$\phi(t)=\begin{cases}
        t/2\text{, for $t\in[1,1/2]$}\\
        t-1/4\text{, for $t\in[1/2,3/4]$}\\
        2t-1\text{, for $t\in[3/4,1]$}
    \end{cases}$$
    Then, as one can check
    $$(\alpha\cdot\beta)\cdot\gamma\simeq((\alpha\cdot\beta)\cdot\gamma)\circ\phi=\alpha\cdot(\beta\cdot\gamma)$$
    To see $\alpha\cdot c_x\simeq\alpha$, just reparameterise $\alpha$ by
    $$t\mapsto\begin{cases}
        2t\text{, for $t\in [0,1/2]$}\\
        1\text{, for $t\in[1/2,1]$}
    \end{cases}$$
    The other side is analogous.\\
    Indeed $c_x\simeq_F\alpha\cdot\bar\alpha$ via
    $$F(s,t)=\begin{cases}
        \alpha(2s)\text{, for $s\in[0,t/2]$}\\
        \alpha(t)\text{, for $s\in[t/2,1-t/2]$}\\
        \alpha(2-2s)=\bar\alpha(2s-1)\text{, for $s\in[1-t/2,1]$}
    \end{cases}$$
    which can be verified to work.
\end{proof}
\begin{proof}[Proof of Theorem \ref{fund_group}]
    Combining Lemma \ref{fund_group_well_def} and Lemma \ref{fund_group_ax} shows the result immediately.
\end{proof}
\begin{example}
    Consider $X=\mathbb R^n$ and $x_0=0$, then for any loop $\gamma$ based at $x_0$ we have $\gamma\simeq c_{x_0}$ via the straightline homotopy $F(x,t)=(1-t)\gamma(x)+tx_0$.
    Therefore $\pi_1(X,x_0)$ is the trivial group.
\end{example}
Now, as we mentioned in the introduction, we still want a property of this algebraic invariant regarding maps between the relevant objects.
\begin{lemma}
    Let $f:X\to Y$ be a map, $x_0\in X$ and $y_0=f(x_0)$.
    Then there is an induced group homomorphism $f_\ast:\pi_1(X,x_0)\to\pi_1(Y,y_0)$ defined by $f_\ast([\gamma])=[f\circ \gamma]$.
    Further:\\
    1. If $f\simeq f'$ relative to $x_0$, then $f_\ast=f_\ast'$.\\
    2. if $g:Y\to Z$ with $g(y_0)=z_0$ is another map, then $g_\ast\circ f_\ast=(g\circ f)_\ast$.\\
    3. $(\operatorname{id}_X)_\ast=\operatorname{id}_{\pi_1(X,x_0)}$.
\end{lemma}
\begin{proof}
    $f_\ast$ is always well-defined as a function.
    Suppose $\gamma_1\simeq_F\gamma_2$, then we obviously have $f\circ \gamma_1\simeq_{f\circ F}f\circ \gamma_2$.\\
    To see it is a group homomorphism, we observe
    $$f\circ(\gamma_1\cdot\gamma_2)=(f\circ \gamma_1)\cdot(f\circ\gamma_2)\implies f_\ast([\gamma_1\cdot\gamma_2])=f_\ast([\gamma_1])\cdot f_\ast([\gamma_2])$$
    For 1, if $f\simeq_F f'$ relative to $x_0$, then $f\circ\gamma\simeq_{F'} f'\circ\gamma$ via $F'(s,t)=F(\gamma(s),t)$, so $f_\ast([\gamma])=[f\circ\gamma]=[f'\circ\gamma]=f_\ast'([\gamma])$.
    2 and 3 are completely obvious.
\end{proof}
One thing we are not satisfied:
We define the fundamental group with reference to a basepoint.
Is there a way to remove it, at least for path-connected space?
\begin{lemma}\label{indep_basepoint}
    Let $X$ be a space.
    A path $\alpha$ from $x_0$ to $x_1$ induces a group isomorphism $\alpha_\#:\pi_1(X,x_0)\to\pi_1(X,x_1)$ via $\alpha_\#([\gamma])=[\bar\alpha\cdot\gamma\cdot\alpha]$.
    Further:\\
    1. If $\alpha\simeq\alpha'$, then $\alpha_\#=\alpha_\#'$.\\
    2. $(c_{x_0})_\#=\operatorname{id}_{\pi_1(X,x_0)}$.\\
    3. If $\beta$ is a path from $x_1$ to $x_2$, then $(\alpha\cdot\beta)_\#=\beta_\#\circ\alpha_\#$.\\
    4. If $f:X\to Y$ is a map with $y_i=f(x_i)$, then
    $$(f\circ\alpha)_\#\circ f_\ast=f_\ast\circ\alpha_\#$$
\end{lemma}
In short, the path $\alpha_\#(\gamma)$ goes from $x_1$, walk through $\bar\alpha$ to $x_0$, do the loop, then go back to $x_1$ via $\alpha$.
\begin{proof}
    It is very easy to see that $\alpha$ is well-defined.
    To see it is a group homomorphism, observe that for loops $\gamma,\delta$ based at $x_0$, we have
    \begin{align*}
        \alpha_\#(\gamma)\cdot\alpha_\#(\delta)&\simeq(\bar\alpha\cdot\gamma\cdot\alpha)\cdot(\bar\alpha\cdot\delta\cdot\alpha)\\
        &\simeq\bar\alpha\cdot\gamma\cdot(\alpha\cdot\bar\alpha)\cdot\delta\cdot\alpha\\
        &\simeq\bar\alpha\cdot(\gamma\cdot\delta)\cdot \alpha\\
        &\simeq\alpha_\#(\gamma\cdot\delta)
    \end{align*}
    Note also that $\bar\alpha_\#$ has
    $$\bar\alpha_\#\circ\alpha_\#(\gamma)\simeq \alpha\cdot(\bar\alpha\cdot\gamma\cdot\alpha)\cdot\bar\alpha\simeq(\alpha\cdot\bar\alpha)\cdot\gamma\cdot(\alpha\cdot\bar\alpha)\simeq c_{x_0}\cdot\gamma\cdot c_{x_0}\simeq\gamma$$
    for any $\gamma$, so $\bar\alpha_\#$ is inverse to $\alpha_\#$, hence $\alpha_\#$ is indeed a group isomorphism.\\
    1,2,3 are completely trivial.
    For 4, we basically just want
    \[
        \begin{tikzcd}
            \pi_1(X,x_0)\arrow{r}{\alpha_\#}\arrow[swap]{d}{f_\ast}&\pi_1(X,x_1)\arrow{d}{f_\ast}\\
            \pi_1(Y,y_0)\arrow[swap]{r}{(f\circ\alpha)_\#}&\pi_1(Y,y_1)
        \end{tikzcd}
    \]
    to commute.
    To see this,
    \begin{align*}
        ((f\circ\alpha)_\#\circ f_\ast)([\gamma])&=(f\circ\alpha)_\#([f\circ\gamma])\\
        &=[(\overline{f\circ\alpha})\cdot(f\circ\gamma)\cdot (f\circ\alpha)]\\
        &=[f\circ(\bar\alpha\cdot\gamma\cdot\alpha)]\\
        &=f_\ast(\alpha_\#(\gamma))
    \end{align*}
    As we want.
\end{proof}
In particular, the fundamental group does not depend on the basepoint if the space is path-connected.
\begin{definition}
    If $X$ is a path-connected soace and $\pi_1(X,x_0)$ is trivial for some (hence every) $x_0\in X$, then we say $X$ is simply connected.
\end{definition}
\begin{lemma}\label{hom_commute_path}
    Let $x_0\in X_1$ and $f,g:X\to Y$ with $f\simeq_Fg$.
    Set $\alpha(t)=F(x_0,t)$ a path from $f(x_0)$ to $g(x_0)$.
    Then
    \[
        \begin{tikzcd}
            \pi_1(Y,f(x_0))\arrow{r}{\alpha_\#}&\pi_1(Y,g(x_0))\\
            \pi_1(X,x_0)\arrow{u}{f_\ast}\arrow[swap]{ur}{g_\ast}&
        \end{tikzcd}
    \]
    commutes.
\end{lemma}
\begin{proof}
    We need to check that for a loop $\gamma$ based at $x_0$, we have $\bar\alpha\cdot(f\circ\gamma)\cdot\alpha\simeq g\circ\gamma$.
    Consider $G:I\times I\to Y$ defined by $G(s,t)=F(\gamma(s),t)$.
    Let $a,b_1,b_2,b_3,b:I\to I^2$ be paths defined by
    $$a(t)=(t,1),b_1(t)=(0,1-t),b_2(t)=(t,0),b_3(t)=(1,t),b=b_1\cdot b_2\cdot b_3$$
    Easily $a\simeq b$.
    Then $(G\circ a)(s)=G(s,1)=F(\gamma(s),1)=(g\circ\gamma)(s)$, so $G\circ a=g\circ\gamma$.
    Calculation shows that $G\circ b_1=\bar\alpha, G\circ b_2=f\circ\gamma, G\circ b_3=\alpha$.
    Hence
    $$g\circ\gamma\simeq G\circ a\simeq G\circ b\simeq G\circ (b_1\cdot b_2\cdot b_3)\simeq (G\circ b_1)\cdot(G\circ b_2)\cdot(G\circ b_3)\simeq \bar\alpha\cdot(f\circ\gamma)\cdot\alpha$$
    As desired.
\end{proof}
\begin{theorem}\label{hom_eqv_iso}
    If $f:X\to Y$ is a homotopy equivalence, then for any $x_0\in X$,
    $$f_\ast:\pi_1(X,x_0)\to\pi_1(Y,f(x_0))$$
    is an isomorphism of groups.
\end{theorem}
\begin{proof}
    Let $g:Y\to X$ be an homotopy inverse of $f$.
    Then $\operatorname{id}_X\simeq_Fg\circ f$ and $\operatorname{id}_Y\simeq_Gf\circ g$ for some homotopy $F,G$.
    Let $\alpha(t)=F(x_0,t)$ be a path joining $\operatorname{id}_X(x_0)=x_0$ and $g(f(x_0))$, then by Lemma \ref{hom_commute_path} we have
    $$g_\ast\circ f_\ast=(g\circ f)_\ast=\alpha_\#\circ (\operatorname{id}_X)_\ast=\alpha_\#$$
    But $\alpha_\#$ is an isomorphism hence injective by Lemma \ref{indep_basepoint}, so in particular $f_\ast$ is an injection.
    By the same argument $\beta(t)=G(f(x_0),t)$ has $f_\ast\circ g_\ast=\beta_\#$ which is an isomorphism hence surjective, so $f_\ast$ is surjective.
    Therefore $f_\ast$ is bijective, hence an isomorphism.
\end{proof}
\begin{corollary}
    Contractible spaces are simply connected.
\end{corollary}
\begin{proof}
    By definition and Theorem \ref{hom_eqv_iso}.
\end{proof}
    \section{Covering Spaces}
\subsection{Definitions and Examples}
\begin{definition}
    Let $p:\hat{X}\to X$ be a map.
    An open subset $U\subset X$ is evenly covered if there exists a set $\Delta_U$ with discrete topology and a homeomorphism $\psi:p^{-1}(U)\cong\Delta_U\times U$ such that the diagram
    \[
        \begin{tikzcd}
            p^{-1}(U)\arrow{r}{\psi}\arrow[swap]{d}{p}&\Delta_U\times U\arrow{dl}{(\delta,u)\mapsto u}\\
            U&
        \end{tikzcd}
    \]
    commutes.
\end{definition}
For $\delta\in\Delta_U$, we write $U_\delta=\{\delta\}\times U\subset\Delta_U\times U$.
We can identify it as a subspace of $p^{-1}(U)$ via $\psi^{-1}$.
Then, we can canonically identify $\Delta_U$ with $p^{-1}(\{x\})$ for any $x\in U$.
Also note that we have
$$p^{-1}(U)\cong\Delta_U\times U\cong\coprod_{\delta\in\Delta_U}U$$
\begin{definition}
    If every point in $X$ has an evenly covered neighbourhood, then $p$ is a covering map and $\hat{X}$ is a covering space of $X$.
\end{definition}
\begin{example}
    1. Take $X=I, \Delta_X=\{1,2,3\}, \hat{X}=\Delta_X\times X$ and take $p:\hat{X}\to X$ to be the projection on second coordinate.\\
    But if we take $\hat{X}=I\sqcup[0,1/2)$, then the obvious projection is not a covering map as it fails at the inverse of a small neighbourhood of $1/2$.\\
    2. Let $\hat{X}=\mathbb R$ and $S^1\subset\mathbb C$ the unit circle.
    We take $p:t\mapsto e^{2\pi it}$.
    This is a covering map.
    Indeed, if $U\subsetneq S^1$ is a proper open subset and $z_1\in S^1\setminus U$, then we can choose a branch of the logarithm well-defined on $S^1\setminus\{z_1\}$.
    Write this choice of branch as $\log$.
    Now every point $\hat{z}\in p^{-1}(U)$ can be uniquely written as $\hat{z}=k+\log(z)/(2\pi i)$ for some integer $k$, which induces the homeomorphism $p^{-1}(U)\cong\mathbb Z\times U$.
    So $p$ is a covering map.\\
    3. Let $\hat{X}=X=S^1\subset C$ and $p(z)=z^n$ for some $n\in\mathbb Z_+$ is a covering map.
    Indeed, choose a local branch for $\sqrt[n]{\cdot}$ on a small enough open proper subset $U\subsetneq S^1$ allows us to write $\hat{z}\in p^{-1}(U)$, uniquely, $\hat{z}=e^{2\pi ik/n}\sqrt[n]{z}$ for $k\in\mathbb Z/n\mathbb Z$.
    So $\Delta_{S^1}=\mathbb Z/n\mathbb Z$ works.\\
    4. Let $\hat{X}=S^2$ and $G=\mathbb Z/2\mathbb Z$ acting on $S^2$ by the antipodal map $(x,y,z)\mapsto (-x,-y,-z)$.
    Take $X=\hat{X}/G$.
    We can identify $X$ with the real projective plane $\mathbb{RP}^2$.
    The quotient map $p$ is easily a covering map.
    For $x\in\hat{X}$, take a small enough neighbourhood $U$ of $x$ such that $(-U)\cap U=\varnothing$.
    Let $V=p(U)$, then $p^{-1}(V)=U\cap(-U)$, so we can just take $\Delta_U=\mathbb Z/2\mathbb Z$.
\end{example}
\begin{definition}
    A covering map $p:\hat{X}\to X$ is $n$-sheeted (where $n\in\mathbb N\cup\{\infty\}$) if $|p^{-1}(\{x\})|=n$ for every $x\in X$.
    If such an $n$ exists, we call $n$ the degree of $p$.
\end{definition}
\subsection{Lifting Properties}
We want to connect fundamental group and covering spaces.
This is done by introducing the notion of lifting.
\begin{definition}
    Let $p:\hat{X}\to X$ be a convering map and $f:Y\to X$ a map.
    A lift of $f$ to $\hat{X}$ is a map $\hat{f}:Y\to\hat{X}$ such that the diagram
    \[
        \begin{tikzcd}
            &\hat{X}\arrow{d}{p}\\
            Y\arrow[swap]{r}{f}\arrow[dashed]{ur}{\hat{f}}&X
        \end{tikzcd}
    \]
    commutes.
\end{definition}
\begin{definition}
    A space $X$ is locally path connected if for any $x\in X$ and open set $U\ni x$, there is a neighbourhood $x\ni V\subset U$ that is path-connected.
\end{definition}
\begin{lemma}[Uniqueness of Lifting]\label{lift_unique}
    Let $p:\hat{X}\to X$ be a covering map and $\hat{f}_1,\hat{f}_2:Y\to\hat{X}$ lifts of $f:Y\to X$.
    If $Y$ is connected and locally path connected, and there is $y_0\in Y$ such that $\hat{f}_1(y_0)=\hat{f}_2(y_0)$, then $\hat{f}_1=\hat{f}_2$.
\end{lemma}
However, there exists connected but not locally path-connected space, like the topologist's comb.
\begin{proof}
    We will show that the following set
    $$S=\{y\in Y|\hat{f}_1(y)=\hat{f}_2(y)\}$$
    equals $Y$ by showing it is open and closed.
    The result then follows by the connectedness of $Y$ and the fact that $y_0\in S$ (so $S$ is nonempty).\\
    Let $y_1\in Y$ be arbitrary and $U$ be an evenly covered open neighbourhood of $f(y_1)$.
    Choose $y\in V\subset f^{-1}(U)$ be a path-connected open neighbourhood of $y_1$.
    We shall show that $V\subset S$ if $y_1\in S$ and $V\subset Y\setminus S$ if $y_1\notin S$, which implies what we want.
    To see this, choose any $y\in V$, then there is a path $\alpha:I\to V$ such that $\alpha(0)=y_1$ and $\alpha(1)=y$.
    Then $\hat{f}_i\circ\alpha$ is a path in $\hat{X}$ connecting $\hat{f}_i(y_1)$ and $\hat{f}_i(y)$.
    But $p\circ\hat{f}_i\circ\alpha=f\circ\alpha$ as $\hat{f}_i$ are lifts.
    The image of $f\circ\alpha$ is contained in $f(V)\subset U$ by design.
    This tells us $\hat{f}_i\circ\alpha$ is a path in $p^{-1}(U)\cong\Delta_U\times U$.
    But $\Delta_U$ is discrete, so for each $i$, $\hat{f}_i(y_1)$ and $\hat{f}_i(y)$ must in fact lie in the same copy of $U$ as they must be in the same path component.
    We are actually done.
    Indeed, if $y_1\in S$, then $\hat{f}_1(y_1)=\hat{f}_2(y_1)$, so the copies of $U$ those paths $\hat{f}_i\circ\alpha$ lie on are actually the same.
    This forces $\hat{f}_i(y)$ to be equal since if we denote that particular copy as $U'$ then we have a homeomorphism $p':U'\to U$ by restricting the covering map which gives
    $$\hat{f}_1(y)=(p')^{-1}\circ f(y)=\hat{f}_2(y)$$
    If $y_1\notin S$ but $y\in S$, then by reversing the argument for $y,y_1$ shows $y_1\in S$ which is false.
    (Alternatively, one can argue by observing that each copy of $U$ contains a unique point of $p^{-1}\circ f(\{y_1\})$).
\end{proof}
\begin{definition}
    Let $\gamma:I\to X$ be a path from $x_0$ and $p:\hat{X}\to X$ be a covering map.
    A lift of $\gamma$ at $\hat{x}_0\in\hat{X}$ is a lift $\hat{\gamma}:I\to\hat{X}$ of $\gamma$ such that $\hat{\gamma}(0)=\hat{x}_0$.
\end{definition}
In particular, $p(\hat{x}_0)=x_0$ necessarily.
In the special case where $\gamma$ is contained in an evenly covered open set, its lift is just picking one of the copies of the open set containing $\hat{x}_0$ in the pre-image under the covering map and carve the same path there.
In fact, we can say more.
\begin{lemma}[Path-Lifting Lemma]\label{path_lift}
    Let $p:\hat{X}\to X$ be a covering map and $\gamma:I\to X$ be a path from $x_0$.
    For any $\hat{x}_0\in p^{-1}(x_0)$, there exists a unique lift $\hat{\gamma}$ of $\gamma$ from $\hat{x}_0$.
\end{lemma}
\begin{proof}
    The uniqueness follows from Lemma \ref{lift_unique}.
    For existence, we consider
    $$S=\{t\in I:\gamma|_{[0,t]}\text{ has a lift at $\hat{x}_0$ to $\hat{X}$}\}$$
    and we will show it is open and closed.
    As $0\in S$, we know $S\neq\varnothing$, therefore it shall imply what we want.\\
    Let $t_0\in I$ and $U$ be an evenly covered neighbourhood of $\gamma(t_0)$ and let $V\subset\gamma^{-1}(U)$ be an open interval containing $t_0$.
    Again we will show $t_0\in S$ implies $V\subset S$ and $t_0\notin S$ implies $V\subset I\setminus S$ which gives the result.
    Let $t\in V$ and suppose first that $t_0\in S$.
    If $t\le t_0$ then automatically $t\in S$.
    Otherwise $t>t_0$.
    Denote the lift of $\gamma|_{[0,t_0]}$ by $\hat{\gamma}|_{[0,t_0]}$.
    But then since $U\supset f(V)$ is evenly covered, we can just pick the copy of $U$ where $\hat{\gamma}|_{[0,t_0]}(t_0)$ resides in and continue $\hat{\gamma}$ there by following $\gamma$.
    More precisely, suppose that copy is $U'$ and $p':U'\to U$ is the homeomoephism by restricting $p$, then
    $$[0,t]\ni s\mapsto\begin{cases}
        \hat\gamma(s)\text{, for $s\in[0,t_0]$}\\
        (p')^{-1}\circ\gamma(s)\text{, for $s\in[t_0,t]$}
    \end{cases}$$
    which works as a lift of $\gamma|_{[0,t]}$.
    So $t\in S$.\\
    If $t_0\notin S$ and $t\in V$, then if $t\in S$ and $t\ge t_0$ we get an immediate contradiction.
    But if $t<t_0$ and $t\in S$, we can extend the lift on $[0,t]$ in the way we just described to $t_0$ which shows $t_0\in S$, another contradiction.
    So we are done.
\end{proof}
\begin{lemma}
    If $p:\hat{X}\to X$ is a covering map and $X$ is path-connected, then $p$ is an $n$-sheeted cover for some $n\in\mathbb N\cup\{\infty\}$.
\end{lemma}
Actually we can do something stronger:
We can prove that if $x,y\in X$ then there is a bijection between $p^{-1}(\{x\})$ and $p^{-1}(\{y\})$.
In fact, this is exactly what we shall prove.
\footnote{I think this lemma is trivial by considering the set of points $x$ such that $p^{-1}(x)$ has a given (fixed) cardinality which can be easily shown to be both open and closed, but the lecturer prefers to use path-lifting lemma.}
\begin{proof}
    Let $\gamma$ be a path in $X$ from $x$ to $y$.
    For any $\hat{x}\in p^{-1}(\{x\})$, then there is a unique path $\hat{\gamma}_{\hat{x}}$ lifting $\gamma$ with starting point $\hat{x}$.
    This gives a map $\psi:p^{-1}(\{y\})\to p^{-1}(\{y\})$ via $\psi(\hat{x})=\hat{\gamma}_{\hat{x}}(1)$.
    We shall show that it has an inverse $\phi$ defined in a similar way but using the path $\bar\gamma$ from $y$ to $x$.
    So $\phi(\hat{y})=\widehat{(\bar\gamma)}_{\hat{y}}(1)$.
    We shall show that $\phi\circ\psi=\operatorname{id}_{p^{-1}(\{x\})}$, the other side is completely analogous.
    Indeed,
    $$\phi\circ\psi(\hat{x})=\phi(\hat{\gamma}_{\hat{x}}(1))=\widehat{(\bar\gamma)}_{\hat{\gamma}_{\hat{x}}(1)}(1)$$
    But then $\hat{\gamma}_{\hat{x}}\cdot\widehat{(\bar\gamma)}_{\hat{\gamma}_{\hat{x}}(1)}$ is a lift of $\gamma\cdot\bar\gamma$, but so is $\hat{\gamma}_{\hat{x}}\cdot\overline{(\hat{\gamma}_{\hat{x}})}$ hence by the uniqueness of lifts they have common endpoints.
    In particular, $\widehat{(\bar\gamma)}_{\hat{\gamma}_{\hat{x}}(1)}(1)=\hat{x}$ as desired.
\end{proof}
Here comes a lemma that really links together fundamental groups and covering maps
\begin{lemma}[Homotopy Lifting Lemma]\label{homotopy_lifting}
    Let $p:\hat{X}\to X$ be a covering map and $f_0:Y\to X$ a map from a locally path-connected space.
    Suppose $F:Y\times I\to X$ is a homotopy with $\forall y\in Y,F(y,0)=f_0(y)$ and there exists a lifting $\hat{f}_0:Y\to\hat{X}$ of $f_0$.
    Then there exists a unique lifting $\hat{F}:Y\times I\to\hat{X}$ with $\hat{F}(y,0)=\hat{f}_0(y)$ for any $y\in Y$.
\end{lemma}
\begin{remark}
    In the special case where $Y$ is a one-point space we reproduce Lemma \ref{path_lift}.
\end{remark}
\begin{proof}
    For each $y\in Y$, the homotopu $F$ defines a path $\gamma_y(t)=F_{y,t}$.
    By Lemma \ref{path_lift}, each $\gamma_t$ has a unique lift $\hat{\gamma}_t$ from $\hat{f}_0(y)$.
    So we essentially need $\hat{F}(y,t)=\hat{\gamma}_t(y)$.
    It remains to show that such an $\hat{F}$ has to be continuous.
    The trick is to construct on open subsets of $Y\times I$ a differently constructed lift $\tilde{F}$ which is a priori continuous, and then show that $\hat{F}$ agrees with $\tilde{F}$ on these open sets.
    Fix $y_0\in Y$.
    The goal is to find a neighbourhood $V$ of $y_0$ and a lifting of $F$ on $V\times I$.
    For any $t\in I,F(y_0,t)\in X$ has an evenly covered neighbourhood $U_t\subset X$.
    Then by continuity and the definition of product topology $F^{-1}(U_t)$ contains an open neighbourhood of $(y_0,t)\in Y\times I$ of the form $V_t\times[(t-\epsilon_t,t+\epsilon_t)\cap I]$ for $\epsilon_t>0$ and $V_t\ni y_0$ is open.
    As $Y$ is locally path-connected we might as well assume $V_t$ are path-connected.
    As $\{y_0\}\times I$ is compact, there is a finite set $T\subset I$ such that $\{(t_i-\epsilon_{t_i},t_i+\epsilon_{t_i}):t_i\in T\}$ covers $I$.
    We then take $V=\bigcap_{t_i\in T}V_{t_i}$ which is open and path-connected.
    Take $J_i=(t_i-\epsilon_{t_i},t_i+\epsilon_{t_i})\cap I$.
    Then we know that $F(V\times J_i)$ is contained in an evenly covered open subset $U_i$ of $X$ for each $i$.
    Let $U_i'$ be the unique copy of $U_i$ in $p^{-1}(U_i)$ such that $\hat{F}(\{y_0\}\times J_i)=\hat{\gamma}_{y_0}(J_i)\subset U_i'$.
    Let $p_i:U_i'\to U_i$ be the homeomorphism between them.
    Now for $(y,t)\in V\times I$, we define $\tilde{F}(y,t)=p_i^{-1}\circ F(y,t)$ for $t\in J_i$.
    It is quite obvious it is well-defined, but let's prove it.
    Suppose $t\in J_i\cap J_j$ and let $\alpha$ be a path in $V$ from $y_0$ to $y$.
    Define $\alpha_t(s)=F(\alpha(s),t)$.
    Then $p_i^{-1}\circ \alpha_t$ and $p_j^{-1}\circ\alpha_t$ are lifts of $\alpha_t$.
    Furthermore they have the same initial values at $\hat{F}(y_0,t)$, so they are equal by Lemma \ref{lift_unique}.
    Consequently $p_i^{-1}\circ F(y,t)=p_j^{-1}\circ F(y,t)$ as desired.
    So $\tilde{F}$ is a well-defined and hence continuous by its definition.\\
    As $V$ is path-connected and $\tilde{F}(\cdot,0)$ is a lift of $f_0$ that agrees with $\hat{f}_0$ at $y_0$, we have $\forall y\in V,\hat{F}(y,0)=\hat{f}_0(y)$ by Lemma \ref{lift_unique}.
    Also, for each $y\in V$, $\hat{F}(y,\cdot)$ is a lift of $\gamma_y$ at $\hat{f}_0(y)$.
    Therefore $\tilde{F}(y,t)=\hat{\gamma}_y(t)$ again by Lemma \ref{lift_unique}, thus $\tilde{F}=\hat{F}$ on $V\times I$.
    In particular, $\hat{F}$ is continuous in $V\times I$, hence it is continuous.
\end{proof}
\subsection{Lifting and Fundamental Groups}
\begin{lemma}
    Let $p:\hat{X}\to X$ be a covering map and $F:I\times I\to X$ be a homotopy of paths.
    Then any lift $\hat{F}$ of $F$ is also a homotopy of paths.
\end{lemma}
\begin{proof}
    As $F$ is a homotopy of paths, $F(0,\cdot)$ and $F(1,\cdot)$ are constant paths in $X$.
    Therefore $\hat{F}(0,\cdot)$ and $\hat{F}(1,\cdot)$ are lifts those constant paths , hence constant.
\end{proof}
\begin{lemma}
    Let $p:\hat{X}\to X$ be a covering map with $\hat{x}\in \hat{X}$ and $x=p(\hat{x})$, then the induced homeomorphism $p_\ast:\pi_1(\hat{X},\hat{x})\to\pi_1(X,x)$ via $[\hat\gamma]\mapsto [p\circ\hat\gamma]$ is injective.
\end{lemma}
\begin{proof}
    Suppose $[\hat{\gamma}]\in\ker p_\ast$, then $\gamma=p\circ\hat{\gamma}\simeq c_x$.
    If $F$ is a homotopy between $\gamma$ and $c_x$, we can lift it to a homotopy $\tilde{F}$ from $\hat{\gamma}$ which is a homotopy of paths by the preceding lemma.
    In particular, $\hat{F}$ is a homotopy between $\hat\gamma$ and $c_{\hat{x}}$, so $[\hat\gamma]=[c_{\hat{x}}]$ is the identity.
    This shows the result.
\end{proof}
\begin{remark}
    So we can view $\pi_1(\hat{X},\hat{x})$ as the subgroup $p_\ast(\pi(\hat{X},\hat{x}))\le\pi_1(X,x)$.
    Also note that given $[\gamma]\in\pi_1(X,x)$, we get a map $p^{-1}(\{x\})\to p^{-1}(\{x\})$ via $\hat{x}\mapsto \hat{\gamma}_{\hat{x}}(1)$ which is necessarily bijective.
    So this defines a right group action of $\pi_1(X,x)$ on $p^{-1}(\{x\})$.
    For $\hat{x}\in p^{-1}(\{x\})$, this is induced by $\hat{x}\cdot\gamma=\hat\gamma_{\hat{x}}(1)$.
    Easily $(\hat{x}\cdot\gamma)\cdot\delta=\hat{x}\cdot(\gamma\cdot\delta)$.
\end{remark}
\begin{lemma}
    Let $p:\hat{X}\to X$ be a covering map and suppose $\hat{X}$ is path-connected.
    Let $x\in X$, and let $\Pi$ denote the set of right cosets of $p_\ast(\pi_1(\hat{X},\hat{x}))$ in $\pi_1(X,x)$.
    Then the map $\Pi\to p^{-1}(\{x\})$ via $p_\ast(\pi_1(\hat{X},\hat{x}))[\gamma]\mapsto \hat{x}\cdot\gamma$ is a bijection.
    Further, this bijection satisfies
    $$p_\ast(\pi_1(\hat{X},\hat{x}))[\gamma][\delta]\mapsto\hat{x}\cdot(\gamma\cdot\delta)$$
    where $\gamma,\delta$ are loops based at $x$.
\end{lemma}
\begin{proof}
    If $[\delta]\in p_\ast(\pi_1(\hat{X},\hat{x}))$, then $\hat{x}\cdot\delta=\hat{x}$.
    This obviously implies the map is well-defined.
    To prove this is a bijection, we shall use the Orbit-Stabiliser Theorem.
    Note that the stabliser of $\hat{x}$ on this right action is the set of loops $[\gamma]$ at $x$ such that $\hat\gamma_{\hat{x}}(1)=\hat{x}$ where $\hat{\gamma}_{\hat{x}}$ is the lift of $\gamma$ from $\hat{x}$.
    But then $\hat\gamma_{\hat{x}}$ is a loop, so $\gamma\in p_\ast(\pi_1(\hat{X},\hat{x}))$.
    This means that the stabiliser of $\hat{x}$ is exactly $p_\ast(\pi_1(\hat{X},\hat{x}))$.
    It remains to show that the action is transitive.
    Let $\hat{y}\in p^{-1}(\{x\})$, then there is a path $\hat{\gamma}$ from $\hat{x}$ to $\hat{y}$, so $p\circ\hat{\gamma}$ is a loop based at $x$ and $\hat{x}\cdot\gamma=\hat{y}$.
\end{proof}
\begin{remark}
    So the degree of the covering map $p$ is just the index of $p_\ast(\pi_1(\hat{x},\hat{x}))$ in $\pi_1(X,x)$.
\end{remark}
\begin{example}
    Consider $\pi_1(S^1,x)$.
    The covering map $p_1:\mathbb R\to S^1$ via $t\mapsto e^{2\pi it}$ has infinite degree and $p_2:S^1\to S^1$ via $z\mapsto z^n$ has degree $n$ for any $n>1$.
    So what we conclude from the lemma above is that $\pi_1(S_1,x)$ is an infinite group with subgroups of every finite index.
    A natural guess is then $\pi_1(S^1,x)\cong\mathbb Z$.
    We will prove it later.
\end{example}
\begin{definition}
    If $p:\hat{X}\to X$ is a covering map and $\hat{X}$ is simply connected, then $\hat{X}$ is called a univeral cover of $X$.
\end{definition}
\begin{example}
    The map $t\mapsto e^{2\pi it}$ makes $\mathbb R$ a universal cover of $S^1$.
\end{example}
\begin{corollary}
    If $p:\hat{X}\to X$ is a universal cover, then for any choice of $\hat{x}\in p^{-1}(\{x\})$ there is a bijection
    $$\pi_1(X,x)\to p^{-1}(\{x\}),[\gamma]\mapsto \hat{x}\cdot\gamma$$
\end{corollary}
\begin{proof}
    Immediate from what we have discussed.
\end{proof}
So we can get a group structure on $p^{-1}(\{x\})$ which is essentially isomorphic to $\pi_1(X,x)$ and determined by $\hat{x}\cdot (\gamma\cdot\delta)=(\hat{x}\cdot\gamma)\cdot\delta$.
This allows us to actually calculate some fundamental groups.
\begin{example}
    Now we compute the fundamental group of the circle.
    Let $p:\mathbb R\to S^1$ be the usual universal cover.
    Hence there is a bijection
    $$\pi_1(S^1,1)\to p^{-1}(\{1\})=\mathbb Z\subset\mathbb R$$
    Now for $n\in\mathbb Z$, take $\tilde{\gamma}_n(t)=nt$ to be a path from $0$ to $n$ in $\mathbb R$ and let $\gamma_n=p\circ\hat{\gamma}_n$ which has to be a loop based at $1$.
    Clearly $0\cdot\gamma_n=\hat{\gamma}_n(1)=n$.
    Thus any loop in $S^1$ based at $1$ is homotopic to some $\gamma_n$.
    So the bijection becomes $[\gamma_n]\mapsto n$.
    It remains to show that it is a homomorphism.
    Indeed, for any $m\in\mathbb Z$, we know that $m+\hat{\gamma}_n$ is a lift of $\gamma_n$ from $m$ to $m+n$.
    So
    $$(0\cdot\gamma_m)\cdot\gamma_n=m\cdot\gamma_n=m+n=0\cdot\gamma_{m+n}$$
    Hence it is indeed a homomorphism, it then follows that it is an isomorphism, which means $\pi_1(S^1,1)\cong\mathbb Z$.
\end{example}
In complex analysis, we defined the winding number as, loosely speaking, the number of times a loop wraps around a certain point.
It then follows that this notion is essentially describing the homotopy class a curve is in when we put the thing in $\pi_1(S^1,1)$.
There are more applications of this idea.
\subsection{Applications of Fundamental Groups}
\begin{theorem}[No-Retraction Theorem]\label{no_retraction}
    The identity map of $S^1$ does not extend to a map $r:D^2\to S^1$ (where we view $S^1$ as $\partial D^2$).
\end{theorem}
That is, $S^1$ is not a retract of $D^2$.
We have essentially covered the proof in the introduction, where all technical details have just been covered.
\begin{proof}
    If such an $r$ exists, let $\iota:S^1\hookrightarrow D^2$ be the inclusion map, then $r\circ\iota=\operatorname{id}_{S^1}$.
    As $D^2$ is contractible, $\pi_1(D^2,1)=0$, so there is a factorisation
    $$(\operatorname{id}_{S^1})_\ast=(r\circ\iota)_\ast=r_\ast\circ\iota_\ast:\pi_1(S^1,1)\to\pi_1(D^2,1)\to\pi_1(S^1,1)$$
    But the last diagram is just $\mathbb Z\to\{0\}\to\mathbb Z$.
    This means that $(\operatorname{id}_{S^1})_\ast$ has to be constant.
    But it isn't.
    Contradiction.
\end{proof}
\begin{theorem}[Brouwer's Fixed Point Theorem]
    Any map $f:D^2\to D^2$ has a fixed point.
\end{theorem}
\begin{proof}
    Suppose not, then let $g:D^2\to S^1$ be the map given by projecting $f(x)$ through $x$ onto $S^1$.
    That is, $g(x)$ is the intersection of the line joining $x$ and $f(x)$ and $S^1$ that is closer to $x$.
    Then $g$ is continuous and $g(x)=x$ for any $x\in S^1$.
    This however means that $g$ restricts to $\operatorname{id}_{S^1}$, contradicing Theorem \ref{no_retraction}.
\end{proof}
\begin{theorem}[The Fundamental Theorem of Algebra]
    Every nonconstant polynomial $p:\mathbb C\to\mathbb C$ has a zero.
\end{theorem}
\begin{proof}
    Let $r:\mathbb C\setminus\{0\}\to S^1$ via $r(z)=z/|z|$ which is a retraction.
    For $R\ge 0$, we define $\lambda_R:S^1\to\mathbb C, \lambda_R(z)=Rz$.
    If $p$ has no zero, then we can define $f_R:r\circ p\circ \lambda_R:S^1\to S^1$.
    Easily for any $R_1,R_2\in\mathbb R_{\ge 0}$, $f_{R_1}$ and $f_{R_2}$ are homotopic, hence $(f_{R_1})_\ast=(f_{R_2})_\ast$.
    But these induced maps are all homomorphisms $\mathbb Z\to\mathbb Z$.
    Therefore all $(f_R)^\ast$ is given by multiplication by $d$ for some fixed $d\in\mathbb N\setminus\{0\}$.
    But $f_0$ is constant, therefore $d=0$ and $(f_R)_\ast$ are constantly zero.
    But for very large $R$, the leading term dominates $p$, so it is clear that $(f_R)_\ast$ is given by multiplication by $\deg p\neq 0$, contradiction.
\end{proof}
\begin{definition}
    A space $X$ is locally simply connected if, for every $x\in X$ and open neighbourhood $U$ of $x$, there exists a simply connected neighbourhood $x\in V\subset U$.
\end{definition}
\begin{example}[Non-example]
    Take the Hawaii Earing space
    $$X=\bigcup_{n=1}^\infty \left\{(x,y)\in\mathbb R^2:\left( x-\frac{1}{n} \right)+y^2=\frac{1}{n^2}\right\}$$
    then no open neighbourhood of $(0,0)$ is simply connected.
\end{example}
\begin{theorem}[Existence of Universal Cover]
    Let $X$ be a path-connected space such that $X$ is localled simply connected, then there exists a universal cover $p:\hat{X}\to X$.
\end{theorem}
\begin{proof}[Sketch of proof]
    Fix $x_0\in X$ and consider the set $\mathscr X$ of all paths $\gamma$ from $x_0$.
    Define $\hat{X}=\mathscr X/\simeq$ where $\simeq$ is the path homotopy equivalence relation.
    The intended map $p:\hat{X}\to X$ is gives by $p([\gamma])=\gamma(1)$.
    The tricky bit is to find a topology on $\hat{X}$ making it work, which -- guess what -- is skipped.
\end{proof}
\begin{example}
    Take $X$ to be the figure eight, then $\hat{X}$ looks like the Caylay graph of the free group $F_2$.
\end{example}
\subsection{The Galois Correspondence}
The idea is to classify all covering spaces using subgroups of the fundamental group, which is in certain ways analogous to the idea of Galois correspondence in Galois theory.
\begin{definition}
    Let $X$ be a path-connected space and $p_1:\hat{X}_1\to X_1, p_2:\hat{X}_2\to X$.
    An isomorphism of covering spaces is a homeomorphism $\phi:\hat{X}_1\to\hat{X}_2$ such that
    \[
        \begin{tikzcd}
            X&\hat{X}_1\arrow{dl}{\phi}\arrow[swap]{l}{p_1}\\
            \hat{X}_2\arrow{u}{p_2}&
        \end{tikzcd}
    \]
    commutes.
\end{definition}
Note also that $\phi^{-1}$ is automatically an isomorphism of covering spaces as well
If $\hat{X}_i$ are equipped with basepoints $\hat{x}_i\in\hat{X}_i$ and $\phi(\hat{x}_1)=\hat{x}_2$, we say $\phi$ is based.
\begin{remark}
    Note that $\phi$ is a lift of $p_1$ to $\hat{X}_2$, so by Lemma \ref{lift_unique}, a based isomorphism is uniquely determined by the basepoints $\phi(\hat{x}_1)=\hat{x}_2$ if $\hat{X}_1$ is connected and locally path-connected.
\end{remark}
\begin{theorem}[Galois Correspondence with Basepoints]\label{based_galois}
    Let $X$ be a path-connected, locally simply connected space with basepoint $x_0$.
    The map which sends a covering $p:\hat{X}\to X$ equipped with a basepoint $\hat{x}_0\in p^{-1}(\{x_0\})$ to the subgroup $p_\ast(\pi_1(\hat{X},\hat{x}_0))\le\pi_1(X,x_0)$ induces a bijection between the set of based isomorphism classes of path-connected covering spaces with basepoint and the set of subgroups of $\pi_1(X,x_0)$.
\end{theorem}
\begin{proof}
    Not dreadfully hard but omitted.
\end{proof}
\begin{example}
    As $\pi_1(S^1,1)=\mathbb Z$, each subgroup of $\mathbb Z$ is of the form $n\mathbb Z$ for natural number $n$ (including $0$).
    Then the usual $p:\mathbb R\to S^1, t\mapsto e^{2\pi it}$ corresponds to the subgroup $\{0\}$, and the maps $p:S^1\to S^1, z\mapsto z^n$ corresponds to the subgroups $n\mathbb Z$ for $n\neq 0$.
    Hence, these are the only path-connected covering spaces of $S^1$ up to based isomorphism.
\end{example}
\begin{corollary}
    Let $X$ be a path-connected and locally simply connected space, then any two universal covers $p_1:\hat{X}_1\to X$ and $p_2:\hat{X}_2\to X$ are isomorphic.
\end{corollary}
\begin{proof}
    Immediate.
\end{proof}
\begin{corollary}[Galois Correspondence without Basepoints]\label{unbased_galois}
    Let $X$ be a path-connected, locally simply connected space with basepoint $x_0$.
    Then the map that sends a covering $p:\hat{X}\to X$ equipped with a basepoint $\hat{x}_0\in p^{-1}(\{x_0\})$ to the subgroup $p_\ast(\pi_1(\hat{X},\hat{x}))\le\pi_1(X,x_0)$ induces a bijection between (unbased) isomorphism classes of path-connected convering spaces of $X$ and conjugacy classes of subgroups of $\pi_1(X,x_0)$.
\end{corollary}
\begin{proof}
    This map is surjective due to Theorem \ref{based_galois}.
    To see it is injective, we need to show that if $(p_1)_\ast(\pi_1(\hat{X}_1,\hat{x}_1))$ and $(p_2)_\ast(\pi_1(\hat{X}_2,\hat{x}_2))$ are conjugate subgroups of $\pi_1(X,x_0)$, then there is an (unbased) isomorphism $\phi:\hat{X}_1\to\hat{X}_2$ of covering spaces.\\
    Suppose
    $$(p_1)_\ast(\pi_1(\hat{X}_1,\hat{x}_1))=[\gamma](p_2)_\ast(\pi_1(\hat{X}_2,\hat{x}_2))[\bar\gamma]$$
    for some $[\gamma]\in\pi_1(X,x_0)$.
    Let $\widehat{\bar{\gamma}}$ be the lift of $\bar\gamma$ at $\hat{x}_2$ and $\hat{x}_2'$ be the other endpoint of $\widehat{\bar\gamma}$.
    By the last part of Lemma \ref{indep_basepoint},
    \begin{align*}
        [\gamma](p_2)_\ast(\pi_1(\hat{X}_2,\hat{x}_2))[\bar\gamma]&=\bar\gamma_\#((p_2)_\ast(\pi_1(\hat{X}_2,\hat{x}_2)))\\
        &=(p_2)_\ast(\widehat{\bar\gamma}_\#(\pi_1(\hat{X}_2,\hat{x}_2)))\\
        &=(p_2)_\ast(\pi_1(\hat{X}_2,\hat{x}_2'))
    \end{align*}
    Therefore $(p_1)_\ast(\pi_1(\hat{X}_1,\hat{x}_1))=(p_2)_\ast(\pi_1(\hat{X}_2,\hat{x}_2'))$.
    Theorem \ref{based_galois} then gives a based isomorphism $(\hat{X}_1,\hat{x}_1)\cong(\hat{X}_2,\hat{x}_2')$.
    In particular, $\hat{X}_1\cong\hat{X}_2$ as (unbased) covering spaces.
\end{proof}
    \section{The Seifert-van Kampen Theorem}
\subsection{Some Group Theory}
Recall that we can sometimes represent groups using generators and relations, e.g. the dihedral group $D_{2n}$ can be written as $\langle r,s|s^2=r^n=srsr=1\rangle$ meaning it is the group generated by $r,s$ subject to the given relations $s^2=r^n=1,srs=r^{-1}$.
\begin{definition}
    Let $A$ be a set.
    The free group on $A$ is a group $F(A)$ such that there is a function $\phi:A\to F(A)$ satisfying the following universal property:\\
    Whenever there is a group $G$ and a function $h:A\to G$, there exists a unique homomorphism $f:F(A)\to G$ such that
    \[
        \begin{tikzcd}
            F(A)\arrow[dashed]{dr}{f}&\\
            A\arrow{u}{\phi} \arrow[swap]{r}{h}&G
        \end{tikzcd}
    \]
    commutes.
\end{definition}
\begin{example}
    Let $A=\{\alpha\}$ and $\phi:A\to\mathbb Z$ be the map $\alpha\mapsto 1$.
    Then for any $h:A\to G$ mapping $\alpha$ to $g\in G$, we necessarily have $f:\mathbb Z\to G$ via $n\mapsto g^n$ such that $f\circ\phi=h$.
    Hence $\mathbb Z$ is the free group on $\{a\}$.
\end{example}
\begin{remark}
    This definition, as a universal property, does not give an explicit construction of $F(A)$ but captured all of its group theorectical properties.
    As such, if such an $F(A)$ exists it must be unique up to group isomorphism -- that's why we used ``the'' free group instead of ``a'' free group.
\end{remark}
\begin{proposition}
    Let $A$ be a set and $F(A),F'(A)$ be both free groups on $A$, then $F(A)\cong F'(A)$ as groups.
\end{proposition}
\begin{proof}
    Suppose $\phi:A\to F(A),\phi':A\to F'(A)$ be the maps described in the definition.
    Take $G=F'(A)$ gives a unique homomorphism $f:F(A)\to F'(A)$ such that $f\circ\phi=\phi'$ since $F(A)$ s a free group on $A$; $G=F(A)$ also gives a unique homomorphism $g:F'(A)\to F(A)$ such that $g\circ\phi'=\phi$.
    We claim that $f,g$ are inverse to each other.
    Now $f\circ g$ makes the diagram
    \[
        \begin{tikzcd}
            F'(A)\arrow{dr}{f\circ g}&\\
            A\arrow{u}{\phi'}\arrow[swap]{r}{\phi'}&F'(A)
        \end{tikzcd}
    \]
    commute.
    But then since $F'(A)$ is free, $f\circ g$ must equal to $\operatorname{id}_{F'(A)}$ (which also makes the diagram commute when putting it in the place of $f\circ g$) by uniqueness.
    Similarly $g\circ f=\operatorname{id}_{F(A)}$.
    Hence $f,g$ are indeed inverses of each other.
    This shows $F(A)\cong F'(A)$.
\end{proof}
The proof also shows that the isomorphism is uniquely determined if we ask it to be compatible with $\phi,\phi'$.
In other word, it is canonical with respect to this universal property.
\begin{definition}
    If $A$ is a finite set of cardinality $r$, then we write $F_r=F(A)$ to be the free group of rank $r$.
\end{definition}
\begin{definition}
    The words in $A$ are the strings composed of elements in $A$ and their (symbolic) inverses.
\end{definition}
For example, $w=aba^{-1}bba^{-1}a^{-1}b$ is a word in $A$.
We can canonically identify a word in $A$ by an element of $F(A)$ by mapping it to the product (with the same order of elements) of the respective images, e.g. $abba^{-1}b$ can be identified with $\phi(a)\phi(b)^2\phi(a)^{-1}\phi(b)\in F(A)$.\\
It is easy to check that $\phi(A)$ also satisfies the same universal property of a free group on $A$, therefore by uniqueness we can assume $F(A)$ is generated by $\phi(A)$.
So $F(A)$ can be seen as the set of words in $A$, possibly with some identifications (e.g. $aa^{-1}b=b$).
As a recreational exercise, one can precisely construct a free group on any given set in this way.
After the construction, we can happily identify $A$ as a generating subset of $F(A)$.
\begin{definition}
    A presentation is a set $A$ and a subset of relations $R\subset F(A)$ which identifies the group
    $$\langle A|R\rangle=F(A)/\langle\langle R\rangle\rangle$$
    where $\langle\langle R\rangle\rangle$ is the smallest normal subgroup of $F(A)$ containing $R$ (i.e. intersection of all normal subgroups containing $R$).
    The presentation is finite if both $A,R$ are.
\end{definition}
Easily $\langle\langle R\rangle\rangle=\langle\{srs^{-1}:s\in F(A),r\in R\}\rangle$.
\begin{lemma}[Universal Property of Presentation]
    Let $q:F(A)\to\langle A|R\rangle$ be the quotient map.
    Whenever $f:F(A)\to G$ is a group homomorphism such that $R\subset\ker f$, there is a unique homomorphism $g:\langle A|R\rangle\to G$ such that
    \[
        \begin{tikzcd}
            \langle A|R\rangle\arrow[dashed]{dr}{g}&\\
            F(A)\arrow{u}{q}\arrow[swap]{r}{f}&G
        \end{tikzcd}
    \]
    commutes.
\end{lemma}
\begin{proof}
    Necessarily $g(w\langle\langle R\rangle\rangle)=f(w)$ which works since $\langle\langle R\rangle\rangle\le\ker f$ by definition.
\end{proof}
\begin{example}
    1. We know that $F(\{a\})\cong\mathbb Z$ and every subgroup of $\mathbb Z$ is normal, so $\langle\langle a^n\rangle\rangle=\langle a^n\rangle$ corresponds to $n\mathbb Z$, so $\langle a|a^n\rangle\cong\mathbb Z/n\mathbb Z$.\\
    2. We claim that $G=\langle r,s|r^n,s^2,rsrs\rangle$ is isomorphic to $D_{2n}$.
    Indeed, the homomorphism $F(\{r,s\})\to D_{2n}$ sending $r$ to a rotation of $D_{2n}$ and $s$ to a reflection takes $R=\{r^n,s^2,rsrs\}$ to the identity.
    Thus it factors through $G$ via $\phi:G\to D_{2n}$ by the universal property.
    Now $\phi$ is obviously surjective and injective since we can write $G=\{1,r,\ldots,r^{n-1},s,sr,\ldots,sr^{n-1}\}$ which has the correct size.\\
    3. Every group has a presentation.
    The identity map from $G$ as a set to $G$ as a group induces a group homomorphism $F(G)\to G$ which is surjective.
    Let $R$ be the kernel of this map, then $G=\langle G|R\rangle$.
    (This is, however, not very useful since this is a very inefficient choice of generators and relations.)
\end{example}
\begin{definition}[Pushouts]
    Consider a commutative square
    \[
        \begin{tikzcd}
            \Gamma&A\arrow[swap]{l}{k}\\
            B\arrow{u}{l}&C\arrow{l}{j}\arrow[swap]{u}{i}
        \end{tikzcd}
    \]
    It is called a pushout if it satisfies the following universal property:
    If $G$ is a group with homomorphism $f:A\to G,g:B\to G$ such that
    \[
        \begin{tikzcd}
            G&A\arrow[swap]{l}{f}\\
            B\arrow{u}{g}&C\arrow{l}{j}\arrow[swap]{u}{i}
        \end{tikzcd}
    \]
    commutes, then there is a unique $\phi:\Gamma\to G$ such that
    \[
        \begin{tikzcd}
            G&&\\
            &\Gamma\arrow[swap,dashed]{ul}{\exists!\phi}&A\arrow[swap]{l}{k}\arrow[swap, bend right]{ull}{f}\\
            &B\arrow{u}{l}\arrow[bend left]{uul}{g}&C\arrow{l}{j}\arrow[swap]{u}{i}
        \end{tikzcd}
    \]
    commutes.\\
    If this is indeed the case, then we write $\Gamma=A\sqcup_CB$ (the arrows $i,j$ are equipped with $A,B$ and $k,l$ are equipped with $\Gamma$).
\end{definition}
Given $A,B,C,i,j$, one can check that such a $\Gamma$ is unique up to isomorphism.
\begin{definition}
    If $C=\{e\}$ and $i,j$ be the unique homomorphisms from $C$ to $A,B$, then $A\sqcup_CB$ is called the free product, denotes by $A\star B$.\\
    If $i,j$ are injective, then $A\coprod_CB$ is called the amalgamated product and is written as $A\star_CB$.
\end{definition}
\begin{lemma}\label{pushout_onetrivial}
    For $i:C\to A,j:C\to B=\{e\}$, we have $A\sqcup_C\{e\}\cong A/\langle\langle i(C)\rangle\rangle$.
\end{lemma}
\begin{proof}
    Take $q:A\to A/\langle\langle i(C)\rangle\rangle$ to be the quotient map and $\iota:\{e\}\to A/\langle\langle i(C)\rangle\rangle$ the natural inclusion, then the diagram
    \[
        \begin{tikzcd}
            A/\langle\langle i(C)\rangle\rangle&A\arrow[swap]{l}{q}\\
            \{e\}\arrow{u}{\iota}&C\arrow{l}{j}\arrow[swap]{u}{i}
        \end{tikzcd}
    \]
    does commute.
    Now suppose that there are $f,g$ such that $j\circ g=i\circ f$, i.e. the bigger diagram
    \[
        \begin{tikzcd}
            G&&\\
            &\Gamma\arrow[swap,dashed]{ul}{?}&A\arrow[swap]{l}{q}\arrow[swap, bend right]{ull}{f}\\
            &\{e\}\arrow{u}{\iota}\arrow[bend left]{uul}{g}&C\arrow{l}{j}\arrow[swap]{u}{i}
        \end{tikzcd}
    \]
    commutes.
    Then by commutativity, $f\circ i$ is the constant homomorphism since $g$ (hence $j\circ g$) has to be, so $f(i(C))=\{e\}\in G$, therefore $i(C)\subset\ker f$.
    As $f$ is normal, necessarily $\langle\langle i(C)\rangle\rangle\subset\ker f$, so we necessarily have to choose
    $$\phi:A/\langle\langle i(C)\rangle\rangle\to G,w\langle\langle i(C)\rangle\rangle\mapsto f(w)$$
    which works.
\end{proof}
Do pushouts always exist?
\begin{lemma}
    Let $A=\langle S_1|R_1\rangle$ and $B=\langle S_2|R_2\rangle$ and let $T\subset C$ be a generating set for $C$.
    Suppose $\tilde\imath:T\to F(S_1)$ is a lift of a function $i|_T$ and $\tilde\jmath:T\to F(S_2)$ is a lift of $j|_T$ (so $q_1\circ\tilde\imath=i,q_2\circ\tilde\jmath=j$ on $T$ where $q_1:F(S_1)\to A,q_2:F(S_2)\to B$ are the projections), then
    $$\Gamma=\langle S_1\sqcup S_2|R_1\cup R_2\cup \{\tilde\imath(t)^{-1}\tilde\jmath(t):t\in T\}\rangle$$
    is a presentation of $A\sqcup_CB$.
\end{lemma}
\begin{proof}
    Again we check the universal property.
    Suppose there is a group $G$ with $f:A\to G,g:B\to G$ homomorphisms such that $j\circ g=i\circ f$, then we have the commutative diagram
    \[
        \begin{tikzcd}
            G&&&\\
            &\Gamma\arrow[swap,dashed]{ul}{?}&A\arrow[swap]{l}{k}\arrow[swap, bend right]{ull}{f}&F(S_1)\arrow[swap]{l}{q_1}\\
            &B\arrow{u}{l}\arrow[bend left]{uul}{g}&C\arrow{l}{j}\arrow[swap]{u}{i}\arrow[hookleftarrow]{dr}&\\
            &F(S_2)\arrow{u}{q_2}&&T\arrow{ll}{\tilde\jmath}\arrow[swap]{uu}{\tilde\imath}
        \end{tikzcd}
    \]
    where $k,l$ are induced by the natural inclusions $S_1\hookrightarrow S_1\sqcup S_2,S_2\hookrightarrow S_1\sqcup S_2$ via the universal property of presentations.
    Let $\tilde{f}=f\circ q_1$ and $\tilde{g}=g\circ q_2$.
    Then $\tilde{f}(R_1)=\tilde{g}(R_2)=\{e\}\subset G$ and $\tilde{f}\circ\tilde\imath=\tilde{g}\circ\tilde\jmath$.
    Now let $\phi:F(S_1\sqcup S_2)\to G$ induced by $\tilde{f}$ on $S_1$ and $\tilde{g}$ on $S_2$.
    But then we have $\phi(R_1\cup R_2)=0$ by definiton and $\phi(\tilde\imath(t)^{-1}\tilde\jmath(t))=e$ for any $t\in T$ since $\tilde{f}\circ\tilde\imath=\tilde{g}\circ\tilde\jmath$.
    So by the universal property of presentations, $\phi$ induces the desired map.
\end{proof}
\subsection{The Seifert-van Kampen Theorem}
\begin{definition}
    let $(X,x_0),(Y,y_0)$ be based spaces, then the wedge of $X,Y$ is $X\vee Y=(X\sqcup Y)/\sim$ where $\sim$ is the smallest equivalence relations such that $x_0\sim y_0$.
    The equivalence class $[x_0]=[y_0]$ is the wedge point.
\end{definition}
So we are basically just glueing $X,Y$ together by attaching $x_0$ to $y_0$.
\begin{theorem}[Seifert-van Kampen Theorem for Wedges]\label{s-vk_wedge}
    Suppose $X=Y_1\vee Y_2$ where $Y_1,Y_2$ are path connected and let $x_0\in X$ be the wedge point.
    Then $\pi_1(X,x_0)\cong\pi_1(Y_1,x_0)\star\pi_1(Y_2,x_0)$.
\end{theorem}
\begin{proof}[Sketch of proof]
    Let $i_1:Y_1\to X,i_2:Y_2\to X$ be the natural inclusions which are continuous.
    We shall attempt to show
    \[
        \begin{tikzcd}
            \pi_1(X,x_0)&\pi_1(Y_1,x_0)\arrow[swap]{l}{(i_1)_\ast}\\
            \pi_1(Y_2,x_0)\arrow{u}{(i_2)_\ast}&\{e\}\arrow{u}\arrow{l}
        \end{tikzcd}
    \]
    is a pushout.
    Suppose we are given $f_1:\pi_1(Y_1,x_0)\to G$ and $f_2:\pi_1(Y_2,x_0)\to G$ for some group $G$.
    We need to prove that there is a unique map $g:\pi_1(X,x_0)\to G$ such that $g\circ (i_1)_\ast=f_1,g\circ (i_2)_\ast=f_2$.
    Given a loop in $X$, write it as a concatenation $\gamma=\alpha_1\beta_1\alpha_2\beta_2\cdots\alpha_n\beta_n$ with $\alpha_i$ loops in $Y_1$ and $\beta_i$ loops in $Y_2$.
    We leave out the details of the proof that it is always possible.
    Hence we necessarily have $g([\gamma])=f_1([\alpha_1])f_2([\beta_1])\cdots f_1([\alpha_n])f_2([\beta_n])$, which one can check is well-defined and works.
\end{proof}
\begin{example}
    1. The figure 8 has fundamental group $\pi_1(S^1\vee S^1)\cong\pi_1(S^1)\star\pi_1(S^1)\cong\mathbb Z\star\mathbb Z\cong F_2$.
    Worth noting that this group is nonabelian.\\
    2. Let $A$ be any finite set, then define $\bigvee_AS^1=(A\times S^1)/\sim$ where $\sim$ is the smallest equivalence relation such that $(a,1)\sim (a',1)$ for any $a,a'\in A$.
    Then inductively $\pi_1(\bigvee_AS^1,1)\cong F_{|A|}$.
    In particular, there exists spaces whose fundamental group is $F_n$ for any positive integer $n$.
\end{example}
\begin{theorem}[Seifert-van Kampen Theorem]\label{s-vk_open}
    Suppose $Y_1,Y_2\subset X$ are open, $X=Y_1\cup Y_2$ and $Z=Y_1\cap Y_2$ is nonempty.
    Suppose also that they are all path-connected.
    Let $i_k:Z\hookrightarrow Y_k$ and $j_k:Y_k\hookrightarrow X$ be the inclusions and fix $x_0\in Z$.
    Then the diagram
    \[
        \begin{tikzcd}
            \pi_1(X,x_0)&\pi_1(Y_2,x_0)\arrow[swap]{l}{(j_2)_\ast}\\
            \pi_1(Y_1,x_0)\arrow{u}{(j_1)_\ast}&\pi_1(Z,x_0)\arrow[swap]{u}{(i_2)_\ast}\arrow{l}{(i_1)_\ast}
        \end{tikzcd}
    \]
    is a pushout.
\end{theorem}
\begin{proof}
    Omitted.
\end{proof}
\begin{example}
    We want to calculate the fundamental group of the $n$-sphere $S^n$ for $n\ge 2$.
    Let $x_\pm=(\pm 1,0,\ldots,0)$ and $U_\pm=S^n\setminus\{x_\pm\}$ and $V=U_+\cap U_-$.
    Now that $V=S^n\setminus\{x_+,x_-\}\cong S^{n-1}\times (-1,1)$ via
    $$(x_1,\ldots,x_{n+1})\mapsto\left( \frac{(x_2,\ldots,x_{n+1})}{|(x_2,\ldots,x_{n+1})|},x_1 \right)$$
    which is path-connected for $n\ge 2$.
    Also, $U_\pm$ are both homeomorphic to $\mathbb R^n$ via stereographic projection.
    We know that $X=U_+\cup U_-$, so by Seifert-van Kampen the diagram
    \[
        \begin{tikzcd}
            \pi_1(S^n)&\{e\}\arrow{l}\\
            \{e\}\arrow{u}&\pi_1(V)\arrow{u}\arrow{l}
        \end{tikzcd}
    \]
    is a pushout.
    But then $\pi_1(S^n)$ is necessarily the trivial group.
    Therefore $S^n$ is simply connected for $n\ge 2$.
    Note that this argument breaks down for $n=1$ since $V$ is not path-connected in that case.
\end{example}
Note that this version of Seifeit-van Kampen does not directly generalise Theorem \ref{s-vk_wedge}.
But of course we want to have a version that generalises it, so here goes.
\begin{definition}
    A subset $Y\subset X$ is called a neighbourhood retract if there is some $V\subset X$ open and contains $Y$ such that $Y$ is a deformation retract of $V$.
\end{definition}
\begin{theorem}[Seifert-van Kampen Theorem for Closed Sets]\label{s-vk_closed}
    Suppose $Y_1,Y_2\subset X$ are closed and $Y_1\cup Y_2=X$, $Y_1\cap Y_2=Z\neq\varnothing$.
    Assume everything is path-connected and $Z$ is a neighbourhood retract in both $Y_1$ and $Y_2$, then the diagram
    \[
        \begin{tikzcd}
            \pi_1(X,x_0)&\pi_1(Y_2,x_0)\arrow[swap]{l}{(j_2)_\ast}\\
            \pi_1(Y_1,x_0)\arrow{u}{(j_1)_\ast}&\pi_1(Z,x_0)\arrow[swap]{u}{(i_2)_\ast}\arrow{l}{(i_1)_\ast}
        \end{tikzcd}
    \]
    is a pushout where as usual $i_k:Z\hookrightarrow Y_k$ and $j_k:Y_k\hookrightarrow X$ are the inclusions and $x_0\in Z$.
\end{theorem}
\begin{proof}
    Also omitted.
\end{proof}
\subsection{Attaching Cells}
\begin{definition}
    Let $X$ be a space and let $\alpha:S^{n-1}\to X$ be a map.
    The space obtained by attaching an $n$-cell to $X$ is $X\cup_\alpha D^n=(X\sqcup D^n)/\sim$ where $\sim$ is the smallest equivalence relation that identifies $x\sim \alpha(x)$ for $x\in S^{n-1}$.
\end{definition}
\begin{example}
    For $n=1$, we are just attaching a string to two (possibly one) points on $X$.
    For $n=2$, things can get pretty complicated.
    Although we are just identifying $S^1$ with a loop in $X$ and attach $D^n$ there accordingly, this can give many varieties as the loop can intersect and/or wind itself.
\end{example}
We want to study what happen to the fundamental group when we attach an $n$-cell.
\begin{lemma}
    If $n\ge 3$ and $i:X\to X\cup_\alpha D^n$ be the natural inclusion.
    The $i_\ast$ is an isomorphism of fundamental groups.
\end{lemma}
\begin{proof}
    We are going to construct something called the mapping cylinder of $\alpha$, which is the space
    $$M_\alpha=(X\sqcup(S^{n-1}\times I))/\sim$$
    where $\sim$ is the smallest equivalence relation containing $(\theta,0)\sim\alpha(\theta)$ where $\theta\in S^{n-1}$.
    Now identify $S^{n-1}$ with $S^{n-1}\times \{1\}\subset M_\alpha$ which is now a neighbourhood retract of $M_\alpha$.
    Note that $X\cup_\alpha D^n\cong M_\alpha\cup_{\operatorname{id}_{S^{n-1}}}D^n$.
    Also, $S^{n-1}$ is a neighbourhood retract in $D^n$.
    Now by Theorem \ref{s-vk_closed} with basepoint $x\in S^{n-1}\subset M_\alpha\cup_{\operatorname{id}_{S^{n-1}}}D^n$ and take $Y_1=M_\alpha$, $Y_2=D^n$ and $Z=Y_1\cap Y_2=S^{n-1}$.
    This gives us the pushout
    \[
        \begin{tikzcd}
            \pi_1(X\cup_\alpha D^n,x_0)&\pi_1(M_\alpha,x_0)\arrow{l}\\
            \pi_1(D^n,x_0)\arrow{u}&\pi_1(S^{n-1},x_0)\arrow{u}\arrow{l}
        \end{tikzcd}
    \]
    Now for $n\ge 3$, $\pi_1(S^{n-1},x_0)=\pi_1(D^n,x_0)=\{e\}$.
    which means that $\pi_1(X\cup_\alpha D^n,x_0)\cong\pi_1(M_\alpha,x_0)$.
    But $X$ is obviously a deformation retract of $M_\alpha$, therefore $\pi_1(X,x_0')\cong\pi_1(M_\alpha,x_0)\cong\pi_1(X\cup_\alpha D^n,x_0)$ where $x_0'$ is the image of $x_0$ under the deformation retract.
\end{proof}
What if we attach a $2$-cell?
\begin{lemma}
    Let $\alpha:S^1\to X$  be a map and $x_0=\alpha(\theta_0)$ for some $\theta_0\in S^1$.
    Then
    $$\pi_1(X\cup_\alpha D^2,x_0)\cong\pi_1(X,x_0)/\langle\langle [\alpha]\rangle\rangle$$
    viewing $\alpha$ as a loop based at $x_0$.
    The quotient map, in particular, is induced by the inclusion map $i:X\to X\cup_\alpha D^2$.
\end{lemma}
\begin{proof}
    By the same procedure as above we obtain the pushout
    \[
        \begin{tikzcd}
            \pi_1(X\cup_\alpha D^n,x_0)&\pi_1(X,x_0)\arrow{l}\\
            \pi_1(D^2,\theta_0)=\{e\}\arrow{u}&\pi_1(S^1,\theta_0)=\mathbb Z\arrow{l}\arrow[swap]{u}{1\mapsto [\alpha]}
        \end{tikzcd}
    \]
    The result then follows from Lemma \ref{pushout_onetrivial}.
\end{proof}
\begin{theorem}
    For any finitely presented group $G$, i.e. $G\cong\langle A|R\rangle$ where $A,R$ are both finite, there exists a compact space $X$ with $\pi_1(X,x_0)\cong G$ for some $x_0\in X$.
\end{theorem}
\begin{proof}
    Let $Y=\bigvee_AS^1$ and let $y_0$ be the common wedge point.
    We already know that $\pi_1(Y,y_0)\cong F(A)$.
    Now for each relation $r\in R$, we get a loop $\alpha_r:S^1\to Y$ representing it in the obvious way.
    Then attach $D^2$ to $Y$ via $\alpha_r$ for each $r\in R$ gives a (necessarily compact) space with fundamental group isomorphic to $G$.
\end{proof}
\subsection{Classification of Surfaces}
\begin{definition}
    An $n$ dimensional (topological) manifold (or $n$-manifold) is a Hausdorff space $M$ such that every point $x\in M$ has a neighbourhood $U$ homeomorphic to an open set of $\mathbb R^n$.
\end{definition}
\begin{example}
    1. The $S^n$ is an $n$-manifold.\\
    2. (non-example) The figure 8 is not a manifold since the wedge point does not have a neighbourhood that is locally homeomorphic to $\mathbb R^n$.\\
    3. Take $\alpha:S^1\to X=\{\ast\}$, then $Y=X\cup_\alpha D^2\cong S^2$ is a $2$-manifold.
\end{example}
There is a more interesting and influential example:
Let $g$ be a positive integer and let
$$\Gamma_{2g}=\bigvee_{i=1}^{2g}S^1_i$$
be the wedge product of $2g$ copies $S_i^1$ of the circle with a common wedge point $x_0$.
Let $\alpha_i:I\to S_i^1,\beta_i:I\to S_{i+g}^1$ be simple loops with basepoint $x_0$ for $i=1,\ldots,g$ and consider the loop
$$\rho_g=\alpha_1\beta_1\bar\alpha_1\bar\beta_1\cdots\alpha_g\beta_g\bar\alpha_g\bar\beta_g$$
If we think of $\rho_g$ as a map $S^1\to \Gamma_{2g}$ and define $\Sigma_g=\Gamma_{2g}\cup_{\rho_g}D^2$.
We claim that $\Sigma=\Sigma_g$ is a compact $2$-manifold.
Obviously any interior points of $D^2$ has an open neighbourhood homeomorphic to an open set in $\mathbb R^2$.
At non-wedge point in $S_i^1$, the path $\alpha_i$ (if $i\le g$) or $\beta_{i-g}$ (if $i>g$) appears with its inverse in $\rho_g$, so we can obtain a neighbourhood we want near that point as well.
At wedge point, note that $g=1$ gives the standard identification of the torus on a square, in which case the wedge point is simply the corners (which are identified as the same point) which obviously has a neighbourhood homeomorphic to an open set in $\mathbb R^2$.
An analogy works for higher $g$.
$\Sigma_g$ is called the orientable surface of genus $g$.\\
Now easily $g=1$ just gives the torus.
For $g=2$, we can think of the disk as an octagon and do a little bit of imagination by gluing the respective edges, which will give a $2$-torus, i.e. a torus with two holes.
A little bit more of imagination shows that $\Sigma_n$ is the $n$-torus.
Their fundamental groups are clear by our previous discussion:
$$\pi_i(\Sigma_g)=\langle a_1,\ldots,a_g,b_1,\ldots b_g|a_1b_1a_1^{-1}b_1^{-1}\cdots a_gb_ga_g^{-1}b_g^{-1}\rangle$$
Now take $\Gamma_{g+1}=\sum_{i=0}^gS_i^1$ as the wedge of $g+1$ circles and $\alpha_i:I\to S_i^1$ be the loop around the $i^{th}$ circle and $\sigma_g=\alpha_0\alpha_0\alpha_1\alpha_1\cdots\alpha_g\alpha_g$ viewed as a map $\sigma_g:\partial D^2\to\Gamma_{g+1}$.
Then take $S_g=\Gamma_{g+1}\cup_{\sigma_g}D^2$ which is a $2$=manifold called the non-orientable surface of genus $g$.
As one can see, $S_0$ is simply the real projective plane, and $S_1$ the Klein bottle.
$S_g$ for $g>1$ would shaped like attaching some orientable surface to the Klein bottle.
The fundamental groups are
$$\pi_1(S_g)\cong\langle a_0,\ldots,a_g|a_0^1a_1^2\cdots a_g^2\rangle$$
So for example $\pi_1(S_0)\cong\mathbb Z/2\mathbb Z$.
\begin{theorem}
    Any compact surface $S$ is homeomorphic to $S_g$ or $\Sigma_g$ for some $g$.
\end{theorem}
\begin{proof}
    Omitted.
\end{proof}
But how do we know that they are topologically distinct?
\begin{lemma}
    Let $g\in\mathbb N$, then $\pi_1(\Sigma_g)$ surjects onto $\mathbb Z^{2g}$ but not $\mathbb Z^{2g}\oplus(\mathbb Z/2\mathbb Z)$ and $\pi_1(S_g)$ surjects onto $\mathbb Z^g\oplus(\mathbb Z/2\mathbb Z)$ but not $\mathbb Z^{g+1}$.
\end{lemma}
\begin{proof}
    Let $\{\bar{a}_i,\bar{b}_i\}$ be the standard basis of $\mathbb Z^{2g}$.
    Then the map $a_i\mapsto\bar{a}_i,b_i\mapsto \bar{b}_i$ respects the relation, therefore by the universal property there is a surjective homomorphism from $\pi_1(\Sigma_g)$ to $\mathbb Z^{2g}$.
    Now suppose we have a surjective homomorphism $f:\pi_1(\Sigma_g)\to\mathbb Z^{2g}\oplus(\mathbb Z/2\mathbb Z)$.
    Compose it with the reduction $\mathbb Z^{2g}\oplus(\mathbb Z/2\mathbb Z)\to(\mathbb Z/2\mathbb Z)^{2g+1}$ gives a surjective homomorphism $f':\pi_1(\Sigma_g)\to(\mathbb Z/2\mathbb Z)^{2g+1}$.
    Therefore $f'(a_1),\ldots,f'(a_g),f'(b_1),\ldots,f'(b_g)$ should generate $(\mathbb Z/2\mathbb Z)^{2g+1}$, which is impossible by viewing $(\mathbb Z/2\mathbb Z)^{2g+1}$ as a vector space over $\mathbb Z/2\mathbb Z$ with dimension $2g+1$.\\
    For $\pi_1(S_g)$, let $\{\bar{a}_i\}$ be a basis for the $\mathbb Z^g$ part of $\mathbb Z^g\oplus(\mathbb Z/2\mathbb Z)$ and let $\bar{c}_0$ generate the $\mathbb Z/2\mathbb Z$ part.
    Then the map $a_0\mapsto \bar{c}_0-\sum_{i=1}^g\bar{a}_i, a_i\mapsto \bar{a}_i$ respects the relation, hence induces the desired surjective homomorphism $\pi_1(S_g)\to\mathbb Z^g\oplus(\mathbb Z/2\mathbb Z)$ via the universal property.
    Now if there is a surjective homomorphism $f:\pi_1(S_g)\to\mathbb Z^{g+1}$, then $\mathbb Z^{g+1}$ is generated by $f(a_0),\ldots,f(a_g)$.
    But then $0=f(\sigma_g)=2f(a_0)+\cdots+2f(a_g)$ which is a nontrivial relation between the generators $f(a_0),\ldots,f(a_g)$.
    This is a contradiction.
\end{proof}
\begin{corollary}
    $\Sigma_g$ and $S_g$ have mutually distinct fundamental groups.
\end{corollary}
In particular, they are mutually distinct in terms of homotopy equivalence and hence in terms of homeomorphism.
\begin{proof}
    If $\Sigma_g$ and $\Sigma_{g'}$ have the same fundamental group but $g<g'$, then by the preceding lemma, there is a surjection $\pi_1(\Sigma_g)=\pi_1(\Sigma_{g'})\to\mathbb Z^{g'}\to\mathbb Z^g\oplus(\mathbb Z/2\mathbb Z)$, contradiction.
    The other cases can be argued similarly.
\end{proof}
\begin{remark}
    We can easily generalise the fundamental group $\pi_1$ to higher dimension, which are known as the $n^{th}$ homotopy groups $\pi_n$ which is the set of homotopy classes of maps from the $n$-sphere $S^n$ to a based space with a slightly more complicated but analogous concatenation law that induces a group operation.
    Turns out, $\pi_n$ is always abelian for $n>1$.
    The problem, however, with this sort of groups is that they are hard to calculate.
    Even $\pi_n(S^m)$ for general $n,m$ are still unknown.
    So they are not an effective algebraic invariant -- we need a different approach called homology (and cohomology).
\end{remark}
    \section{Simplicial Complexes}
Our next family of algebraic invariants is called homology, which are hard to define but easy to calculate when defined.
But first, let us talk about simplicial complexes.
\subsection{Definitions}
\begin{definition}
    A finite set $V=\{v_0,\ldots,v_n\}\subset\mathbb R^m$ is said to be in general position if the smallest affine linear subspace (translation of a linear subspace) of $\mathbb R^m$ containing $V$ has dimension $n$.
\end{definition}
Consequently, no set of $n$ points in $\mathbb R^m$ would be in general position if $n>m$.
Another way of defining this is that $\{v_1-v_0,\ldots,v_n-v_0\}$ are linearly independent.
\begin{definition}
    The span or a convex hull of a set $V=\{v_0,\ldots,v_n\}\subset\mathbb R^m$ is
    $$\langle V\rangle=\left\{ \sum_{i=0}^nt_iv_i:\sum_{i=0}^nt_i=1,t_i\ge 0\right\}$$
    If $V$ is in general position, then $\langle V\rangle$ is said to be an $n$-simplex.
\end{definition}
Sometimes we omit the set bracket and just write $\langle v_0,\ldots,v_n\rangle$ as $\langle V\rangle=\langle \{v_0,\ldots,v_n\}\rangle$.
So a $0$-simplex is just a point, a $1$-simplex is a line segment, a $2$-simplex is a (solid) triangle and a $3$-simplex is a (solid) tetrahedron.
\begin{definition}
    If $V\subset\mathbb R^n$ is in general position and $U\subset V$, then $\langle U\rangle\subset\langle V\rangle$ is said to be a face of $V$, and we write $\langle U\rangle\subset\langle V\rangle$.
    Furthermore, if $U\le V$ and $U\neq V$, then we call $\langle U\rangle$ a proper face of $\langle V\rangle$.
\end{definition}
The empty set is viewed as a face by convention.
\begin{definition}
    A simplicial complex is a finite set $K$ of simplices in $\mathbb R^m$ such that:\\
    1. If $\sigma\in K$ and $\tau\le\sigma$, then $\tau\in K$.\\
    2. If $\sigma,\tau\in K$, then $\sigma\cap\tau$ is a face of both $\sigma$ and $\tau$.
\end{definition}
\begin{example}
    Just glue (well, not really, but you know what I mean) line segments, triangles, tetrahdrons together along the faces (aka edges, endpoints, etc.).
    But if we take a triangle (and its faces) and a line segment (ditto) pointing into its interior, then we do not get a simplicial complex.
\end{example}
\begin{definition}
    For a simplicial complex $K$, the dimension $\dim K$ of $K$ is the largest $n$ such that $K$ contains a $n$-simplex.
    We write $K_{(d)}=\{\sigma\in K:\dim\sigma\le d\}$ as the $d$-skeleton of $K$, which is also a simplicial complex.
\end{definition}
\begin{example}
    The set $F$ of faces of an $n$-simplex $\sigma$ is a simplicial complex, and the set of proper faces (which is called the boundary $\partial\sigma$ of $\sigma$) of $\sigma$ is also a simplicial complex and is the $(n-1)$ skeleton of $F$.
    The interior $\sigma^\circ$ is the points in $\sigma$ that is not contained in any simplex in $\partial\sigma$.
\end{example}
\begin{definition}
    The realisation of a simplicial complex $K$ is
    $$|K|=\bigcup_{\sigma\in K}\sigma\subset\mathbb R^m$$
    viewed as a topological space with topology induced as a subset of $\mathbb R^m$.\\
    If $X$ is a space, then a triangulation of $X$ is a simplicial complex $K$ and a homeomorphism $|K|\to X$.
    If a triangulation of $X$ exists, then $X$ is said to be triangulable.
\end{definition}
\begin{example}
    $D^n$ is triangulable as it is homeomorphic to the $n$-simplex.
    The realisation of the boundary of the $n$-simplex is then homeomorphic to the boundary of $D^n$ which is $S^{n-1}$, so the $n$-spheres are also triangulable.
\end{example}
\begin{remark}
    There might be many different triangulations of the same space.
\end{remark}
\begin{definition}
    Let $K,L$ be simplicial complexes.
    A simplicial map is a map $f:K\to L$ such that:
    1. Each $0$-simplex $\langle v\rangle\in K$ is sent to a $0$-simplex $\langle f(v)\rangle$ of $L$.\\
    2. For any $\langle v_0,\ldots,v_n\rangle\in K$, we have $f(\langle v_0,\ldots,v_n\rangle)=\langle f(v_0),\ldots,f(v_n)\rangle$.\\
    The realisation of $f$ is the map $|f|$ defined in the following way:
    On each $\sigma=\langle v_0,\ldots,v_n\rangle\in K$, we want the restriction of $f$ to be
    $$f|_\sigma=f_\sigma:\sum_{i=0}^nt_iv_i\mapsto\sum_{i=0}^nt_if(v_i)$$
    which is obviously continuous and consistent as $f_\sigma|_\tau=f|_\tau$ for $\tau\le\sigma$.
    So we can glue them together to get the continuous map $|f|:|K|\to |L|$.
\end{definition}
However, there is simply not enough simplicial maps to capture all possible maps $|K|\to|L|$ up to homotopy.
\begin{example}
    Consider the triangulation of the circle by gluing three line segments together (which makes it the realisation of $\partial\sigma_2$, the boundary of the $2$-simplex).
    There are at most $3^3$ simplicial maps from $\partial\sigma_2$ to itself but infinitely many different homeomorphisms (hence homotopy equivalences) from the circle to itself.
\end{example}
The solution to this problem is to introduce the barycentric subdivision to refine triangulations.
\subsection{Barycentric Subdivision and Simplicial Approximation}
\begin{definition}
    Let $V=\{v_0,\ldots, v_n\}$ be in general position, then
    $$\hat\sigma=\frac{1}{n+1}\sum_{i=0}^nv_i$$
    is the barycentre of $\sigma=\langle v\rangle$, which resides in $\sigma^\circ$.
\end{definition}
\begin{definition}
    Let $K$ be a simplicial complex, a barycentric subdivision $K'$ of $K$ is the set of simplices such that:\\
    1. The vertices ($0$-simplices) of $K'$ are the barycentres of simplices in $K$.\\
    2. $\hat\sigma_0,\ldots,\hat\sigma_n$ span a simplex of $K'$ iff $\sigma_0<\sigma_1<\ldots<\sigma_n$.
\end{definition}
So we are just dividing the original simplex with new vertices introduced by barycentres.
\begin{lemma}
    Let $K$ be a simplicial complex, then $K'$ is a simplicial complex and $|K'|=|K|$.
\end{lemma}
\begin{proof}
    First check that $\hat\sigma_0,\ldots,\hat\sigma_n$ is indeed in general position if $\sigma_0<\sigma_1<\ldots<\sigma_n$.
    Indeed, if we have $t_0,\ldots,t_n$ such that
    $$\sum_{i=0}^nt_i=0,\sum_{i=0}^nt_i\hat\sigma_i=0$$
    Take $j$ to be the largest index such that $t_j\neq 0$.
    If no such $j$ exist then we are done, otherwise
    $$\hat{\sigma}_j=\sum_{i=0}^{j-1}\left(-\frac{t_i}{t_j}\right)\hat\sigma_i$$
    is contained in $\sigma_{j-1}$ which is a proper face in $\sigma_j$, contradiction.
    Therefore $\hat\sigma_0,\ldots,\hat\sigma_n$ is in general position.\\
    Now $K'$ is clearly closed under passing to faces.
    Furthermore, if we have two simplices $\sigma',\tau'\in K'$, write $\sigma'=\langle\hat\sigma_0,\ldots,\hat\sigma_m\rangle$ for $\sigma_0<\cdots<\sigma_m$ and $\tau'=\langle\hat\tau_0,\ldots,\hat\tau_n\rangle$ for $\tau_0<\cdots<\tau_n$.
    Obviously $\sigma'\cap\tau'\subset \sigma_m\cap\tau_n$, whcih means that we can assume WLOG that $\sigma',\tau'\in\delta$ for some $\delta\in K$.
    We proceed by induction in $\dim K$.
    If one of $\sigma',\tau'$ does not contain $\hat\delta$, then $\sigma'\cap\tau'$ is contained in $\partial\delta$.
    Otherwise, both $\sigma',\tau'$ contain $\hat\delta$, then $\sigma'\cap\tau'$ is the convex hull of $\{\hat\delta\}\cap(\sigma'\cap\partial\delta)\cap(\tau'\cap\partial\delta)$.
    We are done by induction.\\
    To see $|K'|=|K|$, it suffices to show that $|K|\subset |K'|$.
    We will also use induction on $\dim K$.
    The base case is trivial.
    Let $\sigma=\langle v_0,\ldots,v_m\rangle\in K$ and $x\in\sigma$.
    If $x=\hat\sigma$, then we are done.
    Otherwise, we project $\pi:\sigma\setminus\{\hat\sigma\}\to|\partial\sigma|$ by mapping $y\in\sigma$ to the unique point in the intersection of the ray $\hat\sigma\to y$ and $|\partial\sigma|$.
    By the induction hypothesis, $\pi(x)\in\langle\hat\sigma_1,\ldots,\hat\sigma_n\rangle\in K'$ for some $\sigma_1<\cdots<\sigma_n$, which implies $x\in\langle\hat\sigma_1,\ldots,\hat\sigma_n,\hat\sigma\rangle$.
\end{proof}
\begin{definition}
    Set $K^{(0)}=K$ and $K^{(r)}=(K^{(r-1)})'$.
    The simplicial complex $K^{(n)}$ is the $n^{th}$ barycentric subdivision of $K$.
\end{definition}
\begin{definition}
    If $K$ is a simplicial complex, we set the mesh of $K$ to be
    $$\operatorname{mesh}(K)=\max_{\langle u,v\rangle\in K}|u-v|$$
\end{definition}
\begin{lemma}
    If $\dim K=n$, then
    $$\operatorname{mesh}(K^{(r)})\le\left( \frac{n}{n+1} \right)^r\operatorname{mesh}(K)$$
\end{lemma}
\begin{proof}
    Suffices to show the case for $r=1$.
    If $\langle u,v\rangle\in K'=K^{(1)}$, then we know there is some $\tau<\sigma$ such that $u=\hat\tau,v=\hat\sigma$.
    We can assume that $\tau$ is a $0$-simplex as this maximises the distance between $\hat\sigma$ and $\partial\sigma$.
    Suppose $\sigma=\langle v_0,\ldots,v_n\rangle$ and $\hat\tau=\langle v_0\rangle$, then
    \begin{align*}
        |\hat\sigma-\hat\tau|&=\left|v_0-\sum_{i=0}^m\frac{1}{m+1}v_i\right|=\left|\frac{m}{m+1}v_0-\frac{1}{m+1}\sum_{i=1}^mv_i\right|\\
        &=\left|\frac{1}{m+1}\sum_{i=1}^m(v_i-v_0)\right|\le\frac{1}{m+1}\sum_{i=1}^m|v_i-v_0|\\
        &\le\frac{m}{m+1}\operatorname{mesh}(K)\\
        &\le\frac{n}{n+1}\operatorname{mesh}(K)
    \end{align*}
    The lemma follows.
\end{proof}
Turns out, any map $|K|\to|L|$ is homotopic to a realisation of a simplicial map between some barycentric subdivisions of $K,L$.
\begin{definition}
    Let $K$ be a simplicial complex.
    The (open) star of a vertex $v$ of $K$ is the union of the interiors of simplices of $K$ containing $V$, that is,
    $$\operatorname{St}_K(v)=\bigcup_{v\in\sigma\in K}\sigma^\circ$$
\end{definition}
\begin{definition}
    Let $\phi:|K|\to|L|$ be a map.
    A simplicial map $f:K\to L$ is said to be a simplicial approximation to $\phi$ if $\phi(\operatorname{St}_K(v))\subset\operatorname{St}_L(f(v))$ for any vertex $v$ of $K$.
\end{definition}
It is easy to see that composition of simplicial approximations is also a simplicial approximation.
\begin{note}
    If $\phi=|f|$, then the equality holds.
\end{note}
\begin{lemma}
    If $f:K\to L$ is a simplicial approximation to $\phi:|K|\to|L|$, then $\phi\simeq|f|$.
\end{lemma}
\begin{proof}
    Suppose $|L|\subset\mathbb R^m$.
    We shall show that the straight line homotopy $H(x,t)=t\phi(x)+(1-t)|f|(x)$ works.
    It suffices to show that $\operatorname{Im}H\subset L$.
    Let $x\in |K|$ lie in the interior of some simplex $\sigma$, then $\phi(x)$ lies in the interior of some unique $\tau\in L$.
    We claim that $f(\sigma)\subset\tau$.
    Indeed, for any vertex $v_i$ of $\sigma$, $x\in\operatorname{St}_K(v_i)$, so $\phi(x)\in \phi(\operatorname{St}_K(v_i))\subset\operatorname{St}_L(f(v_i))$ as $f$ is a simplicial approximation of $\phi$.
    Thus $\tau^\circ\subset\operatorname{St}_L(f(v_i))$, so $f(v_i)$ is a vertex of $\tau$ which exactly means $f(\sigma)\le\tau$.
    Now since $\tau$ is convex, the line segment between $|f|(x)$ and $\phi(x)$ is contained in $\tau$, therefore $\operatorname{Im}H\subset |L|$ as desired.
\end{proof}
\begin{note}
    We did not use the assumption of $f$ being a simplicial map in the proof of $f(\sigma)\subset\tau$.
\end{note}
Does a simplicial approximation always exist?
\begin{theorem}[The Simplicial Approximation Theorem]
    Let $K,L$ be simplicial complexes and $\phi:|K|\to |L|$ a map.
    Then there exists a positive integer $r$ and a simplicial approximation $f:K^{(r)}\to L$ to $\phi:|K|=|K^{(r)}|\to|L|$.
\end{theorem}
\begin{proof}
    Consider $\mathcal U=\{\phi^{-1}(\operatorname{St}_L(v)):v=\langle v\rangle\in L\}$ is an open cover of $|K|$.
    We shall show that $\exists \delta>0$ such that $\forall x\in |K|\subset\mathbb R^m$, the ball $B(x,\delta)$ is contained in some element of $\mathcal U$.
    Indeed, if this is not true, then for each $n$ there is some $x_n\in |K|$ such that $B(x_n,1/n)$ is not contained in any open set of $\mathcal U$.
    By the compactness of $|K|$, we can assume $x_n$ converges to some $x\in |K|$ by passing to a convergent subsequence.
    But $x\in U$ for some $U\in\mathcal U$, so there is some $\epsilon>0$ such that $B(x,\epsilon)\subset U$.
    Take $n$ large enough such that $|x_n-x|<\epsilon/2$ and $1/n<\epsilon/2$, then $B(x_n,1/n)\subset U$, contradiction.\\
    Choose $r$ sufficiently large such that $\operatorname{mesh}(K^{(r)})<\delta$, the for each vertex $v$ of $K^{(r)}$, we have $\operatorname{St}_{K^{(r)}}(v)\subset B(v,\delta)\subset U$ for some $U\in\mathcal U$.
    But $U$ takes the form $\phi^{-1}(\operatorname{St}_L(u))$ for some vertex $u$ of $L$.
    So $\operatorname{St}_{K^{(r)}}(v)\subset\phi^{-1}(\operatorname{St}_L(u))$.
    We define $f(v)=u$.
    If $\sigma\in K$ and $x\in\sigma^\circ$, then $f(\sigma)$ is a face of $\tau$ which is the unique simplex of $L$ containing $\phi(x)$ in its interior (this can be proved in the exact same way we proved the claim in the preceding lemma).
    So $f(\sigma)$ is a simplex of $L$, therefore $f$ is a simplicial map and hence approximation to $\phi$ since we have proven $\phi(\operatorname{St}_{K^{(r)}}(v))\subset\operatorname{St}_L(u)$.
\end{proof}
    \section{Simplicial Homology}
\subsection{Oriented Simplices and Boundary Homomorphism}
We slightly modify our definition of simplex by associating each simplex with an orientation.
For a simplex $\langle v_0,\ldots,v_n\rangle$, we can shuffle the vertices around which defines an action of $S_{n+1}$ on it.
Write $A_{n+1}\unlhd S_{n+1}$ as the alternating group, then it also acts on the simplex by restriction.
An orientation on $\sigma=\langle v_0,\ldots,v_n\rangle$ is a choice of ordering defined up to the action of $A_{n+1}$.
So we can now view $\langle v_0,\ldots,v_n\rangle$ as a simplex with an orientation.
\begin{example}
    $0$-simplex does not have a notion of orientation.\\
    There are two possible orderings of $1$-simplex, namely $\langle v_0,v_1\rangle$ and $\langle v_1,v_0\rangle$ which have different orientation.\\
    For $2$-simplices, again there are two orientations and they are $\langle v_0,v_1,v_2\rangle=\langle v_1,v_2,v_0\rangle=\langle v_2,v_0,v_1\rangle$ and $\langle v_1,v_0,v_2\rangle=\langle v_0,v_2,v_1\rangle=\langle v_2,v_1,v_0\rangle$.
\end{example}
So afterwards when we mention simplex we mean oriented simplex.
\begin{definition}
    Let $K$ be a simplicial complex, we define the group $C_n(K)$ of $n$-chains to be the free abelian group generated by the simplices of dimension $n$ in $K$, i.e.
    $$C_n(K)=\bigoplus_{\sigma\in K,\dim\sigma=n}\langle\sigma\rangle$$
    where $\langle\sigma\rangle$ is the free group generated by $\sigma$.
\end{definition}
We will convention that we have made our choice of some orientation.
For a simplex $\sigma$ in this orientation, the same simplex in the other orientation is denoted $\bar\sigma$.
In $\langle\sigma\rangle$, we identify $\bar\sigma$ by $-\sigma$.
\begin{definition}
    The $n^{th}$ boundary homomorphism is a homomorphism $\partial=\partial_n:C_n(K)\to C_{n-1}(K)$ induced by
    $$\partial_n\langle v_0,\ldots,v_n\rangle = \sum_{i=0}^n(-1)^i\langle v_1,\ldots,\hat{v}_i,\ldots,v_n\rangle$$
    where $\langle v_1,\ldots,\hat{v}_i,\ldots,v_n\rangle=\langle v_0,\ldots,v_{i-1},v_{i+1},\ldots,v_n\rangle$.
\end{definition}
\begin{example}
    $\partial\langle v_0,v_1\rangle=\langle v_1\rangle-\langle v_0\rangle$.
    $\partial\langle v_0,v_1,v_2\rangle=\langle v_1,v_2\rangle-\langle v_0,v_2\rangle+\langle v_0,v_1\rangle=\langle v_1,v_2\rangle+\langle v_2,v_0\rangle+\langle v_0,v_1\rangle$ which is indeed the (oriented) topological boundary of the simplex $\langle v_0,v_1,v_2\rangle$.
\end{example}
\begin{remark}
    We have $\partial\bar\sigma=-\partial\sigma$.
\end{remark}
\subsection{The Homology Groups of Simplicial Complexes}
\begin{definition}
    Let $K$ be a simplicial complex and $n\in\mathbb Z$, the group of $n$-cycles of $K$ is $Z_n(K)=\ker\partial_n$.
    The group of $n$-boundaries is $B_n(K)=\operatorname{Im}\partial_{n+1}$.
\end{definition}
\begin{lemma}
    $B_n(K)\subset Z_n(K)$, or in other words $\partial_{n-1}\circ\partial_n=0$.
\end{lemma}
\begin{proof}
    Pick an $n$-simplex $\langle v_0,\ldots,v_n\rangle$, then
    \begin{align*}
        \partial_{n-1}\circ\partial_n(\langle v_0,\ldots,v_n\rangle)&=\partial_{n-1}\left( \sum_{i=0}^n(-1)^i\langle v_0,\ldots,\hat{v}_i,\ldots,v_n\rangle \right)\\
        &=\sum_{i=0}^n(-1)^i\partial_{n-1}(\langle v_0,\ldots,\hat{v}_i,\ldots,v_n\rangle)\\
        &=\sum_{j<i}(-1)^j(-1)^i\langle v_0,\ldots,\hat{v}_j,\ldots,\hat{v}_i,\ldots,v_n\rangle\\
        &\quad+\sum_{j>i}(-1)^{j-1}(-1)^i\langle v_0,\ldots,\hat{v}_i,\ldots,\hat{v}_j,\ldots,v_n\rangle\\
        &=0
    \end{align*}
    as desired.
\end{proof}
\begin{definition}
    The $n^{th}$ (simplicial) homology group of the simplicial complex $K$ is defined as $H_n(K)=Z_n(K)/B_n(K)$.
\end{definition}
\begin{example}
    Take the simplicial complex $K$ generated by boundary of a $2$-simplex (which can be taken as a triangulation of $S^1$).
    Then $C_0(K)\cong C_1(K)\cong\mathbb Z^3$ and $C_n(K)=0$ for all $n>1$.
    So we only need to understand $\partial_1$.
    Say its $0$-simplices are $\langle v_0\rangle,\langle v_1\rangle,\langle v_2\rangle$ and $1$-simplices are $\langle v_0,v_1\rangle,\langle v_1,v_2\rangle,\langle v_2,v_0\rangle$, then $\partial_1:C_1(K)\to C_0(K)$ maps
    \begin{align*}
        \langle v_0,v_1\rangle &\mapsto \langle v_1\rangle-\langle v_0\rangle\\
        \langle v_1,v_2\rangle &\mapsto \langle v_2\rangle-\langle v_1\rangle\\
        \langle v_2,v_0\rangle &\mapsto \langle v_0\rangle-\langle v_2\rangle
    \end{align*}
    So if we take the free abelian groups as free $\mathbb Z$-modules generated by the simplices, then $\partial_1$ has the matrix
    $$\begin{pmatrix}
        -1&0&1\\
        1&-1&0\\
        0&1&-1
    \end{pmatrix}$$
    So $Z_1(K)=\ker\partial_1=\langle(1,1,1)\rangle=\langle \langle v_0,v_1\rangle +\langle v_1,v_2\rangle + \langle v_2,v_0\rangle\rangle$ and $B_1(K)=\ker\partial_2=0$, therefore $H_1(K)\cong\mathbb Z$.\\
    As for the $H_0(K)$, we have $Z_0(K)=\ker\partial_0=C_0(K)\cong\mathbb Z^3$ and $B_0(K)=\operatorname{Im}\partial_1\cong\langle (-1,1,0),(0,-1,1)\rangle$, hence $H_0(K)\cong\mathbb Z^3/\langle (-1,1,0),(0,-1,1)\rangle\cong\mathbb Z$.
    And $H_n(K)=0$ for $n>1$.
\end{example}
\begin{example}
    Take $L$ the simplicial complex generated by the $2$-simplex (so it is a solid triangle which is a triangulation of the closed unit disk).
    Then $C_0(L)=C_0(K),C_1(L)=C_1(K)$ but $C_2(L)=\langle\langle v_0,v_1,v_2\rangle\rangle$ where $K$ is as in the previous example.
    Now $\partial_2(\langle v_0,v_1,v_2\rangle)=\langle v_0,v_1\rangle+\langle v_1,v_2\rangle+\langle v_2,v_0\rangle$ which is the generator of $\ker\partial_1$, therefore $Z_1(L)=B_1(L)$ and hence $H_1(L)=0$.
    Note that $L,K$ coincides on $0$ and $1$-simplices they contain, so $H_0(L)\cong H_0(K)\cong\mathbb Z$.
    Now easily $Z_2(L)=\ker\partial_2=0$ and therefore $H_2(L)=0$, therefore $H_0(L)\cong\mathbb Z$ and $H_n(L)=0$ for any $n\neq 0$.
\end{example}
\begin{lemma}
    Let $K$ be a simplicial complex.
    If $d$ is the number of path components of $|K|$, then $H_0(K)\cong\mathbb Z^d$.
\end{lemma}
\begin{proof}
    Denote by $\pi_0(K)$ the set of path-connected components of $|K|$.
    Write $\mathbb Z[A]$ as the free abelian group generated by a set $A$, then $\mathbb Z[\pi_0(K)]\cong\mathbb Z^d$.
    Consider a $q:C_0(K)\to\mathbb Z[\pi_0(K)]$ sending a vertex $\langle v\rangle$ to the path component containing $v$ and extend it to a homomorphism.
    Then $q$ is surjective.
    Note that $\partial_0=0$, so $Z_0(K)=C_0(K)$, therefore $H_0(K)\cong C_0(K)/B_0(K)$, therefore it suffices to show that $\ker q=B_0(K)$.
    If $\langle v_0,v_1\rangle\in C_1(K)$, then $q\circ\partial_1(\langle v_0,v_1\rangle)=q(\langle v_1\rangle-\langle v_1\rangle)=0$ since $v_0$ and $v_1$ are joined by a path.
    This means that $B_0(K)\subset\ker q$.
    Conversely, $\ker q$ is generated by elements of $C_0(K)$ of the form $\langle w\rangle-\langle v\rangle$ where $v,w$ are in the same path components of $|K|$.
    But then there is a sequence of vertices $v=v_1,v_2,\ldots,v_n=w$ in $K$ with $\langle v_i,v_{i+1}\rangle\in K$.
    Then
    $$\langle w\rangle-\langle v\rangle=(\langle v_n\rangle-\langle v_{n-1}\rangle)+(\langle v_{n-1}\rangle-\langle v_{n-2}\rangle)\cdots +(\langle v_2\rangle-\langle v_1\rangle)\in\operatorname{Im}\partial_1=B_0(K)$$
    Hence $\ker q\subset B_0(K)$.
    This completes the proof.
\end{proof}
\begin{remark}
    There is a very rough analogy between $\pi_1(|K|)$ and $H_1(K)$ as $B_1(K)$ is kind of homotopies between the loops in $Z_1(K)$.
    They are of course not the same as $H_1(K)$ is always abelian, but in fact, there does exist a certain connection as one can show that $H_1(K)\cong\pi_1(|K|)^{\operatorname{ab}}$.
\end{remark}
\subsection{Chain Maps and Homotopies}
We want to understand the maps on homology that is induced by maps of simplicial complexes.
\begin{definition}
    A chain complex $C_\bullet$ is a sequence of abelian groups $C_n,n\in\mathbb Z$ with homomorphisms $\partial_n:C_n\to C_{n-1}$ such that $\partial_{n-1}\circ\partial_n=0$ for all $n$.\\
    A chain map $f_\bullet:C_\bullet\to D_\bullet$ between chain complexes is a collection of homomorphisms $f_n:C_n\to D_n$ indexed by $n\in\mathbb Z$ such that
    \[
        \begin{tikzcd}
            C_n\arrow{r}{\partial_n}\arrow[swap]{d}{f_n}&C_{n-1}\arrow{d}{f_{n-1}}\\
            D_n\arrow[swap]{r}{\partial_n}&D_{n-1}
        \end{tikzcd}
    \]
    commutes for any $n$.
\end{definition}
Homology usually deals with the cases where nontrivial groups only occur at nonnegative $n$.
In these situations, we can just define $C_n$ for $n\ge 0$ and $\partial_n$ for $n\ge 1$ -- because those are what we care about -- and leave the rest of the groups and boundary maps to be zero.
This will be the case for the simplicial complexes.
\begin{definition}
    Given a chain complex $C_\bullet$, we define the group of $n$-cycles to be $Z_n(C_\bullet)=\ker\partial_n$ and the group of $n$-boundaries to be $B_n(C_\bullet)=\operatorname{Im}\partial_n$.
    Then $B_n(C_\bullet)\unlhd Z_n(C_\bullet)$, so we define the $n^{th}$ homology group is then $H_n(C_\bullet)=Z_n(C_\bullet)/B_n(C_\bullet)$.
\end{definition}
\begin{lemma}
    If $f_\bullet:C_\bullet\to D_\bullet$ is a chain map, then for any $n\in\mathbb Z$, we have a well-defined homomorphism $f_\ast:H_n(C_\bullet)\to H_n(D_\bullet)$ via $[c]\mapsto [f_n(c)]$ for $c\in Z_n(C_\bullet)$.
\end{lemma}
\begin{proof}
    Suffices to show that the map is well-defined.
    For $c\in Z_n(C_\bullet)$, we have $\partial_n\circ f_n(c)=f_{n-1}\circ\partial_n(c)=0$, so indeed $f_n(c)\in Z_n(D_\bullet)$.
    Also, if $c\in B_n(C_\bullet)$, then there is some $c'\in C_{n-1}$ such that $c=\partial_{n+1}(c')$, so $f_n(c)=f_n\circ\partial_{n+1}(c')=\partial_{n+1}\circ f_{n+1}(c')$, therefore $f_n(c)\in B_n(D_\bullet)$.
    Hence $f_\ast$ is well-defined.
\end{proof}
\begin{lemma}
    A simplicial map $f:K\to L$ induces a chain map $f_\bullet:C_\bullet(K)\to C_\bullet(L)$ via
    $$f_n:\sigma\mapsto\begin{cases}
        f(\sigma)\text{, if $\dim f(\sigma)=n$}\\
        0\text{, otherwise}
    \end{cases}$$
    for $\sigma\in K,\dim\sigma=n$.
    Hence for each $n\in\mathbb N$, $f_\bullet$ induces a homomorphism $f_\ast:H_n(K)\to H_n(L)$.
\end{lemma}
\begin{proof}
    We need to show that $\partial_n\circ f_n=f_{n-1}\circ\partial_n$.
    Suffices to demonstrate this on generators.
    Let $\sigma=\langle v_0,\ldots,v_n\rangle$.
    If $\dim f(\sigma)=n$, then $f(\sigma)=\langle f(v_0),\ldots,f(v_n)\rangle$, then there is a one-to-one correspondence between faces of $\sigma$ and fases of $f(\sigma)$, hence necessarily $f_{n-1}\circ\partial_n(\sigma)=\partial_n\circ f_n(\sigma)$.
    If $\dim f(\sigma)\le n-2$, then $f_{n-1}\circ\partial_n(\sigma)=0=\partial_n\circ f_n(\sigma)$.
    We are left with the case $\dim f(\sigma)=n-1$.
    Assume $f(v_0)=f(v_1)$ and $f(v_1),\ldots,f(v_n)$ are all distinct.
    In that case $f(\langle v_0,\ldots,v_n\rangle)=f(\langle v_1,\ldots,v_n\rangle)$.
    We know that $f_n(\sigma)=0$, so $\partial_n\circ f_n(\sigma)=0$.
    Now
    \begin{align*}
        f_{n-1}\circ\partial_n(\sigma)&=f_{n-1}\left(\sum_{i=0}^n(-1)^i\langle v_0,\ldots,\hat{v}_i,\ldots,v_n\rangle\right)\\
        &=\sum_{i=0}^n(-1)^if_{n-1}(\langle v_0,\ldots,\hat{v}_i,\ldots,v_n\rangle)\\
        &=f_{n-1}(\langle v_1,\ldots,v_n\rangle)-f_{n-1}(\langle v_0,v_2,\ldots,v_n\rangle)\\
        &=0
    \end{align*}
    Therefore $\partial_n\circ f_n(\sigma)=0=f_{n-1}\circ\partial_n(\sigma)$ too, which means $f_\bullet$ is indeed a chain map.
\end{proof}
\begin{remark}
    If $f:K\to L$ and $g:L\to M$ are simplicial maps, then $(g\circ f)_\ast=g_\ast\circ f_\ast$.
    Also, if $K$ is a simplicial complex, then $(\operatorname{id}_K)_\ast=\operatorname{id}_{H_n(K)}$.
\end{remark}
A natural question is that when do chain maps induce the same maps on homology.
\begin{definition}
    Let $f_\bullet,g_\bullet:C_\bullet\to D_\bullet$ be chain maps.
    A chain homotopy $h_\bullet$ between $f_\bullet$ and $g_\bullet$ is a collection of homomorphisms $h_n:C_n\to D_{n+1}$ such that $g_n(c)-f_n(c)=\partial_{n+1}\circ h_n(c)+h_{n-1}\circ\partial_n(c)$.
    We say $f_\bullet$ and $g_\bullet$ are chain homotopic, written as $f_\bullet\simeq g_\bullet$ if such $h_\bullet$ exists.
\end{definition}
\begin{lemma}
    If $f_\bullet\simeq g_\bullet:C_\bullet\to D_\bullet$, then $f_\ast=g_\ast$.
\end{lemma}
\begin{proof}
    Let $c\in Z_n(C_\bullet)$, then $g_n(c)-f_n(c)=\partial_{n+1}\circ h_n(c)+h_{n-1}\circ\partial_n(c)=\partial_{n+1}\circ h_n(c)\in B_n(D_\bullet)$, therefore $[g_n(c)]=[f_n(c)]$.
\end{proof}
\begin{example}
    Consider the triangle $K$ and the line segment $L$, both as simplicial complexes in the obvious way.
    Say the vertices in $K$ are $e_0,e_1,e_2$ and those in $L$ are $e_0,e_1$ and let $i:L\to K$ be the natural inclusion $e_0\mapsto e_0,e_1\mapsto e_1$, $r$ be the simplicial retraction $e_0\mapsto e_0,e_1\mapsto e_1,e_2\mapsto e_0$, both as simplicial maps.
    Now $r\circ i=\operatorname{id}_L$, but $i\circ r\neq \operatorname{id}_K$.
    However, we can define a chain homotopy between $(i\circ r)_\bullet$ and $\operatorname{id}_{C_\bullet(K)}$.
    This would be given by $h_\bullet$ which is everywhere zero except $h_0(\langle e_2\rangle)=\langle e_2,e_0\rangle$ and $h_1(\langle e_1,e_2\rangle)=-\langle e_0,e_1,e_2\rangle$ which works since
    \begin{align*}
        (\partial_1\circ h_0+h_{-1}\circ\partial_0)(\langle e_2\rangle)&=\partial_1(\langle e_2,e_0\rangle)\\
        &=\langle e_0\rangle-\langle e_2\rangle\\
        &=(i_0\circ r_0-\operatorname{id}_{C_0(K)})(\langle e_2\rangle)\\
        (\partial_2\circ h_1+h_0\circ\partial_1)(\langle e_1,e_2\rangle)&=\partial_2(-\langle e_0,e_1,e_2\rangle)+h_0(\langle e_2\rangle-\langle e_1\rangle)\\
        &=-\langle e_0,e_1\rangle-\langle e_1,e_2\rangle-\langle e_2,e_0\rangle+\langle e_2,e_0\rangle\\
        &=(i_1\circ r_1-\operatorname{id}_{C_1(K)})(\langle e_1,e_2\rangle)\\
        (\partial_3\circ h_2+h_1\circ\partial_2)(\langle e_0,e_1,e_2\rangle)&=h_1(\langle e_0,e_1\rangle+\langle e_1,e_2\rangle+\langle e_2,e_0\rangle)\\
        &=-\langle e_0,e_1,e_2\rangle\\
        &=(i_2\circ r_2-\operatorname{id}_{C_2(K)})(\langle e_0,e_1,e_2\rangle)
    \end{align*}
    Therefore $(i\circ r)_\ast=i_\ast\circ r_\ast$ would just be the identity on $H_n(K)$.
    In particular, $r_\ast$ is an isomorphism on the homology groups.
\end{example}
\begin{definition}
    A simplicial complex $K$ is a cone if there is a vertex $x_0$ such that for all other simplices $\tau\in K$, there exists $\sigma\in K$ such that $x_0\in\sigma$ and $\tau\le\sigma$.
\end{definition}
Perhaps nonsurprisingly,
\begin{lemma}
    If $K$ is a cone, then $H_0(K)\cong\mathbb Z$ and $H_n(K)=0$ if $n\neq 0$.
\end{lemma}
\begin{proof}
    Let $i:\{\langle x_0\rangle\}\to K$ be the obvious inclusion and $r:K\to\{\langle x_0\rangle\}$ be constant.
    Then $r\circ i=\operatorname{id}_{\{\langle x_0\rangle\}}$, therefore $r_\ast\circ i_\ast=\operatorname{id}_{H_n(\{\langle x_0\rangle\})}$.
    We shall show that $i_\ast\circ r_\ast=\operatorname{id}_{H_n(K)}$ which implies $H_n(K)\cong H_n(\{\langle x_0\rangle\})$ from where the result follows.\\
    We will build a chain homotopy between $\operatorname{id}_{C_\bullet(K)}$ and $i_\bullet\circ r_\bullet$ where $i_\bullet$ and $r_\bullet$ are the induced chain maps.
    Let $\sigma=\langle v_0,\ldots,v_n\rangle\in K$, then we define
    $$h_n(\sigma)=\begin{cases}
        0\text{, if $x_0\in\sigma$}\\
        \langle x_0,v_0,\ldots,v_n\rangle\text{, otherwise}
    \end{cases}$$
    which is well-defined as $K$ is a cone with vertex $x_0$.
    We want to show that $\partial_{n+1}\circ h_n+h_{n-1}\circ\partial_n=\operatorname{id}_{C_n(K)}-i_n\circ r_n$.
    Suppose $n>0$ and $x_0\notin\sigma$, then
    \begin{align*}
        &\quad(\partial_{n+1}\circ h_n+h_{n-1}\circ\partial_n)(\sigma)\\
        &=\partial_{n+1}(\langle x_0,v_0,\ldots,v_n\rangle)+h_{n-1}\left( \sum_{i=0}^n(-1)^i\langle v_0,\ldots,\hat{v}_i,\ldots,v_n \rangle\right)\\
        &=\langle v_0,\ldots,v_n\rangle+\sum_{i=0}^n(-1)^{i+1}\langle x_0,v_0,\ldots,\hat{v}_i,\ldots,v_n \rangle\\
        &\quad+\sum_{i=0}^n(-1)^i\langle x_0,v_0,\ldots,\hat{v}_i,\ldots,v_n \rangle\\
        &=\langle v_0,\ldots,v_n\rangle=\sigma\\
        &=(\operatorname{id}_{C_n(K)}-i_n\circ r_n)(\sigma)
    \end{align*}
    which works.
    If $n>0$ but $x_0\in\sigma$, then $x_0=v_j$ for some $j$.
    Consequently,
    \begin{align*}
        &\quad(\partial_{n+1}\circ h_n+h_{n-1}\circ\partial_n)(\sigma)\\
        &=0+h_{n-1}\left( \sum_{i=0}^n(-1)^i\langle v_0,\ldots,\hat{v}_i,\ldots,v_n \rangle\right)\\
        &=(-1)^j\langle x_0=v_j,v_0,\ldots,\hat{v}_j,\ldots,v_n\rangle\\
        &=\langle v_0,\ldots,v_n\rangle=\sigma\\
        &=(\operatorname{id}_{C_n(K)}-i_n\circ r_n)(\sigma)
    \end{align*}
    The case $n=0$ is trivial.
    Therefore $h_\bullet$ is indeed a chain homotopy as desired.
\end{proof}
\begin{example}
    Any $n$-simplex $K$ is a cone, so by the lemma
    $$H_n(K)\cong\begin{cases}
        \mathbb Z\text{, if $n=0$}\\
        0\text{, otherwise}
    \end{cases}$$
    Take $L=\partial\sigma_n$ be the boundary of the $n$-simplex, then $|L|\cong S^{n-1}$ for $n\ge 2$.
    The obvious inclusion $L\hookrightarrow K$ of simplicial complexes induces a chain map
    \[
        \begin{tikzcd}
            0\arrow{r}&0\arrow{r}\arrow{d}&C_{n-1}(L)\arrow{r}{\partial_{n-1}}\arrow[equal]{d}&\cdots\arrow{r}{\partial_1}&C_0(L)\arrow{r}\arrow[equal]{d}&0\\
            0\arrow{r}&C_n(K)\arrow[swap]{r}{\partial_n}&C_{n-1}(K)\arrow[swap]{r}{\partial_{n-1}}&\cdots\arrow[swap]{r}{\partial_1}&C_0(K)\arrow{r}&0
        \end{tikzcd}
    \]
    So evidently $H_d(L)\cong H_d(K)$ for $d\le n-2$.
    Since $H_{n-1}(K)=0$, we have $Z_{n-1}(L)=Z_{n-1}(K)=B_{n-1}(K)$.
    Also $B_{n-1}(L)=0$, therefore $H_{n-1}(L)\cong Z_{n-1}(L)=B_{n-1}(K)$.
    But this is easy enough to calculate:
    $K$ only has one $n$-simplex $\sigma$, so $C_n(K)=\mathbb Z\sigma$ and hence $B_{n-1}\cong\mathbb Z$ as $\partial_n$ has to be injective.
    Thus
    $$H_d(L)=\begin{cases}
        \mathbb Z\text{, if $d=0$ or $d=n-1$}\\
        0\text{, otherwise}
    \end{cases}$$
    This means homology can actually detect ``higher dimensional holes''.
\end{example}
\subsection{Continuous Maps and Homotopies}
For a map $\phi:|K|\to|L|$, we want to associate to it a homomorphism $\phi_\ast:H_n(K)\to H_n(L)$.
Note that a chief difficulty in this is that $\phi$ may not contain much information about the structures of $K,L$ as simplicial complexes, since it is just a continuous map between topological spaces.
The idea is to use a simplicial approximation.
That is, instead of looking for $\phi_\ast:H_n(K)\to H_n(L)$ directly, we seek $\phi_\ast:H_n(K^{(r)})\to H_n(L^{(r)})$ for sufficiently large $r$ and show that $H_n(K^{(r)})\cong H_n(K)$ for any simplicial complex $K$ and $r\in\mathbb N$.
Eventually, we will show that $H_n(K)$ only depends on $|K|$.\\
First step on that journey is the notion of homotopy on simplicial maps.
\begin{definition}
    Two simplicial maps $f,g:K\to L$ are contiguous if, for every $\sigma\in K$, there exists some $\tau\in L$ such that $f(\sigma)$ and $g(\sigma)$ are both faces of $\tau$.
\end{definition}
\begin{remark}
    Suppose given $\phi:|K|\to |L|$ with $f,g:K\to L$ different simplicial approximations to $\phi$.
    Then choose any $x\in\sigma^\circ,\phi(x)\in\tau^\circ$, we know that $f(\sigma)\le\tau$ and $g(\sigma)\le\tau$ and hence $f,g$ are contiguous.
\end{remark}
\begin{lemma}
    If $f,g:K\to L$ are contiguous, then $f_\ast=g_\ast:H_n(K)\to H_n(L)$ for all $n$.
\end{lemma}
\begin{proof}
    We shall construct a chain homotopy.
    Fix a total order $<$ on vertices of $K$ and use the convention that $\sigma=\langle v_0,\ldots,v_n\rangle$ is oriented in such a way that $v_0<\ldots<v_n$.
    We write $\langle v_0,\ldots,v_n\rangle=0$ if $v_0,\ldots,v_n$ are not in general position.
    Easy to see this is compatible with everything.
    Define $h_n:C_n(K)\to C_{n+1}(L)$ by
    $$h_n(\langle v_0,\ldots,v_n\rangle)=\sum_{i=0}^n(-1)^i\langle f(v_0),\ldots,f(v_i),g(v_i),\ldots,g(v_n)\rangle$$
    Note that whenever the summand is nonzero, it would be an $(n+1)$-simplex of $L$ since $f,g$ are contiguous.
    We have some calculations to do.
    \begin{align*}
        (\partial\circ h+h\circ\partial)(\sigma)&=\partial\left( \sum_{i=0}^n(-1)^i\langle f(v_0),\ldots,f(v_i),g(v_i),\ldots,g(v_n)\rangle \right)\\
        &\quad+h\left( \sum_{i=0}^n(-1)^i\langle v_0,\ldots,\hat{v}_i,\ldots,v_n\rangle \right)\\
        &=\sum_{i\le j}(-1)^{i+j}\langle f(v_0),\ldots,\widehat{f(v_i)},\ldots,f(v_j),g(v_j),\ldots,g(v_n)\rangle\\
        &\quad-\sum_{i\ge j}(-1)^{i+j}\langle f(v_0),\ldots, f(v_j),g(v_j),\ldots,\widehat{g(v_i)},\ldots,g(v_n)\rangle\\
        &\quad+\sum_{j<i}(-1)^{i+j}\langle f(v_0),\ldots, f(v_j),g(v_j),\ldots,\widehat{g(v_i)},\ldots,g(v_n)\rangle\\
        &\quad-\sum_{j>i}(-1)^{i+j}\langle f(v_0),\ldots,\widehat{f(v_i)},\ldots,f(v_j),g(v_j),\ldots,g(v_n)\rangle\\
        &=\sum_{i=0}^n\langle f(v_0),\ldots,f(v_{i-1}),g(v_i),\ldots,g(v_n)\rangle\\
        &\quad-\sum_{i=0}^n\langle f(v_0),\ldots,f(v_i),g(v_{i+1}),\ldots,g(v_n)\rangle\\
        &=\langle g(v_0),\ldots,g(v_n)\rangle-\langle f(v_0),\ldots,f(v_n)\rangle\\
        &=g(\sigma)-f(\sigma)
    \end{align*}
    as desired.
\end{proof}
\begin{lemma}
    Let $K$ be a simplicial complex and $K'$ be its barycentric subdivision.
    A simplicial map $s:K'\to K$ is a simplicial approximation to $\operatorname{id}_{|K|}$ iff for every $\sigma\in K$, $s(\hat\sigma)$ is a vertex of $\sigma$.
    Also, such $s$ always exists.
\end{lemma}
\begin{proof}
    Let $s:K'\to K$ be a simplicial approximation to $\operatorname{id}_{|K|}$, which just means $\operatorname{id}_{|K|}(\operatorname{St}_{K'}(\hat\sigma))\subset\operatorname{St}_K(s(\hat\sigma))$.
    In particular, $\sigma\circ\subset\operatorname{St}_K(s(\hat\sigma))$, therefore $s(\hat\sigma)$ is a vertex of $\sigma$.\\
    Conversely, suppose $s(\hat\sigma)$ be a vertex of $\sigma$ for any $\sigma\in K$.
    Choose any $\tau'\in K'$ with $\tau'^\circ\subset\operatorname{St}_{K'}(\hat\sigma)$ (i.e. $\hat\sigma$ is a vertex of $\tau'$).
    Then $\tau'^\circ$ is contained in the interior of a simplex $\tau\in K$ such that $\sigma\le\tau$.
    Thus $s(\hat\sigma)$ is also a vertex of $\tau$.
    But $\tau'^\circ\subset\tau^\circ\subset\operatorname{St}_K(s(\hat\sigma))$.
    But such $\tau'^\circ$ necessarily cover $\operatorname{St}_{K'}(\hat\sigma)$, therefore $\operatorname{id}_{|K|}(\operatorname{St}_{K'}(\hat\sigma))\subset\operatorname{St}_K(s(\hat\sigma))$ as desired.\\
    To see such an $s$ exists, we simply just need to send $\hat\sigma$ to an arbitrarily chosen vertex of $\sigma$ which, as one can verify, works.
\end{proof}
\begin{proposition}\label{barycentric_iso_homol}
    Let $s:K'\to K$ be the simplicial approximaion to the identity obtained like in the proof above, then $s_\ast:H_n(K')\to H_n(K)$ is an isomorphism for all $n$.
\end{proposition}
We will postpone this proof until more machinery is developed.
But let us see some implications first.
\begin{corollary}
    Let $K$ be a simplicial complex, then for all $r$, there is a canonical isomorphism $H_n(K)\cong H_n(K^{(r)})$.
\end{corollary}
\begin{proof}
    Suffices to make the choice for $r=1$.
    Choose a simplicial approximation $s:K'\to K$ to the identity on $|K|$ which induces an ismorphism $s_\ast:H_n(K')\to H_n(K)$ by the preceding proposition.
    To see this is canonical, we shall show that this isomorphism is independent of the choice of $s$.
    But this is obvious since any other choice $s'$ is contiguous with $s$.
\end{proof}
We write $\nu_{K,r,s}:H_n(K^{(r)})\to H_n(K^{(s)})$ as the canonical ismorphism for $r\ge s$ and write $\nu_{K,r}=\nu_{K,r,0}$.
Then $\nu_{K,r_2,r_3}\circ \nu_{K,r_1,r_2}=\nu_{K,r_1,r_3}$.
\begin{proposition}
    To each continuous map $f:|K|\to|L|$, there is an associated homomorphism $f_\ast:H_n(K)\to H_n(L)$ given by $f_\ast=s_\ast\circ\nu_{K,r}^{-1}$ where $s:K^{(r)}\to L$ is a simplicial approximation to $f$.
    This homomorphism does not depend on the choice of $r$ or $s$.
    Furthermore, if $g:|M|\to |K|$ is continuous for some other simplicial complex $M$, then $(f\circ g)_\ast=f_\ast\circ g_\ast$.
\end{proposition}
\begin{proof}
    We already know that the homomorphism does not depend on the choice of $s$.
    If $s:K^{(r)}\to L,t:K^{(q)}\to L$ are both simplicial approximations to $f$ where WLOG $r\ge q$, then let $a:K^{(r)}\to K^{(q)}$ be a simplicial approximation to the identity on $|K|=|K^{(q)}|$.
    Now $s,t\circ a:K^{(r)}\to L$ are both simplicial approximations to $f$, hence are contiguous and induces the same homomorphism $s_\ast=(t\circ a)_\ast=t_\ast\circ a_\ast=t_\ast\circ\nu_{K,r,q}$, so $s_\ast\circ\nu_{K,r}^{-1}=t_\ast\circ\nu_{K,r,q}\circ\nu_{K,r}^{-1}=t_\ast\circ\nu_{K,q}$ as desired.\\
    Now let $s:K^{r}\to L$ and $t:M^{q}\to K^{r}$ be simplicial approximations to $f,g$ respectively (here we used $|K|=|K^{(r)}|$).
    Then $s\circ t$ is a simplicial approximation of $f\circ g$, so
    $$(f\circ g)_\ast=(s\circ t)_\ast\circ\nu_{M,q}^{-1}=s_\ast\circ t_\ast\circ \nu_{M,q}^{-1}=(s_\ast\circ\nu_{K,r}^{-1})\circ(\nu_{K,r}\circ t_\ast\circ\nu_{M,q}^{-1})=f_\ast\circ g_\ast$$
    as desired.
\end{proof}
And now we finally arrive at:
\begin{corollary}
    If $|K|\cong|L|$, then $H_n(K)\cong H_n(L)$.
\end{corollary}
\begin{proof}
    Immediate.
\end{proof}
We can do even better.
\begin{lemma}
    If $L$ is a simplicial complex residing in $\mathbb R^m$, then there exists $\epsilon=\epsilon(L)>0$ such that if $f,g:|K|\to|L|$ satisfies $|f(x)-g(x)|<\epsilon$ for any $x\in |K|$, then $f_\ast=g_\ast$.
\end{lemma}
\begin{proof}
    The set $\{\operatorname{St}_L(w):w\in L\}$ forms an open cover of $|L|$.
    So by Lebesgue number lemma, there exists $\epsilon>0$ such that each ball of radius $2\epsilon$ in $L$ lies in some $\operatorname{St}_L(w)$.
    We take $\epsilon(L)=\epsilon$.
    Let $f,g:|K|\to|L|$ as in the statement and consider the open cover of $|K|$ given by $\{f^{-1}(B_\epsilon(y)):y\in L\}$ which admits $\delta>0$ such that each $B_\delta(x)$ is contained in some member of this cover again by Lebesgue number lemma.
    This would mean that $f(B_\delta(x))\subset B_\epsilon(x)$, so $g(B_\delta(x))\subset B_{2\epsilon}(y)$.
    Choose some large $r$ such that $\operatorname{mesh}(K^{(r)})<\delta/2$, then for each vertex $v\in K^{(r)}$, the diameter of $\operatorname{St}_{K^{(r)}}(v)$ is strictly less than $\delta$, so both $f(\operatorname{St}_{K^{(r)}}(v))$ and $g(\operatorname{St}_{K^{(r)}}(v))$ are contained in some $\operatorname{St}_L(w)$.
    Set $s(v)=w$, then $s$ is a simplicial approximation to both $f$ and $g$, hence $f_\ast=s_\ast\circ \nu_{K,r}^{-1}=g_\ast$.
\end{proof}
\begin{theorem}
    If two maps $f,g:|K|\to|L|$ are homotopic, then $f_\ast=g_\ast$.
\end{theorem}
\begin{proof}
    Let $H:|K|\times I\to |L|$ be the homotopu between $f,g$, so $H(\cdot,0)=f,H(\cdot,1)=g$.
    As $|K|\times I$ is compact, $H$ is uniformly continuous.
    Thus for $\epsilon=\epsilon(L)$ as in the preceding lemma, there is some $\delta>0$ such that $|H(x,s)-H(x,t)|<\epsilon$ whenever $|s-t|<\delta$.
    Now choose $0=t_0<t_1<\ldots<t_k=1$ such that $t_i-t_{i-1}<\delta$ for any $i$ and let $f_i(x)=H(x,t_i)$.
    By construction $|f_i(x)-f_{i-1}(x)|<\epsilon$ for any $x\in |K|$, therefore $(f_i)_\ast=(f_{i-1})_\ast$ for all $i$.
    In particular, $f_\ast=(f_0)_\ast=\cdots=(f_k)_\ast=g_\ast$.
\end{proof}
\begin{corollary}
    If $|K|,|L|$ are homotopy equivalent, then $H_n(K)\cong H_n(L)$ for all $n$.
\end{corollary}
\begin{proof}
    Follows directly.
\end{proof}
\begin{definition}
    We write $H_n(X)=H_n(K)$ if $X=|K|$.
\end{definition}
    \section{Homology Calculations and Applications}
\begin{example}
    If $S^n\simeq S^m$, then $m=n$.
    This follows immediately from our calculation of the homology groups of simplices which are the triangulations of the spheres.
\end{example}
\begin{theorem}
    If $\mathbb R^n\cong\mathbb R^m$, then $m=n$.
\end{theorem}
\begin{proof}
    Suppose we are given a homeomorphism $\phi:\mathbb R^m\to\mathbb R^n$, then by a translation we can assume WLOG that $\phi(0)=0$.
    Then we can restrict $\phi$ so that $\mathbb R^n\setminus\{0\}$ and $\mathbb R^m\setminus\{0\}$, which are homotopic to $S^{n-1}$ and $S^{m-1}$ respectively, consequently $S^{n-1}\simeq S^{m-1}$, so $n=m$.
\end{proof}
\begin{proposition}
    Any map $\phi:D^n\to D^n$ has a fixed point.
\end{proposition}
\begin{proof}
    Repeat the argument as in the $D^2$ case and finish the argument by homology.
\end{proof}
\subsection{Mayer-Vietoris Theorem}
The Mayor-Vietoris theorem is an analog of Seifert-van Kampen theorem as in both deals with the algebraic invariants of spaces that are constructed by gluing.
\begin{definition}
    A sequence of homomorphisms of abelian groups
    \[
        \begin{tikzcd}
            \cdots\arrow{r}&A_{i+1}\arrow{r}{f_i}&A_i\arrow{r}{f_{i-1}}&A_{i-1}\arrow{r}&\cdots
        \end{tikzcd}
    \]
    is exact at $A_i$ if $\operatorname{Im}f_i=\ker f_{i-1}$.
    We say the sequence is exact if it is exact at every $A_i$.
\end{definition}
Consequenly, any exact sequence is a chain complex with all homology groups trivial.
\begin{definition}
    A short exact sequence is an exact sequence in the form
    \[
        \begin{tikzcd}
            0\arrow{r}&A\arrow{r}&B\arrow{r}&C\arrow{r}&0
        \end{tikzcd}
    \]
\end{definition}
So necessarily $A\to B$ is injective, $B\to C$ is surjective and $C\cong B/A$.
An exact sequence that is not short is of course called a long exact sequence.\\
Suppose $K$ is a simplicial complex with $K=L\cup M$ where $L,M$ are subcomplexes of $K$.
Then $N=L\cap M$ is also a subcomplex.
We usually write $K=L\cup_NM$.
Analogous to what we did in Seifert-van Kampen theorem, we want to relate the homology groups of $L,M,N$ to the homology group of $K$.
Of course we immediately get the natual inclusion maps
$$i:N\to L,j:N\to M,l:L\to K,m:M\to K$$
\begin{theorem}[Mayer-Vietoris]\label{mayer-vietoris}
    There exists a map (known as the connecting homomorphism) $\delta_\ast:H_n(K)\to H_{n-1}(N)$ for each $n$ such that the sequence
    \[
        \begin{tikzcd}
            \cdots\arrow{r}{\delta_\ast}&H_n(N)\arrow{r}{i_\ast\oplus j_\ast}&H_n(L)\oplus H_n(M)\arrow{r}{l_\ast-m_\ast}&H_n(K)\arrow[swap,overlay,out=0,in=180]{dll}{\delta_\ast}&\\
            &H_{n-1}(N)\arrow[swap]{r}{i_\ast\oplus j_\ast}&H_{n-1}(L)\oplus H_{n-1}(M)\arrow[swap]{r}{l_\ast-m_\ast}&H_{n-1}(K)\arrow[swap]{r}{\delta_\ast}&\cdots
        \end{tikzcd}
    \]
    is exact.
\end{theorem}
We will prove this after further development on homological algebra.
\begin{definition}
    We say a sequence of chain complexes
    \[
        \begin{tikzcd}
            \cdots\arrow{r}&A_\bullet\arrow{r}{f_\bullet}&B_\bullet\arrow{r}{g_\bullet}&C_\bullet\arrow{r}&\cdots
        \end{tikzcd}
    \]
    is exact at $B_\bullet$ if
    \[
        \begin{tikzcd}
            \cdots\arrow{r}&A_n\arrow{r}{f_n}&B_n\arrow{r}{g_n}&C_n\arrow{r}&\cdots
        \end{tikzcd}
    \]
    is exact for every $n$.
\end{definition}
So we can analogously define (short) exact sequences of chain complexes.
\begin{lemma}[Snake Lemma]
    Let
    \[
        \begin{tikzcd}
            0\arrow{r}&A_\bullet\arrow{r}{f_\bullet}&B_\bullet\arrow{r}{g_\bullet}&C_\bullet\arrow{r}&0
        \end{tikzcd}
    \]
    be a short exact sequence of chain complexes.
    Then for any $n$ there is a homomorphism $\delta_\ast:H_{n+1}(C_\bullet)\to H_n(A_\bullet)$ such that
    \[
        \begin{tikzcd}
            \cdots\arrow{r}{\delta_\ast}&H_{n+1}(A_\bullet)\arrow{r}{f_\ast}&H_{n+1}(B_\bullet)\arrow{r}{g_\ast}&H_{n+1}(C_\bullet)\arrow[swap,overlay,out=0,in=180]{dll}{\delta_\ast}&\\
            &H_n(A_\bullet)\arrow[swap]{r}{f_\ast}&H_n(B_\bullet)\arrow[swap]{r}{g_\ast}&H_n(C_\bullet)\arrow[swap]{r}{\delta_\ast}&\cdots
        \end{tikzcd}
    \]
    is exact.
\end{lemma}
\begin{proof}
    We already have this huge commutative diagram with exact rows
    \[
        \begin{tikzcd}
            &\vdots\arrow{d}&\vdots\arrow{d}&\vdots\arrow{d}&\\
            0\arrow{r}&A_{n+1}\arrow{r}{f_{n+1}}\arrow{d}{\partial_{n+1}}&B_{n+1}\arrow{r}{f_{n+1}}\arrow{d}{\partial_{n+1}}&C_{n+1}\arrow{r}\arrow{d}{\partial_{n+1}}&0\\
            0\arrow{r}&A_n\arrow{r}{f_n}\arrow{d}{\partial_n}&B_n\arrow{r}{f_n}\arrow{d}{\partial_n}&C_n\arrow{r}\arrow{d}{\partial_n}&0\\
            0\arrow{r}&A_{n-1}\arrow{r}{f_{n-1}}\arrow{d}&B_{n-1}\arrow{r}{g_{n-1}}\arrow{d}&C_{n-1}\arrow{r}\arrow{d}&0\\
            &\vdots&\vdots&\vdots&
        \end{tikzcd}
    \]
    We now construct $\delta_\ast:H_{n+1}(C_\bullet)\to H_n(A_\bullet)$.
    Take $[x]\in H_{n+1}(C_\bullet)$ for $x\in Z_{n+1}(C_\bullet)$.
    As $g_{n+1}$ is surjective, there is $x\in B_{n+1}$ such that $g_{n+1}(y)=x$.
    Now $g_n\circ\partial_{n+1}(y)=\partial_{n+1}\circ g_{n+1}(y)=\partial_{n+1}(x)=0$.
    So by exactness, there exists $z\in A_n$ such that $f_n(z)=\partial_{n+1}(y)$.
    Note that $f_{n-1}\circ\partial_n(z)=\partial_n\circ f_n(z)=\partial_n\circ\partial_{n+1}(y)=0$, so $\partial_n(z)=0$ since $f_{n-1}$ is injective.
    This means that $z\in Z_n(A_\bullet)$, so we define $\delta_\ast([x])=[z]$.
    To see it is well-defined, we have two issues to deal with:\\
    First, if we replace $x$ by $x+\partial_{n+2}(x')$ and $g_{n+2}(y')=x'$ , then replace $y$ with $y+\partial_{n+2}(y')$.
    Then $g_{n+1}(y+\partial_{n+2}(y'))=g_{n+1}(y)+\partial_{n+2}\circ g_{n+2}(y')=x+\partial_{n+2}(x')$.
    But $\partial_{n+1}(y+\partial_{n+2}(y'))=\partial_{n+1}(y)$, so $z$ does not change.\\
    Secondly, if we have made up our mind on $x$ and $y,y'$ are chosen with $g_{n+1}(y')=g_{n+1}(y)=x$, then $g_{n+1}(y'-y)=0$, therefore by exactness there is some $z'$ such that $f_{n+1}(z')=y'-y$, thus $y'=y+f_{n+1}(z')$.
    So $\partial_{n+1}(y')=\partial_{n+1}(y)+\partial_{n+1}\circ f_{n+1}(z')=\partial_{n+1}(y)+f_n\circ\partial_{n+1}(z')$, which means $f_n(z+\partial_{n+1}(z'))=\partial_{n+1}(y)+f_n\circ\partial_{n+1}(z')=\partial_{n+1}(y')$.
    This is saying that replacing $y$ by $y'$ modifies $z$ by adding $\partial_{n+1}(z')$.
    But $[z]=[z+\partial_{n+1}(z')]$, so $\delta_\ast$ is well-defined.\\
    The proof of the sequence being exact is tedious routine work.\\
    At $H_n(B_\bullet)$, if $[a]\in H_n(A_\bullet)$, then by definition $g_\ast\circ f_\ast([a])=[g_n\circ f_n(a)]=0$ by the exactness of our original short exact sequence.
    So $\operatorname{Im}f_\ast\subset \ker g_\ast$.
    Now if $g_\ast([b])=0$, then $[g_n(b)]=g_\ast([b])=0$, thus there is some $c\in C_{n+1}$ with $g_n(b)=\partial_{n+1}(c)$.
    But then choosing $b'\in B_{n+1}$ with $g_{n+1}(b')=c$ by exactness gives $g_n(b-\partial_{n+1}(b'))=0$, hence $b-\partial_{n+1}(b')\in\ker g_n=\operatorname{Im}f_n$, so there is some $a\in A_n$ such that $f_n(a)=b-\partial_{n+1}(b')$.
    In particular, $f_\ast([a])=[f_n(a)]=[b-\partial_{n+1}(b')]=[b]$, therefore $\operatorname{Im}f_\ast\supset\ker g_\ast$.
    Combining these gives $\operatorname{Im}f_\ast=\ker g_\ast$ which implies the exactness at $H_n(B_\bullet)$.\\
    At $H_n(A_\bullet)$, suppose $[z]=\delta_\ast([x])$, then $f_\ast([z])=[f_n(z)]=[\partial_{n+1}(y)]=0$ where $y$ is as in the construction of $\delta_\ast$, so $\operatorname{Im}\delta_\ast\subset\ker f_\ast$.
    Suppose now that $f_\ast([z])=0$, then $f_n(z)=\partial_{n+1}(y)$ for some $y\in B_{n+1}$.
    Take $x=g_{n+1}(y)$, then $\partial_{n+1}(x)=\partial_{n+1}\circ g_{n+1}(y)=g_n\circ\partial_{n+1}(y)=g_n\circ f_n(z)=0$, so $x$ is a cycle, so $[x]$ is a homology class.
    But $\delta_\ast([x])=[z]$ by construction, hence $\operatorname{Im}\delta_\ast\supset\ker f_\ast$, therefore $\operatorname{Im}\delta_\ast=\ker f_\ast$ which gives the exactness at $H_n(A_\bullet)$.\\
    At $H_n(C_\bullet)$, if $[x]\in\operatorname{Im}g_\ast$, then (ater replacing $x$ with another representative if necessary) WLOG there exists a cycle $y\in B_n$ such that $g_n(y)=x$.
    But then immediately $\delta_\ast([x])=0$, so $\operatorname{Im}g_\ast\subset\ker\delta_\ast$.
    Conversely, suppose $\delta_\ast([x])=0$, then let $y,z$ be the corresponding element in the construction of $\delta_\ast$.
    By hypothesis there exists $a\in A_n$ such that $\partial_n(a)=z$.
    Then $\partial_n\circ f_n(a)=f_{n-1}\circ\partial_n(a)=f_{n-1}(z)=\partial_n(y)$.
    Thus $\partial_n(y-f_n(a))=0$, so $y-f_n(a)$ is a cycle.
    Also $g_n(y-f_n(a))=x$, so $[x]=g_\ast([y-f_n(a)])$.
    Consequently $\operatorname{Im}g_\ast\supset\ker\delta_\ast$, so $\operatorname{Im}g_\ast=\ker\delta_\ast$.
    This finished the proof.
\end{proof}
\begin{proof}[Proof of Theorem \ref{mayer-vietoris}]
    Easy to check that $C_\bullet\oplus D_\bullet$ is a chain complex with all data obtained from direct sums of data of $C_\bullet,D_\bullet$.
    Also $H_n(C_\bullet\oplus D_\bullet)=H_n(C_\bullet)\oplus H_n(D_\bullet)$.
    It then suffices to check that
    \[
        \begin{tikzcd}
            0\arrow{r}&C_\bullet(N)\arrow{r}{i_\bullet\oplus j_\bullet}&C_\bullet(L)\oplus C_\bullet(M)\arrow{r}{l_\bullet-m_\bullet}&C_\bullet(K)\arrow{r}&0
        \end{tikzcd}
    \]
    is exact.
    The proof is then done by snake lemma.\\
    Note that $C_\bullet(N)$ naturally embeds into $C_\bullet (L)$ and $C_\bullet(M)$ via $i_\bullet\oplus j_\bullet$, so in particular $i_\bullet\oplus j_\bullet$ is a injective.
    Also since $K=L\cup M$, for any $c\in C_n(K)$ we can write $c=c_L+c_M$ where $c_L$ is a linear combination of simplices in $L$ and $c_M$ is that in $M$.
    Let $b_L,b_M$ be the respective copies of $c_L,c_M$ in $C_n(L),C_n(M)$ respectively, then $l_\bullet(b_L)=c_L,m_\bullet(b_M)=c_M$, and hence $c=(l_\bullet-m_\bullet)(b_L,-b_M)$, so $l_\bullet-m_\bullet$ is surjective.\\
    It remains to show the exactness in the middle.
    For $(b_L,b_M)\in C_n(L)\oplus C_n(M)$, note that $l_\bullet(b_L)-m_\bullet(b_M)=0$ iff every simplex which occurs in $b_L$ also occurs in $b_M$ with the same coefficient.
    This is just saying that $b_L,b_M$ are linear combinations of simplices in $L\cap M=N$.
    Therefore $\operatorname{Im}(i_\bullet\oplus j_\bullet)=\ker (l_\bullet-m_\bullet)$, hence the sequence is indeed exact.
\end{proof}
\begin{lemma}[Five Lemma]
    Given a commutative diagram
    \[
        \begin{tikzcd}
            A\arrow{r}\arrow{d}{\alpha}&B\arrow{r}\arrow{d}{\beta}&C\arrow{r}\arrow{d}{\gamma}&D\arrow{r}\arrow{d}{\delta}&E\arrow{d}{\epsilon}\\
            A'\arrow{r}&B'\arrow{r}&C'\arrow{r}&D'\arrow{r}&E'
        \end{tikzcd}
    \]
    If the rows are exact and $\alpha,\beta,\delta,\epsilon$ are isomorphisms, then $\gamma$ is also an isomorphism.
\end{lemma}
\begin{proof}
    Exercise.
\end{proof}
We can now prove Proposition \ref{barycentric_iso_homol}.
Recall that we want to show that $s_\ast:H_n(K')\to H_n(K)$ is an isomorphism where $s:K'\to K$ is a simplicial approximation to the identity map on $|K|=|K'|$ which sends each $\hat\sigma$ to a vertex of $\sigma$.
\begin{proof}[Proof of Proposition \ref{barycentric_iso_homol}]
    Induction on the number of simplices of $K$.
    If $K$ has only one simplex, then it is just a vertex, so the proposition is trivial.
    For the induction step, let $\sigma\in K$ be of maximal dimension, then $L=K\setminus \{\sigma\}$ is also a simplicial complex.
    Let $M$ be the simplicial complex consists of $\sigma$ and its faces.
    Then $N=M\cap L$ is just the proper faces of $\sigma$.
    By construction of $s$ we can restrict $s$ to $L',M',N'$ to get the maps
    $$s_\ast:H_n(L')\to H_n(L),s_\ast:H_n(M')\to H_n(M),s_\ast:H_n(N')\to H_n(N)$$
    By the induction hypothesis, these are all isomorphisms.
    Then by Mayer-Vietoris, we have the diagram
    \[
        \begin{tikzcd}[column sep=0.34em]
            H_n(N')\arrow{r}\arrow{d}{s_\ast}&H_n(L')\oplus H_n(M')\arrow{r}\arrow{d}{s_\ast\oplus s_\ast}&H_n(K')\arrow{r}\arrow{d}{s_\ast}&H_{n-1}(N')\arrow{r}\arrow{d}{s_\ast}&H_{n-1}(L')\oplus H_{n-1}(M')\arrow{d}{s_\ast\oplus s_\ast}\\
            H_n(N)\arrow{r}&H_n(L)\oplus H_n(M)\arrow{r}&H_n(K)\arrow{r}&H_{n-1}(N)\arrow{r}&H_{n-1}(L)\oplus H_{n-1}(M)
        \end{tikzcd}
    \]
    with exact rows.
    Easy to check that it commutes.
    Also, by the induction hypothesis, all vertical arrows except $s_\ast:H_n(K')\to H_n(K)$ are isomorphisms.
    Hence $s_\ast:H_n(K')\to H_n(K)$ is an isomorphism by five lemma.
\end{proof}
\subsection{Homology of Compact Surfaces}
Recall that we constructed the oriented surfaces of genus $g$ as $\Sigma_g=\Gamma_{2g}\cup_{\rho_g}D^2$ where $\Gamma_{2g}$ is a bouquet of $2g$ copies of $S^1$ with generators $\alpha_1,\ldots,\alpha_g,\beta_1,\ldots,\beta_g$ and $\rho_g=\alpha_1\beta_1\alpha_1^{-1}\beta_1^{-1}\cdots \alpha_g\beta_g\alpha_g^{-1}\beta_g^{-1}$.
\begin{example}
    $\Gamma_r$ can be triangulated in the obvious way by taking it as $r$ hollow triangles joined by a common vertex.
    We know that $H_n(\Gamma_1)$ is $\mathbb Z$ when $n=0,1$ and $0$ otherwise.
    We claim that
    $$H_i(\Gamma_r)=\begin{cases}
        \mathbb Z\text{, if $i=0$}\\
        \mathbb Z^r\text{, if $i=1$}\\
        0\text{, otherwise}
    \end{cases}$$
    which we shall show inductively.
    The cases except for $i=1$ are all trivial, so it suffices to show that.
    Suppose we have shown the case for $r-1$, then take $K=\Gamma_r$, $L$ be a natural copy of $\Gamma_{r-1}$ in $\Gamma_r$ and $M$ be the remaining triangle.
    Then $N=L\cap M=\{\ast\}$ and hence Mayer-Vietoris gives the exact sequence
    \[
        \begin{tikzcd}
            0\arrow{r}&H_1(N)\arrow{r}&H_1(\Gamma_{r-1})\oplus H_1(S^1)\arrow{r}& H_1(\Gamma_r)\arrow[overlay, out=0, in=180]{dll}&\\
            &H_0(N)\arrow{r}&H_0(\Gamma_{r-1})\oplus H_0(S^1)\arrow{r}& H_0(\Gamma_r)\arrow{r}&0
        \end{tikzcd}
    \]
    which, after putting in everything we already know,
    \[
        \begin{tikzcd}
            0\arrow{r}&0\arrow{r}&\mathbb Z^{r-1}\oplus \mathbb Z\arrow{r}& H_1(\Gamma_r)\arrow[overlay, out=0, in=180]{dll}&\\
            &\mathbb Z\arrow{r}&\mathbb Z\oplus \mathbb Z\arrow{r}& \mathbb Z\arrow{r}&0
        \end{tikzcd}
    \]
    By exactness of the bottom row, the map $\mathbb Z\to\mathbb Z\oplus\mathbb Z$ is injective (alternatively this can also be easily seen from the definition of that map), hence the connecting homomophism is the zero map.
    Therefore the top row reduces to the exact sequence
    \[
        \begin{tikzcd}
            0\arrow{r}&\mathbb Z^{r-1}\oplus \mathbb Z\arrow{r}& H_1(\Gamma_r)\arrow{r}&0
        \end{tikzcd}
    \]
    which precisely means that $H_1(\Gamma_r)\cong\mathbb Z^{r-1}\oplus \mathbb Z\cong\mathbb Z^r$.
\end{example}
Worth noting that $H_1(\Gamma_r)$ is the free abelian group on $r$ letters while $\pi_1(\Gamma_r)$ is the free group on $r$ letters.
Also, if $\alpha_1,\ldots,\alpha_r$ are the generating paths of the circles, then $H_1(\Gamma_r)$ is generated by $\alpha_1,\ldots,\alpha_r$.
\begin{remark}
    Like in the preceding exacmple, whenever $N$ is connected, the map $H_0(N)\to H_0(L)\oplus H_0(M)$ is injective, and hence $H_1(L)\oplus H_1(M)\to H_1(K)$ is surjective.
\end{remark}
Now we want to attach our $2$-cell to get $\Sigma_g$.
We'll do this in two steps.\\
First, we attach a cylinder $S^1\times I$ via the same $\rho_g$ but on $S^1\times\{0\}$.
Write $\Sigma_g^\star=\Gamma_{2g}\cup_{\rho_g} (S^1\times I)$
By shrinking $I$ to a point we can obtain the deformation retraction of $\Sigma_g^\star$ onto $\Gamma_{2g}$, therefore they have the same homology groups.\\
Now $\Sigma_g=\Sigma_g^\star\cup_\alpha D^2$ where $\alpha:\partial D^2\to S^1\times \{1\}$ is the natural identity.
We can choose triangulations of $\Sigma_g^\star$ and $D^2$ so that they are compatible under this gluing (so we are essentially removing a $D^2$ from $\Sigma_g$ to get $\Sigma_g^\star$).
\footnote{There are of course some technical details regarding why we can do this, but we are not going into this part of details here. It is not too hard though.}
Let $L$ be the triangulation of $\Sigma_g^\star$ and $M$ be the triangulation of $D^2$ such that this holds.
Then $N=L\cap M$ is a triangulation of $S^1\times\{1\}\cong S^1$.
Majer-Vietoris then gives the exact sequence
\[
    \begin{tikzcd}
        &H_2(L)\oplus H_2(M)\arrow{r}&H_2(\Sigma_g)\arrow[overlay, out=0,in=180]{dll}&\\
        H_1(N)\arrow[swap]{r}{i_\ast}&H_1(L)\oplus H_1(M)\arrow{r}&H_1(\Sigma_g)\arrow{r}&0
    \end{tikzcd}
\]
Note that the zero at the end is due to the preceding remark.
Again putting everything we already know into it gives
\[
    \begin{tikzcd}
        0\arrow{r}&H_2(\Sigma_g)\arrow{r}&\mathbb Z\arrow{r}{i_\ast}&\mathbb Z^{2g}\arrow{r}&H_1(\Sigma_g)\arrow{r}&0
    \end{tikzcd}
\]
The exactness means that $H_2(\Sigma_g)\cong\ker i_\ast$ and $H_1(\Sigma_g)\cong\operatorname{coker}i_\ast=\mathbb Z^{2g}/\operatorname{Im}i_\ast$.
So we just need to understand $i_\ast:\mathbb Z\to\mathbb Z^{2g}$, or in other words $i_\ast:H_1(S^1)\to H_1(\Sigma_g^\star)$.
Note that the generator of $H_1(S^1)$ is the cycle runs through it (which is basically the one corresponding to the generator of $\pi_1(S^1)$).
The deformation retract of $\Sigma_g^\star$ to $\Gamma_{2g}$ identifies $S^1$ with the image of the cycle under $\rho_g$.
Therefore $(\rho_g)_\ast$ maps the cycle to $[\alpha_1]+[\beta_1]-[\alpha_1]-[\beta_1]+\cdots+[\alpha_g]+[\beta_g]-[\alpha_g]-[\beta_g]=0$.
Therefore $i_\ast=0$ and hence $H_2(\Sigma_g)\cong\mathbb Z$ and $H_1(\Sigma_g)\cong\mathbb Z^{2g}$.
To conclude,
$$H_n(\Sigma_g)\cong\begin{cases}
    \mathbb Z\text{, if $n=0,2$}\\
    \mathbb Z^{2g}\text{, if $n=1$}\\
    0\text{, otherwise}
\end{cases}$$
which in particular allows us to know $g$ given the homology group of some $\Sigma_g$.\\
We also have the non-orientable surfaces $S_g$ which is given by $\Gamma_{g+1}\cup_\alpha D^2$ where $\alpha=\alpha_0^2\alpha_1^2\cdots\alpha_g^2$ where $\alpha_i$ is the generator of the $i^{th}$ circle in the bouquet.
Repeating the same process gives the exact sequence
\[
    \begin{tikzcd}[row sep=tiny]
        0\arrow{r}&H_2(\Sigma_g)\arrow{r}&H_1(S^1)\arrow{r}{i_\ast}&H_1(\Gamma_{g+1})\arrow{r}&H_1(\Sigma_g)\arrow{r}&0\\
        &&\mathbb Z\arrow[equal]{u}&\mathbb Z^{g+1}\arrow[equal]{u}&
    \end{tikzcd}
\]
So again $H_2(S_g)\cong\ker i_\ast$ and $H_1(S_g)\cong\operatorname{coker}i_\ast$.
Note that $i_\ast$ maps the generator of $H_1(S^1)$ to the nonzero element $2[\alpha_0]+\cdots+2[\alpha_g]$.
Hence $\ker i_\ast=0$ and $\operatorname{coker}i_\ast\cong\mathbb Z^{g+1}/(2,\ldots,2)\mathbb Z\cong\mathbb Z^g\oplus\mathbb Z/2\mathbb Z$.
In conclusion
$$H_n(S_g)\cong\begin{cases}
    \mathbb Z\text{, if $n=0$}\\
    \mathbb Z^g\oplus\mathbb Z/2\mathbb Z\text{, if $n=1$}\\
    0\text{, otherwise}
\end{cases}$$
In particular we can tell apart $S_g$ and $\Sigma_g$ as well.
    \section{Rational Homology and Euler Characteristics}
\subsection{Rational Homology}
When we defined the $n$-chains of simplicial homology, we took it as the free abelian group, i.e. free $\mathbb Z$-module, generated by the $n$-simplices.
But there is no reason why we should just restrict ourself to $\mathbb Z$-modules.
\footnote{Well, you have the universal coefficient theorem.}
If we replace $\mathbb Z$ by $\mathbb Q$, we get a vector space.
And the homology theory on this is called the rational homology.
\begin{definition}
    Let $K$ be a simplicial complex.
    Define the $\mathbb Q$-vector space of rational $n$-chains to be the $\mathbb Q$-vector space with basis being the set of $n$-simplices of $K$.
    Denote this space by $C_n(K;\mathbb Q)$.\\
    The boundary map $\partial_n:C_n(K;\mathbb Q)\to C_{n-1}(K;\mathbb Q)$, the cycles $Z_n(K;\mathbb Q)$, the boundaries $B_n(K;\mathbb Q)$ and the homology groups $H_n(K;\mathbb Q)$ are defined exactly as before.
\end{definition}
Correspondingly, by viewing abelian groups as $\mathbb Z$-modules, we can think of the objects $C_n(K),Z_n(K),B_n(K),H_n(K)$ appeared in our original simplicial homology theory instead as $C_n(K;\mathbb Z),Z_n(K;\mathbb Z),B_n(K;\mathbb Z),H_n(K;\mathbb Z)$.
This base ring (e.g. $\mathbb Z$ or $\mathbb Q$ as we have seen) is called the coefficient of the simplicial homology.\\
Perhaps unsurprisingly, homology with coefficient in $\mathbb Q$ actually contains less information than homology with coefficients in $\mathbb Z$.
\begin{lemma}
    If $H_n(K;\mathbb Z)\cong\mathbb Z^b\oplus F$ where $F$ is a finite abelian group, then $H_n(K;\mathbb Q)=\mathbb Q^b$.
\end{lemma}
\begin{proof}
    There is a natural map $C_n(K;\mathbb Z)\to C_n(K;\mathbb Q)$ via the inclusion of $\mathbb Z$ in $\mathbb Q$.
    This induces a chain map $C_\bullet(K;\mathbb Z)\to C_\bullet(K;\mathbb Q)$ and hence a natural homomorphism $H_n(K;\mathbb Z)\to H_n(K;\mathbb Q)$.
    If $c\in Z_n(K;\mathbb Q)$, then there is an integer $m$ such that $mc\in Z_n(K;\mathbb Z)$ has integer coefficients.
    So $mc$ is in the image of the map $Z_n(K;\mathbb Z)\to Z_n(K;\mathbb Q)$.
    Now $H_n(K;\mathbb Q)\cong\mathbb Q^{b'}$ for some natural number $b'$ as it has to be a finite dimensional vector space.
    But the above argument then shows $b'\le b$.
    Let $[c_1],\ldots,[c_b]\in H_n(K;\mathbb Z)$ generate the $\mathbb Z^b$ factor in $H_n(K;\mathbb Z)$.
    Then it makes sense to talk about them as elements of $H_n(K;\mathbb Q)$ as well.
    Suppose there is $\lambda_1,\ldots,\lambda_b\in\mathbb Q$ not all zero such that $\sum_i\lambda_i[c_i]=0$ in $H_n(K;\mathbb Q)$, then there exists $c\in C_{n+1}(K;\mathbb Q)$ such that $\partial_{n+1}c=\sum_i\lambda_ic_i$.
    Pick integer $m>0$ such that $m\lambda_i$ are all integers, then $\partial_{n+1}(mc)=\sum_i(m\lambda_i)c_i$, therefore $\sum_i(m\lambda_i)[c_i]=0$ in $H_n(K;\mathbb Z)$.
    But $[c_i]$ are linearly independent in $H_n(K;\mathbb Z)$, so $m\lambda_i=0$ for all $i$, hence $\lambda_i=0$ and therefore $[c_i]$ are linearly independent in $H_n(K;\mathbb Q)$.
    This means $b'\ge b$.
    Combining the two gives $b'=b$.
\end{proof}
Consequently we cannot distinguish $\mathbb RP^2$ and a point with rational homology.
Then what is the point of having it?
Well, throwing away information isn't necessarily bad.
\subsection{Euler Characteristics}
\begin{definition}
    Let $K$ be a simplicial complex.
    The Euler characteristic of $K$ is
    $$\chi(K)=\sum_{n=0}^\infty(-1)^n\dim_{\mathbb Q}H_n(K;\mathbb Q)$$
    If $X$ is a topological space with $X=|K|$, then we write $\chi(X)=\chi(K)$ which is well-defined as the homology groups do not depend on specific triangulation.
\end{definition}
There is no issue about the convergence of the series as eventually $H_n=0$ and hence the series terminates.
\begin{lemma}\label{euler}
    $$\chi(K)=\sum_{n=0}^\infty(-1)^n\dim_{\mathbb Q}C_n(K;\mathbb Q)=\sum_{n=0}^\infty(-1)^n|\{\sigma\in K:\dim\sigma=n\}|$$
\end{lemma}
\begin{proof}
    Write $\dim=\dim_\mathbb Q$.
    Note that we have
    $$\dim H_n(K;\mathbb Q)=\dim Z_n(K;\mathbb Q)-\dim B_n(K;\mathbb Q)$$
    $$\dim C_n(K;\mathbb Q)=\dim\ker\partial_n+\dim\operatorname{Im}\partial_n=\dim B_{n-1}(K;\mathbb Q)+\dim Z_n(K;\mathbb Q)$$
    Thus we can just write
    \begin{align*}
        \sum_{n=0}^\infty(-1)^n\dim C_n(K;\mathbb Q)&=\sum_{n=0}^\infty (-1)^n\dim Z_n(K;\mathbb Q)+\sum_{n=1}^\infty (-1)^n\dim B_{n-1}(K;\mathbb Q)\\
        &=\sum_{n=0}^\infty (-1)^n(\dim Z_n(K;\mathbb Q)-\dim B_n(K;\mathbb Q))\\
        &=\sum_{n=0}^\infty(-1)^n\dim_{\mathbb Q}H_n(K;\mathbb Q)
    \end{align*}
    as desired.
\end{proof}
\begin{example}
    If $\dim K=2$, we get the familiar $\chi(K)=V-E+F$.
    So for example $\chi(S^2)=1-0+1=2$.
    Consequently any triangulation of the $2$-sphere, i.e. any polyhedron, has $V-E+F=2$.
\end{example}
\subsection{The Lefschetz Fixed-Point Theorem}
\begin{definition}
    Let $X$ be triangulable and $\phi:X\to X$.
    The Lefschetz number of $\phi$ is
    $$L(\phi)=\sum_{n=0}^\infty(-1)^n\operatorname{tr}(\phi_\ast:H_n(X;\mathbb Q)\to H_n(X;\mathbb Q))$$
\end{definition}
Again the sum is eventually zero.
\begin{example}
    $L(\operatorname{id}_X)=\chi(X)$.
\end{example}
\begin{lemma}
    If $f:K\to K$ is a simplicial map, then
    $$L(|f|)=\sum_{n=0}^\infty (-1)^n\operatorname{tr}(f_n:C_n(K;\mathbb Q)\to C_n(K;\mathbb Q))$$
\end{lemma}
\begin{proof}
    Given a commutative diagram of vector spaces
    \[
        \begin{tikzcd}
            0\arrow{r}&A\arrow{d}{\alpha}\arrow{r}&B\arrow{d}{\beta}\arrow{r}&C\arrow{d}{\gamma}\arrow{r}&0\\
            0\arrow{r}&A'\arrow{r}&B'\arrow{r}&C'\arrow{r}&0
        \end{tikzcd}
    \]
    with exact rows, then it is easy to check that $\operatorname{tr}\beta=\operatorname{tr}\alpha+\operatorname{tr}\gamma$.
    The proof follows almost immediately by using the same idea as we did in Lemma \ref{euler}.
\end{proof}
\begin{theorem}[Lefschetz Fixed-Point Theorem]
    Let $\phi:X\to X$ be a map where $X$ is triangulable.
    If $L(f)\neq 0$, then $\phi$ has a fixed point.
\end{theorem}
We can actually count the fixed point in the full version of the theorem, which is out of the scope of this course.
\begin{proof}
    We shall show that if $\phi$ has no fixed point, then $L(\phi)=0$.
    As $X$ is triangulable, it has to be compact, so there is some $\delta>0$ such that $\|x-\phi(x)\|>\delta$ for all $x\in X$.
    Choose $K$ such that $X=|K|$ and, possibly using barycentric subdivision, $\operatorname{mesh}K<\delta/2$.
    Then if $x\in\sigma\in K$, then $\phi(x)\notin\sigma$.
    Let $f:K^{(r)}\to K$ be a simplicial approximation to $\phi$.
    If $v\in K^{(r)}$ is a vertex with $v\in\sigma\in K$, then $\phi(v)\in\operatorname{St}_K(f(v))$, consequently $\|\phi(v)-f(v)\|<\delta/2$.
    But $\|\phi(v)-v\|>\delta$, so $\|v-f(v)\|>\delta/2$, so $f(v)\notin\sigma$.
    Let $i_\bullet:C_\bullet(K;\mathbb Q)\to C_\bullet(K^{(r)};\mathbb Q)$ be a chain map that induces the canonical isomorphism on homology, which is supposed to map an $n$-simplex in $K$ to the sum of the $n$-simplices in $K^{(r)}$ supported in it.
    Since $f$ takes vertices of $\sigma$ out of it, it follows that $f_n\circ i_n(\sigma)$ is supported on simplices disjoint from $\sigma$.\\
    Since $\phi_\ast$ is induced at the level of chains by $f_n\circ i_n$, we now have
    $$L(\phi)=\sum_{n=0}^\infty(-1)^n\operatorname{tr}f_n\circ i_n$$
    by the preceding lemma.
    But $i_n\circ i_n$ throws any $n$-simplex elsewhere, hence $\operatorname{tr}f_n\circ i_n=0$, so $L(\phi)=0$.
\end{proof}
\begin{corollary}
    If $X$ is triangulable and contractible, then any map $\phi:X\to X$ has a fixed point.
\end{corollary}
\begin{proof}
    We just need to show that it has nonzero Lefschetz number.
    $X\simeq\{\ast\}$, so essentially $H_n(X;\mathbb Q)=0$ if $n>0$ and $\dim H_0(X;\mathbb Q)=\mathbb Q$.
    So the only nonzero map $\phi_\ast$ is $\phi_\ast:H_0(X;\mathbb Q)\to H_0(X;\mathbb Q)$ which is the identity.
    Thus $L(\phi)=1$.
\end{proof}
\end{document}